%\documentclass[twoside,11pt,oldfontcommands,paperheight=8.5in,paperwidth=5.5in]{memoir} % Font size
\documentclass[]{bookest}

\usepackage{geometry}
%\geometry{showframe=true}
\geometry{
inner=0.75in, 
outer=0.5in, 
top=0.8in, 
bottom=0.6in, 
paperheight=8.5in, 
paperwidth=5.5in,
%textwidth=4.25in,
%textheight=7.5in,
headsep=11pt, 
footskip=0.2in
}

%%%%%%%%%%%%%%%%%%%%%%%%%%%%%%%%%%%%%%%%%%%

\newenvironment{dedication}
{
  % \cleardoublepage 
   \thispagestyle{empty}
   \vspace*{\stretch{1}}
   \hfill\begin{minipage}[t]{0.66\textwidth}
   %\begin{textit}
   \begin{flushright}
   %\raggedright
}%
{
   \end{flushright}
   %\end{textit}
   \end{minipage}
   \vspace*{\stretch{3}}
   \clearpage
}


%%%%%%%%%%%%%%%%%%%%%%%%%%%%%%%%%%%%%%%%%%%

\let\footruleskip\undefined
\usepackage{fancyhdr}
\usepackage{lastpage}
\usepackage{afterpage}
\usepackage{graphicx}

\usepackage[utf8]{inputenc} % Required for inputting international characters
\usepackage[T1]{fontenc} % Output font encoding for international characters

\usepackage[osf]{libertine} % Use the Libertine font
\usepackage{microtype} % Improves character and word spacing

\usepackage{tikz} % Required for drawing custom shapes
\definecolor[named]{color01}{rgb}{.2,.4,.6} % Color used in the title page
\definecolor[named]{myred}{RGB}{200,55,55}
\definecolor[named]{myviolet}{RGB}{200,176,209}
\usepackage{wallpaper} % Required for setting background images (title page)

%%%%%%%%%%%%%%%%%%%%%%%%%%%%%%%%%%%%%%%%%%%

\newcommand{\myparttitle}{bla-bla}

%%%%%%%%%%%%%%%%%%%%%%%%%%%%%%%%%%%%%%%%%%%

\newcommand{\mypart}[4]{
\addcontentsline{toc}{chapter}{#1}
\renewcommand{\myparttitle}{#1}
\thispagestyle{fancy}
\fancyhf{}
\renewcommand{\headrulewidth}{0in}
\vspace*{2in}
\begin{center}
{\Huge{\textbf{#2}}}\\
\vskip20pt
\includegraphics[height=30pt]{Pics/#3}
\end{center}
\vskip20pt
\hrule
\vskip14pt
#4
\vskip14pt
\hrule
\restoregeometry
\newpage
}

%%%%%%%%%%%%%%%%%%%%%%%%%%%%%%%%%%%%%%%%%%%

\newcommand{\card}[5]{
\addcontentsline{toc}{section}{#3}
\thispagestyle{fancy}
\fancyhf{}
\renewcommand{\headrulewidth}{0pt}
\lhead{\thepage \hskip14pt Physics Is My Favorite Game}
\fancyfoot{}
\begin{minipage}{0.05\textwidth}
\includegraphics[height=20pt]{Pics/#1}
\end{minipage}
\ifthenelse{\equal{#1}{Qr.png} \OR \equal{#1}{Qb.png} \OR \equal{#1}{10r.png} \OR \equal{#1}{10b.png}}{\hspace{3pt}}{\ifthenelse{\equal{#1}{Ar.png} \OR \equal{#1}{Ab.png}}{\hspace{0pt}}{\hspace{-5pt}}}
\begin{minipage}{0.05\textwidth}
\includegraphics[height=20pt]{Pics/#2}
\end{minipage}
\hspace{\fill}
\begin{minipage}{0.78\textwidth}
%\hspace{12pt}
\huge{\textbf{#3}}
\end{minipage}
\vskip12pt
#4
\newpage

\thispagestyle{fancy}
\fancyhf{}
\rhead{\myparttitle \hskip14pt \thepage}

%\cfoot{\thepage}
\vspace*{0.2in}

%\begin{figure}[t]
\begin{center}
\includegraphics[width=\textwidth]{Cards/#5}
\end{center}
%\end{figure}
\newpage
}

%%%%%%%%%%%%%%%%%%%%%%%%%%%%%%%%%%%%%%%%%%%

\newcommand{\joker}[3]{
\addcontentsline{toc}{section}{#1}
\thispagestyle{fancy}
\fancyhf{}
\renewcommand{\headrulewidth}{0pt}
\lhead{\thepage \hskip14pt Physics Is My Favorite Game}
\fancyfoot{}
%\begin{minipage}{0.78\textwidth}
%\hspace{12pt}
{\huge{\textbf{#1}}}
%\end{minipage}
\vskip12pt
#2
\newpage

\thispagestyle{fancy}
\fancyhf{}
\rhead{Jokers \hskip14pt \thepage}

%\cfoot{\thepage}
\vspace*{0.2in}

%\begin{figure}[t]
\begin{center}
\includegraphics[width=\textwidth]{Cards/#3}
\end{center}
%\end{figure}
\newpage
}

\title{Grimms' Fairy Tales} % Book title
\author{The Brothers Grimm} % Author
\newcommand{\edition}{Second Edition} % Book edition

%----------------------------------------------------------------------------------------

\begin{document}

{\huge{\textbf{Introduction}}}
\vskip12pt

Hi, reader! My name is Mikhail Stolpovskiy, and I am a PhD in physics. During my scientific career, I worked in the fields of particle physics, cosmology, and astrophysics. And during the last  years, I also engaged myself as a science populariser, writing articles for the web journals. In addition, I am slightly talented in drawing. So I decided to fuse  my diverse competencies and create this deck of playing cards, where I dedicate each of the suits to one of the four major fields of research in modern physics: quantum mechanics, particle physics, astrophysics, and cosmology.

Why cards? That was an idea that came to me at one conference some years ago. All the days were tightly packed with the reports about the frontiers of  modern physics, and in the evenings we amused ourselves with playing card games. No surprise that on the last day of the conference these two have merged into something funny and, I hope, useful.

I study physics for 16 years, and I learned very well the first commandment of learning: practice makes perfect. And it is especially important for such a challenging field as physics. When you play with my cards, you effectively recall all these subjects. And it creates an overall vision of modern physics in your head. When I was a student at the University, I have to admit I was a lousy student. I was bored with physics because I didn't understand it. The crucial point is that I had not this vision of beautiful interconnection of all the different areas of this marvellous science. I didn't have a broad picture. Now I want to fix this for the future students, and to all people interested in physics.

On each page of this book, you will find a short explanation of a card. I tried to design it in such a way that the low-value cards are the most basic notions, the jacks, queens and kings are the scientists that created the foundation of this science, and the aces summarize the most important ideas. The high-value cards, from 8's to 10's, describe the very edge of the current knowledge and prospects.

I start with the small scales of elementary particles to the large sizes, finishing with the whole Universe. In first glance, it looks very patchy. However, on every card, you will find a lot of links between the different parts of the book (I give them in the parentheses, like for example (2$\heartsuit$) is the link to the gauge theory). And the two jokers describe two competing theories of everything, which aim to give a single description to the whole physics.

I am particularly proud to have an opportunity to emphasize the role of women in physics with the queen cards. They represent four brilliant female scientists who made truly ground-breaking discoveries. I will be especially proud if some girl will choose the fundamental research as her future specialisation thanks to my cards. The historically set under-presence of women in science is a stopping factor, and who knows how much more we will discover if we accept the diversity!

With this deck, you can play all the usual card games. Or you can play unique physics games. Here are a couple of suggestions. The physics poker: instead of finding the combinations of values you search for the combinations of meanings. For example, Plank (K$\diamondsuit$) links to the cosmic microwave background (3$\spadesuit$). Looking at the anisotropies of the latter (6$\spadesuit$), we hope to study the inflation epoch (9$\spadesuit$). Who knows, maybe the inflation is explained by the Higgs field (A$\heartsuit$). Thus you get five interconnected cards, so it is a flush! After the usual turn of betting, when you open your cards, you have to justify that you got a flush because it is maybe not so visible to other players. So to win you need not only be good at poker but in physics too!

Another suggestion is the physics cheat game. Deal the cards between the players. On your turn, you put some number of cards on the table and say: I know all these cards. The next player can trust you or not. In the first case s/he can turn face up your cards, pick any of them and ask you about its meaning. If you answer correctly, your opponent takes all the cards on the table. Otherwise, it is you who grab them. However, if the other player trusts you, s/he can continue the turn by adding more cards on the pile saying that s/he knows all of them, and the next player in the circle will choose to trust this or not. The goal of the game is to get rid of all your cards.

This book doesn't pretend to be a complete description of the discussed topics. It is instead a mere introduction. At the end of each suit, you'll find a list of suggested materials to dig down. That will give you a solid background in the subject. But I haven't seen so far anything that would aim to provide a global overview of the modern fundamental physics. That is the primary goal of this book and this deck. So use it as a map in your journey to the world of physics. And farewell!

\newpage

\mypart{Quantum Mechanics -- Diamonds}{Quantum Mechanics \\Diamonds}{diamond.png}{Quantum mechanics is the part of physics that works with the very tiny particles. In everyday life, we used to see material objects, for example, this book. And we intuitively know that if you grind it down, you'll have a handful of dust, where every little particle consists of the same paper as the entire book. Here is where our intuition lets us down. The tiny constituents of the usual matter behave utterly different from ordinary things. To see these effects you need to make really fine dust, about a million times smaller than the smallest dust particle, but what you'll be able to see will blow your mind!}

%------------------------------------------------------------------------------------

\card{Kr.png}{diamond.png}{Max Planck -- the black body spectrum}{
The black body is a physical term meaning a body that accumulates all the light, emitted towards it. Typically such a body should also emit light. The dependence of the intensity of this light on its wavelength (wavelength is the distance between two subsequent peaks of the wave) could be theoretically predicted. Such dependence is called the spectrum. And according to the calculations of XIX century physicists, this theoretically predicted spectrum was dramatically different from the experimentally observed one.

Although the problem sounds quite abstruse, it is very practical. Any hot body emits light. For example, a red-hot steel billet on a factory glows in the black body spectrum. Your body glows too, though you don't see it. The reason why you don't see it is that the spectrum of the black body falls steeply on short wavelengths, and the visible light has very short waves. However, the theoretical model predicts the opposite: the intensity should increase to infinity on the short wavelengths. So every object around you must be glowing very bright, according to the theory (this problem was called \textit{the ultra-violet catastrophe}).

And what was this theory? It is the theory, developed 100 years before by Thomas Young, that tells that the light is a wave. Right. That's what we told -- the light, the wavelength... And then came Max Planck (1858 - 1947) who in 1900 solved the long-standing mystery with the black body spectrum. He postulated that the light waves are \textit{quantum}, which means they are particles. Look, here comes the only formula in our book:
$$ E = n h \nu , $$
where $E$ is the energy of light, $h$ is just a constant, $n$ is the number of quanta and $\nu$ (it is a greek letter, called \textit{nu}) is the frequency of light, which is directly related to the wavelength through the speed of the wave. The constant $h$ was later called the Planck constant. It is one of the fundamental constants of Nature. So here is the frequency $\nu$, which is a wave characteristic. But there is also $n$ for the number of... what? Waves? But you can't count waves! A wave is a continuous object! You can count only particles!

So this equation implies that the light is at the same time a wave and a particle. This self-contradictory Planck's equation and his idea of quantization of light laid the foundation of the totally new quantum physics. The odd combination of wave and particle natures, introduced in quantum mechanics, is called \textit{wave-particle duality}.
}{KDiamonds.jpeg}

%------------------------------------------------------------------------------------

\card{2r.png}{diamond.png}{The photoelectric effect}{
At the end of the XIX century, many physicists thought that physics is over. There will be no more big discoveries. The overall theory is already developed. Sometimes professors in the Universities even told students to do not choose physics, because there’s nothing left to do. There were only a few unsolved problems. One of them we already discussed, it’s the black body spectrum (K$\diamondsuit$). And it brought a surprisingly new concept to the field -- the quantization. Nevertheless, explanation of the black body spectrum was somehow indirect evidence for the quantization.

Another unexplained phenomenon was the photoelectric effect. When you shine on a metallic surface, you can detect electrons, freed from the metal by the action of light. Now, remember that at that time, the light was thought to have the wave nature. If it is so, then you should expect that the more intense light will create more energetic electrons, right? That would be logical, because more intense light means more energy deposited on the surface, hence the electrons having more energy.

The photoelectric effect was studied carefully and it has been found that it acts just in the opposite way: the energy of the freed electrons doesn’t depend on the intensity of the light, but it does depend on its color.

It was Albert Einstein (K$\spadesuit$) who in 1905 proposed to apply the Planck’s equation (K$\diamondsuit$) for the photoelectric effect. And it fitted to this problem perfectly: the predicted by the quantum theory behaviour of the freed electrons was exactly the same, as the observed in the experiment. Thus Einstein has given the solid proof to the revolution in physics, started by Planck.
Let’s see what is the photoelectric effect, according to Einstein. It is the phenomenon of kicking out the electrons from the metal when it is exposed to light. The particles of light -- photons -- act here like the little pellets, which pass their momentum to the electrons of the material. Thus the resulting speed of the freed electrons depends on the energy of each photon and it doesn’t depend on the number of the photons. Note also, that the photon energy, in its turn, depends on its frequency. Thus again, we see these strange quantum particles with a wavy characteristic - frequency, which we first encountered when discussed the Planck’s equation.

It happened that Einstein was way more confident in the idea of quantization than its actual farther Planck. While Planck first tried to avoid quantization, thinking that it is just a kind of a mathematical trick that hides some deeper physics behind, Einstein immediately gripped this theory and promoted it as much as he could. He recognized that the mixed wave-particle nature of microscopic particles is the accurate description of the World.

}{2Diamonds.png}

%------------------------------------------------------------------------------------

\card{Jr.png}{diamond.png}{Werner Heisenberg --\\ the uncertainty principle}{
Remember, when we introduced the quantization (K$\diamondsuit$) we told that the waves could not be counted. You might be asking yourself -- why is it? The waves breaking on the shore are easy to count. One wave after another, it's two, and so on.

You are partially right in your suspect. Let's clarify: the wave in physics is considered in an idealistic approach. When a physicist talks about a wave, he thinks about an infinite sequence of equal bumps and valleys. In everyday life, we call each bump a wave. But in physics, it is different. Can we localize this physical wave? Yes and no. No, because the wave is infinite, hence it has no specific location. But if we start to add one wave to another, we can make such a combination of waves with different wavelengths that the infinite wave will transform to a single well-localized bump.

Now let's remember that we speak about the waves of quantum objects. Let's take a photon with the given energy. According to Planck's equation, when we define the energy of a photon, we set its wavelength. It is a single, well-defined wavelength, hence the single, well-defined energy. So far, everything is fine. We are very well accustomed to well-defined things.

But let me ask you, where is this photon? As we said before, it cannot have any specific location, because it is a wave. In quantum physics, we say that we are completely uncertain about the position of the photon.

What would happen if we start summing up different wavelengths to obtain a single bump? If we succeed, then the position of the photon will be well defined. But what would happen with its energy? Since we added up multiple wavelengths -- and to obtain a clear bump we need an infinite number of different wavelengths -- we cannot say which wavelength has this particular photon. Hence we can not say anything about its energy. We found ourselves in the opposite position: we defined the location of the photon, but we are completely uncertain about its energy.

It is interesting to specify the role of the observer. When you measure the position of the particle, you do it by interacting with it. By measuring the particle position, you change the particle, so its energy becomes uncertain because of the act of measurement. The same thing happens with the energy measurement. It is the fundamental principle of quantum experiments that you can never have complete knowledge of the properties of your system.

It was Werner Heisenberg who first described all this mess in his famous uncertainty principle. It is one of the essential things in quantum mechanics.
}{JDiamonds.png}

%------------------------------------------------------------------------------------

\card{3r.png}{diamond.png}{The wave function}{
How can we describe the behaviour of quantum particles? To do it, we use the so-called wave function formalism.

In the framework of classical mechanics, where everything is certain, we can identify the position of each particle in each moment. In quantum mechanics it is impossible. However, we can determine the probability for a particle to be in a specific position. It means that we cannot know the location of the particle until we measure it. The probability of finding a particle in a given position is defined by the wave function.

Here we implicitly introduced the very important concept of measurement. In quantum mechanics, every property of the particle remains uncertain until measured. If we try to measure two quantities simultaneously, we are limited in the precision by the uncertainty principle. Each measurement changes the particle state to make the measured property sure and other properties -- uncertain. So far, there is no mystery: we measure the state of the particle by interacting with it, so it is not wholly surprising that we change the particle by the act of measurement.

However, it has some peculiar consequences. Let's imagine the experiment: the quantum particles fly one by one towards the screen with two slits. Since the position of each particle is uncertain, the particle looks like the wave, which passes through both slits simultaneously. The waves create the interference pattern on the sensitive plate beyond the screen (interference is the effect of summation of the waves from the two slits, which creates the striped pattern on the plate).

But what would happen if we place a detector in front of the slits to identify through which of the slits the particle went through? Then the position of the particle becomes certain through the act of measurement, hence the particle looks less like the wave, and you won't observe the interference pattern on the sensitive plate.

But how the particle could know that we will measure it and it should not be wavy and try to go through both slits at the same time? The answer is that the position of the particle is described in a completely non-classical way. One cannot say that the particle is passing through one slit or another. Neither one can say that it's going through two slits simultaneously. The particle behaviour is quantum, meaning that it is described through the wave function.

If you read a bit about the wave function and especially about the double-slit experiment, you'll find a curious notion of the wave function collapse. This is exactly what we were talking about: the phenomenon of the sudden appearance of certainty in the behaviour of the quantum particle.
}{3Diamonds.png}

%------------------------------------------------------------------------------------

\card{4r.png}{diamond.png}{Bohr's atom}{
The quantum mechanics, which describes each particle as a wave and a particle in the same time, is very successful in explaining some phenomena with the free particles, for example, the black body spectrum (K$\diamondsuit$) or the photoelectric effect (2$\diamondsuit$). But can it predict the behaviour of such a complex system as an atom? Yes, it can. Let's see how it works on the simple example of the hydrogen atom (a bound system of a heavy positively charged proton and a light negatively charged electron).

Let's conduct a little experiment. Take a guitar (if you don't have a guitar,  use your imagination) and pick an open string. You will hear the sound with the frequency $\nu$. Now put the finger on one of the strings just above the twelfth fret, don't push it down and pick the string again, releasing the finger at the same moment. This technique is called the flageolet. If you did everything right, you should hear the sound one octave higher -- it's frequency is $2\nu$.

What happened? You made the same string to sound at two distinguish frequencies, which are defined by its tension and length. High frequency means faster trembling of the string. Hence, it gets more energy, when you pick it with a flageolet than if you pick the open string. It means that the string can have two distinct energy states.

You can make flageolets also on the seventh and fifth frets. They will give the frequencies $3\nu$ and $4\nu$ correspondingly. But the higher flageolet you take, the harder it is. In other words, the high energy states of the string are not stable.

The string has these discrete energy states, because of the wave nature of its trembling. The atom is just the same: since the electron has the wave nature, it can take only some discrete number of energy states in the atom. It is traditionally illustrated as if the electron as a particle rotates on different orbits around the atom nucleus (I adopted this tradition for the picture on the card). Still, in reality, the electron in the atom is a wave, which can take some number of discrete energy states. When the electron passes from the high energy state to the low one, it releases the extra energy in the form of a photon. The high energy states of the atom are not stable, so the atom tends to return to the low energy state.

Historically, it was first understood that the atom has a dense positive nucleus and electrons are flying around it. And such a system seemed quite tricky: according to all the known laws of physics the electrons had to fall very fast on the nuclei. Only the creation of quantum theory helped to unravel this scientific mystery.}{4Diamonds.png}

%------------------------------------------------------------------------------------

\card{Qr.png}{diamond.png}{Marie Curie -- radioactivity}{
Marie Curie is one of the most famous scientists of all the times, and certainly the most prominent scientific woman. She was the first-ever person who received two Nobel Prizes, and several other ``first-ever" too. Curie worked on the topic of radioactivity. Let's see what made her work so important to us.

The term radioactivity describes the radiations of different nature which appear when the atom nucleus decays. After that, the atom often becomes a different chemical element. The radioactivity has many practical applications. It is used to treat cancer, to date archaeological findings, to produce energy in the nuclear reactor and many other fields. Now let's consider how the quantum nature of microscopic particles rules the radioactivity.

Remember, we told that the wave function (3$\diamondsuit$) doesn't corner the exact state of the particle, but describes only the probability for the particle to be in some state. This probability is linked to the fundamental uncertainty principle (J$\diamondsuit$), which means that quantum mechanics makes the realisation of a truly random process.

Random process? What's that? To explain, let's see what is not random. When you throw a dice, the resulting score is not accidental: it is well-defined by the way you've thrown the dice. Every coincidence in our life is not random. There is almost no truly random processes in our life -- everything is well-defined by the previous events. Many processes are indeed practically impossible to predict. But even though they are not random: if you track down every little motion that constitutes the complicated process, you'll see that everything in the process was quite certain.

However, the quantum mechanical processes are genuinely random. It means that there is no reasoning for the given particle to get this or that final state. The particle ``chooses'' its state at random.

The radioactivity is a quantum process. That is the nuclei behaviour described by the nuclear wave function. Hence the decay of the nuclei occurs at random.

The deep understanding of the physics of radioactivity became one of the cornerstones of the quantum mechanics and turned it to the new level: remember that the quantum mechanics started with free massless photons (K\&2$\diamondsuit$). Then we considered the behaviour of electrons in an atom (4$\diamondsuit$). Now it turned to the much more complex system, the atom nuclei.
}{QDiamonds.png}

%------------------------------------------------------------------------------------

\card{Ar.png}{diamond.png}{Schr\"odinger's cat}{
Now we have introduced all the ingredients to sort out the famous Schr\"odinger's cat paradox. Let's recall: the behaviour of the quantum objects is defined by the wave function (3$\diamondsuit$), which makes it behave randomly (Q$\diamondsuit$): the result of the quantum process appears random after the measurement. Otherwise, we have to assume that the quantum object takes all the possible states at once. It is probably fine for you to think that all this mess appears only on the microscopic scales and doesn't touch you directly. But actually, things are not so clear.

Erwin Schr\"odinger, whose the main contribution to the physics was the writing of the equation, which describes the transformation of the wave function in time, presented the following paradox. Let's imagine a box, which contains an ampule with poison. The ampule can be broken by the mechanism, which is triggered by the radioactive decay of an atom. Let's also imagine that the probability for the atom to decay is 50\%: after the finish of the experiment, there is 50\% chance to find the atom decayed and hence the poison ampule was broken, and 50\% chance to see everything in its place. 

Now let's put a cat in this box. What is the state of the radioactive atom, while the box is closed? According to quantum mechanics, it takes both possible states at once: decayed and not decayed one. But it means that the ampule with poison is both broken and unbroken too! Hence the cat is dead and alive at the same time. And the cat, which is quantum here, takes the specific state -- either dead or alive -- only after the measurement. That is, only when you open the box.

The Schr\"odinger's paradox is not just some theoretical construction, which you can neglect. It reveals the deep problem: we are living in the world, where each tiny particle is quantum. And at the same time we, people, seem to be not quantum at all. It's not quantum physics which is strange. Instead, it is weird that we, who are built of quantum particles, are not quantum too! The paradox with the cat is all about it. 

To solve this paradox, we introduce different ``interpretations'' of quantum mechanics. If you believe that the wave function of radioactive atom collapses when its radiation interacts with the detector, then you are an adept of the Copenhagen interpretation. If by contrary you think that cat shares its wave function with the atom and the cat is indeed dead-and-alive until you open the box, you believe in many world interpretation. Unfortunately, I ran out of place on this page. So try to study some of the materials suggested at the end of the diamond suit.
}{ADiamonds.png}

%------------------------------------------------------------------------------------

\card{5r.png}{diamond.png}{Isotope stability}{
We have introduced the basics of quantum mechanics. Now we are stepping farther and start exploring the current research activities in this area. And the first thing to learn is the isotope stability. This subject is directly related to the work of Marie Curie (Q$\diamondsuit$) because the instability of the isotopes implies the radioactive decay of them. Let's see how it works precisely.

We've seen how the atom of hydrogen is constructed (4$\diamondsuit$). But hydrogen, the lightest chemical element, is the simplest example of an atom. The nucleus in its centre consists of just single proton. Hence it can hold only one electron: the negatively charged electron attracts to the positively charged nucleus. To obtain a different chemical element, we need to put more electrons in it. It is possible only by increasing the charge of the nucleus. But you can't do it by just adding more protons because the equally charged protons push each other with electric force. To bind them together we need neutrons -- particles with no electric charge, otherwise pretty similar to protons.

You may ask how it works because the negatively charged neutrons don't electrically attract protons. To fully answer this question, you should glance forward in this book and learn a bit about the exchanging interactions (2$\heartsuit$) and strong force (5$\heartsuit$). In a few words, the binding force between protons and neutrons in the nucleus is created by the exchange of $\pi$-mesons, which are created by the strong interaction. This $\pi$-meson interaction doesn't work between the neutrons. Thus we have to balance between the nuclei with too many protons and the nuclei with too many neutrons. The first ones decay, because they cannot hold the protons, which push each other away. The second ones decay too because they miss the binding $\pi$-mesons-guarded force. The different sets with various numbers of protons and neutrons are called isotopes. Some of them are well balanced, so they are stable. Some are not -- they decay and do it faster as they are farther from the stable state.

Nowadays, many laboratories around the globe work on the creation and study of the new heavy elements. It is one of the hottest topics in modern physics, which promises some significant discoveries in the upcoming years. For example, the so-called island of stability: the protons and neutrons stay on the energy levels in the nucleus, similar to the electrons in the atom (4$\diamondsuit$). Energy levels on their turn form the \textit{shells}. The nucleus with a filled shell is way more stable than other configurations. Scientists hope to find stable elements with full shell and a very high number of constituent protons and neutrons. Such elements will be separated from the main sequence of stability (depicted on the card) by a gap, so they stay on an ``island'' of stability.
}{5Diamonds.png}

%------------------------------------------------------------------------------------

\card{6r.png}{diamond.png}{Laser}{
Another interesting topic is the laser. Did you know that the word laser is actually an acronym? It means ``light amplification by stimulated emission radiation''. Nowadays lasers are used everywhere, from laser pointers to advanced applications in surgery. The laser is the only way to produce the very intense, well-controlled light beam. The modern scientific lasers are used to turn the matter to the new extreme states and to push the border of our knowledge in physics.

When we studied the atom (4$\diamondsuit$), we told that the electrons in the atom could be in various energy states. The specific set of possible energy states depends on the charge of the nucleus. The nucleus charge, on its turn, defines the chemical elements. Hence all the atoms of some chemical element have the same energy states. We also told that when the electron passes from the high energy state to the low one, it releases the extra energy in the form of a photon, that is it emits the light. Since the energy states are the same for all the atoms of a certain chemical element, all such atoms emit light on the same well-defined frequency. In fact, it is a set of frequencies, because there are many possible transitions between the energy states in the atom.

The decay of an excited atom (in high energy state) occurs at random, as any quantum process. However, we can push the atom to decay. As we said, the high energy states are not stable. Which means an excited atom decays if you only push it a bit.

In laser, you do it by the light, emitted by the other atoms in the same system. It goes like this: one atom decays, the radiated photon pushes another atom to decay too. On the next step two atoms decay, then four, eight, and so on. The number of photons grows like a snowball. When the bunch of light releases from the laser, it consists of identical photons, flying in the same direction. This characteristic of laser light is crucially important for numerous applications.

Finally, let's mention how do we excite atoms. It is pretty simple. All the atoms undergo the thermal movement. When you heat the material, this movement becomes more intensive. At some point, the atoms move so fast, that when they hit each other, they become excited. That's it. You simply need to heat the body of the laser a bit, and it will start to shine on its own.
}{6Diamonds.png}

%------------------------------------------------------------------------------------

\card{7r.png}{diamond.png}{Superconductivity}{
Another funny effect of quantum mechanics is the superconductivity. Did you know that every electric device suffers from the resistance of the conductive materials to the electric current? When the electrons pass by in the wires, they get bumped by the atoms and lose their energy. So when you pay your bill for the electricity, there is always a part to cover this loose.

To reduce the electric resistance, the manufacturers use different materials for the wires. Coper, for example, is cheap and has pretty low resistance. Silver and gold are known for their excellently low resistance too. But quantum mechanics proposes the ultimate solution to the problem.

If you cool down the conductive material, the thermal motion of the atoms reduces. At some temperature, lower than some critical temperature (Tc), when the atoms move very slow, an interesting effect occurs. The electrons start to form so-called Cooper pairs. The size of such a couple is larger than the typical distance between the atoms. That means that the coupled electrons begin to move like a kind of cloud through the material. It is a purely quantum phenomenon, which involves all the quantum mechanical basics discussed before. In the warm material, the fast-moving atoms easily break the Cooper pairs. That's why you need a low temperature.

The funny thing is that when the electrons start to form the Cooper pairs, the resistance drops down virtually to zero. This effect is called superconductivity. It means you can, in principle, make a superconductive wire and pass a high current through it, and you won't miss a single penny for the excessive heat. Excellent, isn't it?

Not so much, because the typical value for the critical temperature is about several Kelvin ($-270^\circ$C or $-450^\circ$F). There are works to produce a material which would hold the superconductivity to higher temperatures, but the highest critical temperature reached so far is about $-100^\circ$C ($-150^\circ$F). To cool down the wires to this temperature you have to spend a lot of electricity for the fridges, so finally you even loose. The superconductivity is used in practice, but very rarely, mostly in scientific devices.

Another exciting thing is that the superconductor expels the magnetic field from itself -- it is called the Meissner effect. This expulsion results in a force that makes a piece of a superconductor to levitate above the magnet. Many people believe that this effect will be the base of future zero friction transport. But for the moment, when we don't yet dispose of the warm superconductors, this idea is just fantastic.
}{7Diamonds.png}

%------------------------------------------------------------------------------------

\card{8r.png}{diamond.png}{Bose-Einstein condensate}{Let's return to our favourite Planck's equation (K$\diamondsuit$). If we reduce the energy of the particle, then what is going on with its frequency $\nu$? It also reduces. Hence the wavelength increases. How much can it grow? Pretty much. For the very low energy particles, their wavelengths are so huge that they start to overlap. In practice, it looks like a single sizeable wavy particle instead of multiple small ones. This strange quantum mechanical state of matter is called Bose-Einstein condensate after the names of its discoverers (K$\spadesuit$).

Bose-Einstein condensate cannot be made of any particles, but only of bosons (as you correctly guessed they are called after Bose). And to explain what are bosons we have to introduce the notion of spin. We'll do it with another little experiment. Let's spin a spinning top and then push it slightly. You'll see that the spinning top doesn't fall after the push, as any non-rotating body would do. The particles behave quite similar: they tend to keep their orientation. The spin quantifies this particle quality. You can force the particle to change the orientation of spin if you put it into the magnetic field. Similar to the energy levels of the electron in the atom (4$\diamondsuit$), particles can have only a predefined set of orientations of spin in the magnetic field. If a particle takes half-integer values of spin orientation like $\pm 0.5$, $\pm 1.5$ etc. then it is called a fermion. If the possible values are integers like $0$, $\pm 1$, $\pm 2$ etc. then it is a boson. The fundamental difference between fermions and bosons is that the fermions cannot form the Bose-Einstein condensate. It could be illustrated as follows: imagine a glass full of marbles. Each marble takes some space and doesn't let others take the same position. The fermions behave the same. While the bosons do not obey this logic. Instead, the extremely cold bosons would look like a single marble in the glass, while you know that you have put there a dozen of them.

The Bose-Einstein condensate could be made in practice, although it is very challenging and was done only quite recently. The study of this exotic state is crucially essential for the experimental test of the fundamental laws of quantum mechanics, that's why many laboratories around the globe try to get pure condensate.

An interesting effect appears when you try to cool down the fermions. Fermions expel each other, so they cannot form the condense state as bosons. However, if fermions pair up, they get an integer spin so that they can condensate. An example of a fermion is the electron. And we already considered the situation when the electrons form pairs at low energy: it is the formation of Cooper pairs while transiting to the superconductivity state (7$\diamondsuit$).}{8Diamonds.png}

%------------------------------------------------------------------------------------

\card{9r.png}{diamond.png}{Entanglement}{
Entanglement is one of the most studied questions in physics with several attractive practical applications. And it's one of the strangest. Even Albert Einstein (K$\spadesuit$) has called the entanglement ``spooky''. First, let's repeat what we know about the quantum particles. Their behaviour is described by the wave function (3$\diamondsuit$), which implies that the state of the particle is defined only under the measurement. Otherwise, we have to assume that the particle takes all the possible states at once. Particles also have a spin, which value obeys this quantum behaviour too. What I haven't yet told about the spin is that the total spin conserves in the processes of creation or annihilation of the particles.

What would happen if two particles are created in one process? In this case, they become entangled, which means that they share the single wave function for both. We already learned about the strange behaviour of the individual quantum particles. When there are two of them, it becomes even more bizarre. It turns out that the measurement of one particle in some mysterious way immediately changes the properties of another particle. Let's consider the most often example with the measurement of the spin.

As soon as we haven't measured the particles, their spins remain undefined. But at the moment of the measurement, something interesting happens. The total spin conserves, so if, for example, the spin before the creation of our pair of particles was zero and the spin of the measured particle turns out to be oriented up, then the spin of another particle must be down. And if you measure it, it turns out to be down! But how? The particles were in a completely uncertain state before the measurement, and under the measurement of the first one, the second becomes immediately certain. Immediately -- means that the mysterious information about the orientation of the spin of the first particle reaches the second one with speed faster than the speed of light, which is prohibited by the relativity theory of Einstein.

But maybe the particles keep their spin from the moment of their creation? No, this is excluded by the quantum nature of spin. Imagine having a particle with the spin to the right. And we measure it in the vertical orientation. Then it will have a 50/50 chance to have the spin up or down after the measurement. So the act of measurement changes the spin. But if now you measure the spin of the entangled particle, again in the vertical orientation, you will necessarily find it down. The second particle receives information about the measurement result of the first one with speed faster than the speed of light. That's indeed spooky!
}{9Diamonds.png}

%------------------------------------------------------------------------------------

\card{10r.png}{diamond.png}{Fusion reaction}{This topic of quantum mechanics has probably the most attractive possible application. We already learned that the protons and neutrons could form stable and non-stable structures -- nuclei (5$\diamondsuit$). It means that these systems possess binding energy, like tiny strained springs that hold the protons and neutrons together.

It turns out that heavy nuclei if being broken, radiate a bit of energy. Moreover, in some cases, the broken atom emits a few neutrons. When those neutrons reach other atoms, they break them too. Thus the reaction becomes chained -- more and more atoms become involved in the reaction, and you get a nuclear explosion. In a nuclear reactor, some neutrons are absorbed, so the intensity of the reaction remains stable. Nuclear fuel is the most energy capacious among other sources like coil or gas.

But there is a possibility to get even more energy from light atoms. Unlike the heavy ones, which break apart releasing energy, the light atoms tend to merge, with even larger energy deposit. This is called the fusion reaction. And this process produces much-much more energy, than fission of heavy nuclei. Fusion reaction undergoes in the interior of the Sun (2$\clubsuit$) and other stars (A$\clubsuit$). As a possible source of energy fusion reaction is way more advantageous than the fission reaction. But why don't we use it?

The problem is that it is tough to control. The energy deposit is so large that it blows away the light atoms which you tried to use as the fuel for the reaction. As a result, the fusion doesn't become "chainy" -- the energy released in one reaction doesn't help other atoms to fuse because the last ones are already too far from each other. However, if you try to increase the density of your nuclear fuel, the reaction turns to the explosion, which is called thermonuclear. It is the most powerful and destructive explosion mechanism invented by people.

Different techniques are tried to keep the fusion reaction inside a reactor. But so far no one succeeded in this task. While in the stars (4$\clubsuit$) the hot plasma is held tight by the gravitational force, we don't dispose of such a vast mass to afford it. Instead, we use electromagnetic force to keep the plasma. In the same time, you have to feed somehow the plasma with the fuel to maintain the fusion reaction. This task is extremely challenging and involves dozens of very complex components. But the return would be very attractive too. Many scientists believe that this will be the ultimate solution for all our energy problems. So far, all the existing fusion reactors consume more energy to maintain the plasma than they radiate.
}{10Diamonds.png}

%--------------------------------------------------------------------------------

\thispagestyle{fancy}
\fancyhf{}
\renewcommand{\headrulewidth}{0pt}
\lhead{\thepage \hskip14pt Physics Is My Favorite Game}
\fancyfoot{}
{\huge{\textbf{Suggested materials}}}
\vskip12pt

For this book, I've chosen the format of one page - one card. It limits the number of explanations I can give, although I tried to do not lose in terms of quality. There are different levels of understanding. One is when you grasp only the basic principles and terms. Another level is when you are fluent in handling the formulas (and you are the head of the theoretical division in an international institute). There are plenty of intermediate levels, so you can choose yourself the deepness of understanding that suits to you. I tried to compose the list of suggested readings and videos to help you to go a bit deeper.

Subscribe to Veritasium channel on Youtube. There are plenty of interesting videos, in particular on quantum mechanics. Their explanation of the entanglement is just brilliant! The video ``Is This What Quantum Mechanics Looks Like?" is also quite interesting.

Read the news. For example, subscribe to ITER on Twitter. ITER is the future largest fusion reactor, which is in the construction phase now. It is impossible to give a fast introduction to all the innovations they use. But when you encounter these notions from time to time, you learn it very efficiently. Read also the phys.org website. The base I've given you should suffice for the understanding of most of the articles.

If you like reading books, and I think you do, because you are holding one in your hands right now, then I would advise you to read ``How to Teach Physics to Your Dog'', by Chad Orzel. The book has a lot of beautiful analogies, and introduces in a simple manner the complicated notions as quantum teleportation, for example. And this book is quite recent, so all the modern researches are there.

And if you are looking for something more serious, then there is nothing better than Feynman's lectures: ``The Feynman Lectures on Physics, Volume 3''. This book reaches for a higher level -- roughly speaking it is a half of the university course of quantum mechanics (and twice the knowledge that remains in the student's head after s/he passes the exam). If you want a deep solid understanding, this is the best choice for you.

\newpage
\mypart{Particle Physics -- Hearts}{Particle Physics \\Hearts}{heart.png}{Now, as you understand the principles of quantum mechanics, it is time to discuss the fundamental constituents of matter. We already mentioned electrons, photons, as well as protons and neutrons. But first, the latest two are not elementary particles, that is they are divisible into smaller parts. And second, there are way more particles existing in nature. This part of physics is often called the high energy physics, on the contrary to the low energy quantum physics which we discussed before.}

\card{Qr.png}{heart.png}{Emmy Noether -- symmetry}{
Emmy Noether is a very exceptional figure in this deck. While other scientists depicted here are physicists, she is the only mathematician. You may ask: what does she do here then? Oh, she is here on purpose. If I only could, I would make her the king. Her idea of symmetry serves as the foundation of the whole area of particle physics.

It's quite easy to express the Noether's idea, but not easy to explain. The idea is the following: any symmetry in Nature implies some conservation law. Let's first clarify what here meant by symmetry. Usually, we say that an object is symmetric if, for example, its left side looks the same as its mirrored right side. But what if the object looks the same after being moved at some distance? From the mathematical point of view, it is also a symmetry, called translation symmetry (here an object may mean not only a single object, but a system of objects too). If the object looks the same after some time passed, then this kind of symmetry is called time translation symmetry.

If a body has the same speed after passing some distance, then it is symmetric under space and time translations. But from the simple mechanics, we know that if a body has the same speed, then it's momentum and energy are conserved. Precisely, the space translation symmetry corresponds to the momentum conservation and the time translation -- to the energy. If you ever wondered why the physics teacher in school insists that these quantities conserve, here is the reason: because Nature is symmetric.

In the field of particle physics, the Noether theorem becomes truly special. Sometimes the particle physicists are even called the "symmetry hunters". What does it mean and how does it work we will be learning all through the hearts suit. Generally, we are looking for some hidden symmetries, which imply some conservation laws. These symmetries -- hence the conservation laws -- could be sometimes broken, which means that the corresponding quantity doesn't always conserve and the processes, being symmetrically translated, don't always go the same. Particle physics is all about these symmetries. And very often we even use the words ``symmetry'' and ``conservation law'' as synonyms.}{{Cards.014}.png}

%------------------------------------------------------------------------------------

\card{Kr.png}{heart.png}{Paul Dirac -- antimatter}{
In 1928 Paul Dirac considered the movement of an electron on the orbit around an atom nucleus. We've spoken already about this system (4$\diamondsuit$). In turned out that the equation Dirac wrote for the electron also has another solution. It describes a particle identical to the electron but with the opposite electric charge. Such a particle was named positron because it has a positive charge (while the charge of an electron is negative). The Dirac's equation is symmetric on the charge reversal. It necessarily means that, since the equation is mathematically correct, there should exist these anti-electrons, the positrons.

Soon the positrons were discovered in the cosmic rays (5$\clubsuit$). But what's more, it turned out that every elementary particle has its anti-companion. Hence, as the ordinary matter consists of particles, the matter that consists of antiparticles is called antimatter. The particles and antiparticles annihilate upon collision: they disappear, radiating a large quantity of energy and (optionally) creation of new particles. According to the Noether theorem (Q$\heartsuit$), there should exist a conservation law that corresponds to the symmetry between matter and antimatter. It is the law of the charge conservation: in any interaction the total charge conserves.

But if the antimatter is just a symmetric reflection of ordinary matter, then where is it? If the Big Bang (J$\spadesuit$), from which the Universe started, was purely symmetric, then there should exist the equal quantity of antimatter as the amount of ordinary matter. Still, one of the biggest questions to modern physics is the absence of antimatter in the Universe. It is called the baryogenesis problem because in cosmology the ordinary matter is generally called the baryonic matter (the term \textit{baryons} for the particle physics means protons and neutrons and other similar particles). Thinking about this problem, scientists came up to the minimal requirements to generate this disbalance. I'm speaking about the Sakharov conditions for baryogenesis. One of them is -- guess what -- the violation of the charge symmetry. Then this violated symmetry should produce the observed disbalance of matter and antimatter. Another approach to the solution of this problem we will mention later when we will be discussing the inflation theory (9$\spadesuit$).

Finally, one should mention that matter-antimatter symmetry implies not only conversion of the electric charge, but all the charges. For example, when we will get acquainted with the strong interaction (5$\heartsuit$), we will learn about the colour charge. So if a particle (a quark) has, let's say, the red colour charge and electric charge $2\over3$, then its antiparticle will have colour charge anti-red and electric charge $-{2\over3}$. 
}{{Cards.015}.png}

%------------------------------------------------------------------------------------

\card{2r.png}{heart.png}{Gauge}{
Gauge is one of the essential notions in particle physics. It is the way to introduce all the interaction.

In classical physics, we had two types of interactions. The first type is the direct interactions of bodies when they touch each other: for example, the collision of two billiards balls on the table. A different kind of interactions is more interesting because it can act without a touch, on distance. For example, if you rub a balloon on your hair and then move it away from your head, you will see that the hair will be pulled towards the balloon. But there is nothing between the balloon and the hair. It is an example of field interaction. The hair is pulled to the balloon because of the electromagnetic field between them.

Particle physics attempts to explain the field interactions by the exchange of some mediate particle.  The diagram on the right shows a process like this: an elastic scattering of a quark on an electron. The time axis is from left to right. At the beginning (left) there are a quark and an electron, and in the final state (right) it is the same, that's what the term ``elastic'' means. But there was a moment when these two particles interacted. They interact by the exchange of a photon. In this kind of charts (called Feynman diagram) most of the particles, like quarks or electrons, are denoted by solid lines with arrows. Unlike in this example, particles can change to another particle after the interaction. The mediate particles are shown with wavy lines. Each line type is reserved for some mediate particle. The mediate particles are bosons (remember? These guys have integer spin). One of the fundamental bosons which we already acquainted is the photon, the mediate particle of the electromagnetic interaction.

So between your hair and the balloon, there are photons, flying there are back and transmitting the attractive interaction. The same is going on the microscopic scale when we speak about the collisions of tiny particles. The particles don't collide like the billiards balls. Instead, they exchange a photon, or another boson, and thus interact. What we told before about the collision of billiards balls -- on the fundamental level -- is nonsense: if you zoom in this collision and look what's going on between the atoms of one ball and another you will see that the atoms don't collide. Instead, when they approach each other, they exchange the photons and thus repulse. The way to explain the field interactions by the exchange of a mediate particle is called gauge theory. All the fundamental interactions are gauge interactions, with one exception: gravity. Gravity physics is entirely different. We will learn this difference later (K$\spadesuit$).
}{{Cards.016}.png}

%------------------------------------------------------------------------------------

\card{4r.png}{heart.png}{Electroweak interaction}{
Here is the first fundamental interaction that we will be considering -- the electroweak one. Quite often, scientists speak about two distinct interactions: weak and electromagnetic. We are all familiar with the electromagnetic part. It is not only the basis of all the electronic devices. The electromagnetic force holds the electrons within the atoms and the atoms within the molecules. On its turn, the molecules attract each other again by the electromagnetic force. That is the existence of matter as we know it in everyday life is due to the electromagnetism. This interaction is mediated (2$\heartsuit$) by the photons (K$\diamondsuit$ and 2$\diamondsuit$)

Now let's consider the weak part, which for a long time believed to be quite distinct from the electromagnetic interaction. The weak interaction can apply to all the particles, but its strength is extremely low. That is the weak interactions are the rarest of all. A classic example of the weak interaction is the $\beta$-decay of a neutron (remember that the atom nuclei consist of protons and nuclei). The neutron on its turn consists of three quarks -- we will be learning about the quarks few cards later (5$\heartsuit$). In the weak neutron decay, one of the down quarks (\textit{down} here is just a name), depicted on the picture with a little blue ball with a letter \textit{d}, changes to a small red ball with \textit{u} -- the up quark. In this process, the quark changes its electric charge from $-1\over3$ to $2\over3$, so it gains $+1$. This charge the quark gets from the W boson (yellow wavy line). W is the fundamental boson of the weak interaction. It is the only boson that possesses the electric charge. There is also another weak interaction boson called Z (which has zero charge). Both W and Z bosons are quite distinct from all other bosons because they have non-zero mass. Their mass is not just non-zero, but these are ones of the most massive known fundamental particles. They are about 90 times heavier than a proton.

The fact that these particles are so massive explains the weakness of the weak interaction. As stated in the uncertainty principle (J$\diamondsuit$), the energy of a particle can be uncertain. It means that we can ``borrow'' some energy from nowhere, from the uncertainty, to create a massive W or Z boson. However, the probability of succeeding in it is reduced due to the large mass of these particles. Remember also that the considerable borrowed energy means very precisely defined position. That is the typical range of weak interaction should be quite short. It is about $10^{-17}$ meters or one-hundredth of the size of a proton. Hence to see two particles interacting weakly, you have to approach them very-very close to each other.
}{{Cards.017}.png}

%------------------------------------------------------------------------------------
\card{3r.png}{heart.png}{Symmetry breaking}{
While considering the electroweak interaction (4$\heartsuit$), we mentioned that it consists of two, but didn't explain what does it mean. Let's fill this gap now.

We told about the notion of symmetry (Q$\heartsuit$) and its importance to the particle physics. But what is even more interesting is the phenomenon of the symmetry breaking, when the symmetry becomes imperfect. Hence, the corresponding conservation law violates.

Take a wine bottle. As you know, the wine bottles have a convex on the bottom (if you look from inside). Take a marble and drop it inside. Since you drop the marble right on top of the centre of the bottle, one could naively expect that the marble will stay there atop the convex. However, you will never get this situation. Instead, the marble will roll down to the side of the bottle. Now, what happened: you started with a symmetric situation -- symmetric bottle and symmetric position of the marble relative the central bottle axis. But you end up with a non-symmetric state, though this asymmetric final state is the most stable one.

The same happens in particle physics and is called spontaneous symmetry breaking. The ``wine bottle bottom'', depicted on the card, is the \textit{potential}. When the particle rolls down the potential to reach its minimum, it obtains mass. The electromagnetic force, mediated by photons, represents the symmetric realization of the electroweak potential, without a convex. However, sometimes the circumstances change, and the potential becomes like it's shown on the picture. Then the photon becomes massive and turns to the W or Z boson.

What makes the electromagnetic force turn to the weak force? What changes the potential? It is the Higgs mechanism, associated with another fundamental particle (which later became very famous) -- the Higgs boson. Higgs mechanism, initially introduced for the electroweak interaction, also acts to other particles and makes them massive. On the fundamental level, the particles have mass because they interact with the Higgs boson.

The Higgs mechanisms explain why the tiny constituents of the matter have mass. But it is very little responsible for the masses of macroscopic bodies. For example, a proton consists of three quarks, but each of these quarks has masses 200-500 times smaller than the mass of the proton. It means that the mass of a proton is only by 1\% explained by the mass of the quarks (hence by the Higgs mechanism: the quarks have their masses because of Higgs boson). The extra 99\% of the mass is actually due to the binding energy of quarks within the proton. The same is true for neutrons. The protons and neutrons are the constituents of atomic nuclei. Thus they are the main contributors to the mass of macroscopic bodies.
}{{Cards.018}.png}


%------------------------------------------------------------------------------------
\card{5r.png}{heart.png}{Strong interaction}{
We have already mentioned several times the quarks -- the constituents of protons and neutrons. They are probably the most unusual particles. All other particles have charge either -1 (like an electron) or 0 (like Z boson) or 1 (like $W^+$ or proton). The quarks are different. They have charge either $2\over3$ or $-{1 \over 3}$. There are six quarks, and two of them are the most important ones -- the \textit{up} and \textit{down} quarks. The up quark has charge $2\over3$, and the down one has $-{1\over3}$. The proton is a bound state of two up quarks and one down. And the neutron is a bound state of one up and two downs.

Since the quarks have the electric charge, they can interact electromagnetically. We have also seen that they can interact weekly (4$\heartsuit$). But the most interesting is that they
have an interaction exclusively reserved for them -- the strong force. The mediating particle (2$\heartsuit$) for the strong interaction is gluon (because it ``glues'' the quarks together).

Quarks not only possess a fractional electric charge. They have also another charge of different nature, called \textit{color} charge. While electric charge can be either negative or positive, so two options, the colour charge can have six possible values: red, green and blue, and also antired, antigreen and antiblue. Each quark has one of these colours. Gluons have two colour charges at the same time. For example, there could be a red-antigreen gluon. Quarks in a proton are unceasingly exchanging gluons. Thus their colour charge always changes. Don't let this colour notation perplex you. There is no actual colour for the elementary particles. What we call here colour is just a strong charge. And the colour terms were artificially chosen to illustrate the idea that the observed states, like protons and neutrons, are always white, or strongly-neutral.

And this is the most exciting property of the strong interaction. The strong force is so strong, that when you try to pull one quark from some bound state of several quarks, it immediately creates new quarks: one to replace the quark that was pulled out, and another to make a pair to that lonely pulled out one. In an experiment, we can never see a lone quark. We can only see systems of three or two quarks. The system of three quarks exists because the quarks have colour charges red -- green -- blue, hence white in total. And the two quarks may have, for example, red -- antired charges, so be again white.

It should be mentioned that the strong force not only binds the quarks inside the protons and neutrons but also indirectly explains the attraction between protons and neutrons. This attraction is also a gauge interaction (2$\heartsuit$), but it doesn't act through any fundamental boson. Instead, the protons and neutrons exchange with the double quark states called pions. The pions are created inside the protons and neutrons due to the uncertainty principle. }{{Cards.019}.png}

%------------------------------------------------------------------------------------
\card{Ar.png}{heart.png}{Standard Model}{
Mainly we have already introduced all the fundamental interactions that we know exist in Nature. The Standard Model (note the Upper Case!) is the way to systematise this knowledge.

First, there are six quarks (5$\heartsuit$), divided into three generations (or flavours). The first generation is the already mentioned up and down quarks, the second is the charmed and strange quarks, and the third is the top (or truth) and bottom (or beautiful) quarks (from those names you learn that physicists are quite romantic). You can also divide the quarks into two other groups: the up-type and down-type. The up-type is: up, charmed and top. The down-type is: down, strange and bottom. The up-type quarks are more massive than the down-type quarks from the same generation. And the up-type quarks have electric charge $2\over3$, while the down-type ones have charge $-{1\over3}$. The masses of the quarks increase with each generation. For example, the top quark is more than $50000$ times heavier than the down quark. On the card, the relative masses of the particles are illustrated with the darkness of the colours.

Then there are six leptons. They also divide into three flavours. In each flavour there are one charged particle: electron, muon and tau (the latter two have the greek letters to denote them: $\mu$ -- mu and $\tau$ -- tau). And there are also neutral particles, neutrinos. We will learn about them later (7$\heartsuit$). The charged leptons also become more massive from electron to tau: the electron is about 200 times lighter than the muon, and almost 4000 times lighter than the tau. The charged leptons don't interact strongly, because they have no colour charge. But they can interact with the electroweak bosons (4$\heartsuit$): photon (denoted with the Greek letter $\gamma$ -- gamma), W (these bosons are charged. Depending on their charge they are indicated as W$^+$ or W$^-$), and Z boson. The neutrinos have no electric charge, so the only interaction left for them is the weak one. Hence, the probability that a neutrino interacts with even a considerable amount of matter is negligibly small. For example, the Earth is almost entirely transparent for neutrinos.

All these particles obtain their masses by interaction with the Higgs boson (H in the centre). This particle was found experimentally in 2012, and it finalises our understanding of the Universe on the elementary level.

Does it mean that there is nothing left to study? Not at all. First, not all the parameters of the Standard Model are well measured. But the most exciting task is to look for the particles beyond the Standard Model -- for example, the hypothesised supersymmetry particles (9$\heartsuit$).
}{{Cards.020}.png}

%------------------------------------------------------------------------------------
\card{Jr.png}{heart.png}{Synchrotron}{
We already mentioned that if you want to observe some interactions, you need to draw particles very close to each other. For the electroweak interaction (4$\heartsuit$) it is, for example, about $10^{-17}$m. The strong interaction (5$\heartsuit$) is not much easier to achieve: since the gluons don't allow the quarks to fly away from them, the typical range of the strong force is about $10^{-15}$m. However, accomplishing such a rapprochement, it is not that easy. Typically, particles repel. To overcome this effect, physicists smash particles into each other on the speeds close to the speed of light. But accelerating particles is not an easy task.

Think of the electric current. Essentially it is just the movement of the electrons in the wires under the action of the electric field. The same is valid for the other charged particles. By simply putting a proton between the plus and minus charges, you can accelerate it to some energy. But the more working idea is to put a bunch of protons in a variable field and make them fly on a circle. Veksler first showed that in this case the protons are ``surfing'' the wave and don't fly apart. The particle accelerating machine that works on this principle is called the {\it synchrotron} -- because the electric field is synchronous to the particle beam. Lap after lap the protons gain the energy. Then, once the particles reach the required energy, they are expelled from the accelerator ring and fly to the experimental setup.

Why the protons? Well, you need some charge to accelerate the particles, so obviously one can't use the neutrons. Electrons could also be used in the accelerators. However, the electrons moving on a circular orbit emit their energy, so you cannot accelerate them very much. Protons are more massive, and this effect is way less severe for them. In modern accelerators, they also use the nuclei of the elements. The most popular choice is the nuclei of carbon or lead. Also, you can study a lot of exciting physics by accelerating muons (A$\heartsuit$), which are $\sim200$ times heavier than electrons. But the projects to build such machines have only recently appeared.

While the energy emission due to the circular movement of a particle is unwanted in the task of gaining maximum energy, this emission itself could be used in a lot of applications. Since it has high intensity and comes in very narrow beams, it is used, for example, to make ultra-high resolution images of in biology or material science. Nowadays the mood has changed, and precisely this type of machines are now called synchrotrons -- or being more precise, sources of the synchrotron radiation.

Most modern accelerators are colliders. It means that they have two accelerators, usually implemented in a single ring, where the bunches of particles rotate towards each other. It allows increasing the energy of the collision by factor 2 while using the machinery of the same power.
}{{Cards.021}.png}


%------------------------------------------------------------------------------------
\card{6r.png}{heart.png}{Vacuum}{
What is the vacuum? In the most simplistic explanation, it is just emptiness. In the more accurate form vacuum is what lefts when you remove everything from some space. You may ask: what is the difference? If you remove everything, then there is nothing lefts, just empty space! No, not really.

Let's take a closer look at the vacuum and consider a very tiny spot in it. Very tiny means that you know the position and the size of this spot with high precision. According to the uncertainty principle (J$\diamondsuit$) if you know the position very well, then you become very uncertain about the energy in this spot. This uncertain energy manifests itself in the form of particles that emerge out of thin air (though there is not even air in the vacuum). These are pairs of symmetric particles: a particle and its antiparticle. They appear and immediately annihilate, releasing the energy borrowed from the vacuum. Such particles are called \textit{virtual}. Their appearance is also called \textit{vacuum fluctuation}.

There are no effects of this physical vacuum in everyday life. However, it changes everything once you go down to microscopic scales. In any interaction, the virtual particles may appear at the place and time of the interaction and influence the resulting process. It is very useful for physicists: the mass of the particles that you can create in the experiment is limited by the energy of the collision. While the mass of the virtual particles is not limited at all. Hense, through the vacuum you can access to the energies, you would not be able to reach with just an accelerator. The vacuum can also make the impossible interactions possible. For example, photons do not interact with each other. However, when two photons approach each other, a vacuum fluctuation may create a pair of charged particles in the same spot. In this case, the photons would interact with these particles. Efficiently at the end of the day, we get the photon-photon scattering.

To be precise, the bosons that mediate any interaction, and all the other possible intermediate particles are also called virtual. To explain this, let's return to the electroweak card (4$\heartsuit$) and ask ourselves a simple question: what charge does the W boson have? Let's consider it is flying from top to bottom. Then it carries the negative electric charge, obtained from the down quark. Thus it is W$^-$. But similarly, it could be a positive W$^+$ that flies from bottom to top and balances the negative charge of the electron. What option do you prefer? Although in this particular example the first possibility seems more logical (otherwise you should assume a W-e-$\bar\nu^e$ triplet appearing out of nowhere) I assure you that there is no actual difference. This W boson is virtual, and it could be equivalently considered as W$^+$ or W$^-$.}{{Cards.022}.png}


%------------------------------------------------------------------------------------
\card{7r.png}{heart.png}{Neutrino}{
The last fundamental particle that we haven't yet considered is neutrino (denoted with the Greek letter $\nu$ -- nu). It belongs to the class of leptons, but unlike the electron, muon and tau, the neutrino has no electric charge. What is also interesting, they have almost zero mass. What mass they have remains undefined. But they are not massless, that's for sure. As we already told, there are three flavors of neutrino: $\nu^e$, $\nu^\mu$, $\nu^\tau$. Each of those may convert by weak interaction to the corresponding charged lepton. There is no other way for neutrons to interact with matter, only weakly. And the weak interaction is so weak and rare that a neutrino can fly through the Earth without any interaction.

The most exciting phenomenon with neutrino is its oscillations. Neutrino changes its flavour on the flight. Thus, for example, a tau neutrino can suddenly become $\nu^\mu$ and then $\nu^e$. This effect was found when physicists measured the flux of solar neutrinos. In the detector, we can see the signal of the muon or the electron, created when the neutrino by chance interacts in the detector medium. But the tau lepton from the $\nu^\tau$ is usually missed. Hence the observed flux of neutrinos is always much lower than the expected one because the detectable $\nu^\mu$ and $\nu^e$ convert to the undetectable $\nu^\tau$.

This behaviour is related to the wave nature of any particle. The neutrino ``consists'' of three oscillating parts. Or better saying the neutrino is the quantum mixture of three neutrino mass states. The relative amount of these states in each neutrino changes with time. To illustrate it, let's conduct a little experiment. And this one will be again with a music instrument. A tremolo harmonica this time (if you don't have one, try to search a sample sound on the internet). The tremolo harmonica sound is unique. It sounds like ``whoa-whoa-whoa''. It happens because, inside this instrument, there are two reeds for each note. They are slightly mistuned relative to each other. This ``whoa-whoa-whoa'' is created when the waves from the different reeds add up.

The same happens inside the neutrino. The mass states of neutrino vibrate inside it like the reeds. These vibrations interfere with each other, resulting in a kind of neutrino ``whoa-whoa-whoa'' -- neutrino oscillations. The only difference is that inside the neutrino there are three ``reeds''. The oscillations necessarily imply that the neutrino has mass (which is not measured so far). Still, they are the lightest massive particles.

Nowadays the scientists suspect that there is the fourth neutrino flavour, which manifests itself by the oscillations observed very close to the neutrino source (usually we use a nuclear reactor as a neutrino source). This hypothetical fourth state is called \textit{sterile} neutrino.
}{{Cards.023}.png}


%------------------------------------------------------------------------------------
\card{8r.png}{heart.png}{Quark-gluon plasma}{
When we considered the strong interactions (5$\heartsuit$), we told that the quarks could not fly out from the colourless bound state of two or three quarks. This property we call \textit{confinement}. In terms of force, this could be explained as follows: the attraction between the quarks increases with the distance. It is very unusual if you think of it a bit: all the attraction forces we see in everyday life decrease with distance. The planets ``feel'' the attraction of the Sun, because they are relatively close to it. And the attraction to the other stars in the Galaxy is practically negligible. If the quarks were planets, the situation would be the opposite: the farther the planet is from the star, the stronger the attraction between them is. However, this doesn't work forever. At some point, when you take away two quarks from each other, the gluon connecting them breaks, creating a pair of new quarks. And you end up with four quarks in two pairs.

A bit of terminology: the possible colorless bound states of quarks are called -- \textit{mesons} for two quarks state, \textit{baryons} for three quark state and \textit{glueballs} for the unstable colorless multiquark states. Mesons and baryons are also called \textit{hadrons}. 

On practice, how can you pull a quark from a hadron? You cannot grip it with the tweezers! No, instead you accelerate the hadron in the synchrotron (J$\heartsuit$) and smash it into another hadron. If the energy of the collision is greater than the binding energy of the hadron, it breaks apart on several less energetic hadrons. If they are still too energetic, they can divide again. And again and again. At the end of the day, you see the \textit{hadron jet}: the shower of a vast number of quarks. This process is called \textit{hadronization}.

As you can see, by an increase of the collision energy, you cannot see the free quark, because they fly apart creating the new hadrons. But what if instead you hold the quarks in a tiny volume and increase the energy? Like this, you would break the connections between the quarks and would not let them fuzz. This exotic state is called the \textit{quark-gluon plasma}. To create this state, you smash the nuclei in the collider. Then, for a tiny fraction of second, at the collision point, you have this quark-gluon plasma, where they fly freely.

The quark-gluon plasma is believed to fill the Universe few microseconds after the Big Bang (J$\spadesuit$). Thus the study of this state is directly related to the cosmology (A$\spadesuit$).
}{{Cards.024}.png}


%------------------------------------------------------------------------------------
\card{9r.png}{heart.png}{Supersymmetry}{
We considered already a lot of symmetries (Q$\heartsuit$) in elementary particle physics. Some of them didn't mention explicitly. So let's list them: the symmetry between the particles and anti-particles; gauge symmetry (yes, on the fundamental level it is symmetry too); the symmetry between the generations of quarks; symmetry between different colour charges; symmetry between the different flavours of leptons. Admittedly, there are more symmetries in this highly symmetric branch of physics. Now we'll consider the symmetry between the bosons and fermions.

First, let's remind what bosons and fermions are. By definition, the bosons are the particles -- not necessarily elementary ones (that is they can be composed of other particles) -- with the integer spin. We have already told about spin earlier while speaking about the Bose-Einstein condensate (8$\diamondsuit$). There are non-elementary bosons: all the mesons, the nuclei with the even total number of protons and neutrons etc. And there are fundamental bosons: photon, W$^\pm$ and Z, gluon and Higgs boson. Now we are interested in the fundamental ones. Besides, there are fermions with the half-integer spin. Again, there are composite fermions, like protons, neutrons and all other baryons, nuclei with an odd number of protons and neutrons etc. And there are fundamental fermions: leptons and quarks. The fundamental fermions are the constituents of matter (A$\heartsuit$), and the fundamental bosons are the mediators (2$\heartsuit$).

The symmetry between fermions and bosons, if it exists, is very strongly broken. Remember what happened when the symmetry in the electroweak interaction (4$\heartsuit$) was broken (3$\heartsuit$)? The corresponding W$^\pm$ and Z bosons became very heavy. Similarly, the particles that correspond to this fermion-boson symmetry are way heavier than all the other particles. This symmetry is called supersymmetry.

The hypothetical supersymmetric particles have names with the suffix ``-ino'': for the photon, there is a heavy-fermion particle photino; for the electron, there is a heavy boson electrino etc. All these particles, if they exist, are not discovered yet. Current searches limit the masses of the supersymmetric particles to be about one thousand times larger than the masses of the corresponding ordinary particles.

Supersymmetry offers a natural explanation to some long-standing problems in particle physics. For example, it is not yet understood why the gravity is so weak even relative to the weak force. Also, if the supersymmetric particles exist, the lightest of them would be stable and would not interact with the ordinary matter. Particles with precisely these properties are searched as the dark matter particles -- what is a dark matter we will be speaking later (Q$\spadesuit$).
}{{Cards.025}.png}


%------------------------------------------------------------------------------------
\card{10r.png}{heart.png}{Grand Unification}{
We managed to get through the elementary particle physics without even mentioning the energy scales. Let's fill this gap now. The units of energy used in this field of physics are called electronvolts. One electronvolt measures a very small amount of energy: for example a tennis ball on the court has an energy of billions of billions of electronvolts. In fact, one electronvolt energy is pretty small even for the elementary particles. In the modern accelerators (J$\heartsuit$) the energy of particles reach billions and even thousand of billions of electronvolts. One billion electronvolts is called giga-electronvolt, or GeV.

Nature, being more powerful than humans, can accelerate particles to the higher energies. For example, the remnants of the supernova explosions (6$\clubsuit$) can accelerate particles up to millions GeV. But we can observe particles with energies reaching a hundred billion GeV (5$\clubsuit$)! Now we'll change the subject a bit, but please keep in mind these numbers.

We considered two types of interactions in particle physics: strong (5$\heartsuit$) and electroweak ones (4$\heartsuit$). The electroweak interaction is interesting because at low energies -- below about one hundred GeV -- it looks like two quite distinct interactions. It means that the electromagnetic interactions appear way more intensive than the weak ones. However, above one hundred GeV they merge, forming an electroweak interaction.

It may seem natural to expect that the electroweak and strong interactions can fuse at some very high energy. Scientists suspect that this is indeed the case. They call the theory that describes this phenomenon the Grand Unified Theory. Such a theory can explain how the fundamental forces emerged from the one unified interaction that existed at the very first moment after the Big Bang (J$\spadesuit$) and clarify, for example, why the protons have the same electric charge as electrons (just with the opposite sign).

But how high the energy should be for the three interactions to merge? The intensity of the fundamental forces changes with energy. So measuring these intensities on the low energies, one can extrapolate the measurements to the high energies and try to find the point where these extrapolated intensities are equal. It turns out that the Grand Unification should happen on the energy about one million billion GeV! It is ten thousand times more than the highest energies ever observed on Earth (in the cosmic rays -- 5$\clubsuit$). It seems impossible that we will be ever able to study such energies. Directly -- probably not. But it is not excluded that we will be able to research this theory indirectly. That is without creating the Unified particles in the accelerator, but observing their signature in virtual particles (6$\heartsuit$) which is allowed by the uncertainty principle (J$\diamondsuit$).

%The primary motivation for the Grand Unified Theory is the fact that the electric charge of a proton is precisely equal to that of the electron (except that they have the opposite sign). The Standard Model (A$\heartsuit$) can give a natural explanation to this. So the Grand Unification is needed to avoid this unnaturalness in particle physics.
}{{Cards.026}.png}

%--------------------------------------------------------------------------------

\thispagestyle{fancy}
\fancyhf{}
\renewcommand{\headrulewidth}{0pt}
\lhead{\thepage \hskip14pt Physics Is My Favorite Game}
\fancyfoot{}
{\huge{\textbf{Suggested materials}}}
\vskip12pt

Particle physics has one crucial advantage for a non-specialist. The experiments in this field can be only conducted using huge machines -- accelerators, the experimental setups are super-expensive, and the teams of scientists working on each research are large. In these circumstances, if the team put even just a little per cent of their effort to the outreach activities, you get a lot of exciting projects in the result.

One of such projects is, for example, the ATLAS Minecraft map, where you can learn quite a serious physics in the environment of a popular video game.

CERN is today the world-leading centre for the particle physics research, and it will remain in this position for a long time. Check out their Youtube channel, and you'll find a lot of cute videos on their activities.

Don't neglect the TED lectures. They are usually short and straightforward, though scientifically precise. Don't miss Brian Cox's speech on the CERN's supercollider. It will give you the sense of the scales for the experimental setups.

As for the reading, the best pop-book on particle physics I ever held in hands is the ``Particle Physics: A Very Short Introduction'' by Frank Close. However -- and this remark should be taken into account for any book on this subject -- the particle physics is a fast-changing field. This particular book was published in 2004 -- eight years before the discovery of the Higgs boson. Keep in mind that quite often the information given even in excellent writing may be incomplete.

And the best serious introduction to particle physics is, on my opinion, the book of Donald Perkins ``Introduction to High Energy Physics''. It is a kind of classics -- so no surprise this book is pretty old. It was last edited in 2000.

\newpage
\mypart{Astrophysics -- Clubs}{Astrophysics \\Clubs}{club.png}{This section will be quite mixed. Astrophysics is pretty much about everything: interstellar gas and black holes, star clusters and galaxies, gravitational waves and cosmic particles.}

%------------------------------------------------------------------------------------

\card{Kb.png}{club.png}{Edwin Hubble -- galaxies}{
In XVIII-XIX centuries scientists believed that Milky Way (our galaxy) was itself the Universe. The question about the actual size of the Universe was especially keenly posed at the beginning of the twentieth century when scientists started to think about the nature of numerous nebulae that could be seen in telescopes. In 1920 a discussion between two authoritative American astronomers Harlow Shapley and Heber Curtis arose. The dispute was about the nature of nebulae. Shapley affirmed that all the nebulae were nothing but gas formations situated in our galaxy. Meanwhile, Curtis contended that many nebulae were individual galaxies, containing billions of stars and are located far away of our galaxy. According to Curtis, our world is by many orders of magnitude larger than any of galaxies. Both scientists gave observational and theoretical arguments for their concepts, but couldn't conclude.

In 1917 the Mount Wilson Observatory was equipped with the largest telescope at the time with the primary mirror diameter 2.5m. Edwin Hubble (1889 – 1953) started to work there. Using the photographic method in 1923 – 1924 he resolved, for the first time, three spiral nebulae to individual stars. Among the stars of the Andromeda nebula, he found some variable stars – cepheids, which allowed to measure the cosmic distances. According to the estimation, the distance to the Andromeda nebula is much higher than the size of our galaxy. Thereby Hubble proved that the Andromeda nebula is situated outside the Milky Way and constitutes a large star system, as big as our galaxy. Thus, with the inauguration of a new telescope, the size of the Universe was increased severely. The Curtis's point of view won.

Hubble studied multiple galaxies and composed a classification of their morphology to make the first step to an understanding of the galaxy formation and evolution. It is so-called ``Hubble's tuning-fork''. According to this scheme, at the early stages, the galaxies are elliptical, without any apparent features. Then they start to develop the spiral arms. For the spiral galaxies, Hubble differentiated the types with or without the central bar. Nowadays, we understand that the spiral structure is inherent mostly to the young galaxies, while the ellipticals dd are in the main old ones. And the bar can appear and disappear along with the history of a galaxy. However, even today, we find useful the Hubble's classification.

Hubble also made a significant discovery of the recession of all the galaxies from each other. We will touch upon this later (J$\spadesuit$). Although we omit it on this page, it cannot be neglected as it is one of the most important breakthroughs in XX century science.
}{KClubs.png}

%------------------------------------------------------------------------------------

\card{2b.png}{club.png}{Sun}{
The Sun, the ancient god for humans, the subject of extensive research for scientists. An ordinary star in the Galaxy (A$\clubsuit$), and the most important source of information about stellar physics (4$\clubsuit$).

All the cycles on Earth are driven by the Sun. For example, the biological material from which we are made is always in circulation: from us to the ground, then to the plants and back to us with food. And the source of energy for this cycle is the Sun.

The Sun was created about 4.6 billion years ago. Before that, there was a cloud of hydrogen and dust. Eventually, due to fluctuations in the cloud density, it started to rotate and shrink. As the density in the centre of the cloud grown, the temperature increased. At some point, the temperature was high enough to light up the fusion reaction (10$\diamondsuit$). This moment could be considered as the birth of the Sun.

The Sun is enormously large -- its diameter is 109 times bigger than that of Earth. The outer shells of the Sun are not so hot, ``just'' about 6000 $^\circ$C (about 10000 $^\circ$F). But its core, where the fusion reactions take place, has temperature more than ten million $^\circ$C and it is more than ten times denser than lead. On Earth, we cannot handle the fusion reaction. But in the Sun's core, the high pressure of this powerful reaction is balanced by the enormous gravitational field. 

The hot plasma from the centre of the Sun continually goes up to the surface, where it cools down and falls back. This plasma motion makes complex structure of layers. Since the plasma is charged, its movement also creates the magnetic field of the Sun.

The complex structure of the Sun is studied with the same seismic method we use to analyse the interior of Earth. Every disturbance in the Sun's core results in the vibrations of the surface, like the earthquakes. Observing these vibrations, we can learn what caused it.

The Sun does not only give us warm and light. Unfortunately, its bright power is unavoidably linked to the radiation and the magnetic storms. The magnetic field of Earth blocks most of the Sun's radiation, but the rest is still pretty harmful to us, especially for our electronics and for the people sensitive to it. Such phenomena as Sun's black spots and solar flares are the main features due which we can predict the magnetic storms and fluctuations in the flux of the solar radiation. By the way, the latter one is considered as one of the main challenges during the human exploration of the Solar System (3$\clubsuit$).
}{2Clubs.png}

%--------------------------------------------------------------------------------

\card{3b.png}{club.png}{Solar System}{
To explain in details this card I would have to write another book. All I can do on one page is to list the main features of our home planetary system.

Since the dawn of time, humans knew the six planets of the Solar, the ones closest to the Sun: Mercury, Venus, Earth, Mars, Jupiter and Saturn. Uranus and Neptune can not be seen with a naked eye. So they were discovered much later -- in XVIII and XIX centuries correspondingly. The terrestrial planets (the four closest to the Sun) are composed of rocks and metals. Giant planets: Jupiter, Saturn (two most massive planets of our system), Uranus and Neptune consist of light materials like gas or ice. This discrimination is caused by the way the Solar System was formed: the light elements and ice could exist in solid form in the outer regions, where they formed the giant planets. The remaining material forms the asteroids and comets. Most of them are concentrated in the asteroid belt between Martian and Jovian orbits and in Kuiper belt beyond Neptunian orbit. The largest body of the Kuiper belt, Pluto, has recently lost the title of a planet. However, studies show that there is probably yet another planet in the very outer regions of the Solar System. Yet undiscovered Planet Nine would explain the similarities in the orbital motion of the trans-Neptunian objects. 

The borders of the Solar System are still not well studied. The Oort cloud, comprising billions of comets, spans from about 1000 to 100,000 AU (AU -- astronomical unit -- the distance between Earth and Sun, equal to about 150 million kilometres or 90 million miles). The most distant spacecraft Voyager 1 for 40 years of its mission has passed only about 0.1\% of the Solar System's radius.

Today we reached so high level of development that we are even attempting to spread our civilization to the other planets. But here we face significant challenges. Other planets are either too cold (Mars), or too hot for us (Venus). The cosmic radiation is also a very striking factor. However, in 2030s, we are aiming to land people on Mars. And this is only the start of the future massive exploitation of the planets and moons of the Solar System.

The mentioned asteroids and comets constitute a severe menace to life on Earth. Although small meteors burn down in the atmosphere, the large ones can cause serious damage: explosive impact, tsunami, earthquakes. A large deposit of dust in the atmosphere will cause the reduction of Sun warmth reaching the ground. Sixty-six million years ago such impact resulted in the extinction of the dinosaurs. But today we hope to dispose of enough means to avoid collisions. Much effort is made to monitor the dangerous asteroids around the Earth. If any large body were spotted on a hazardous orbit, it would suffice to deflect it slightly and thus save our planet.
}{3Clubs.png}

%--------------------------------------------------------------------------------

\card{4b.png}{club.png}{Stellar evolution}{
The life cycle of any star begins in the disperse cloud of gas and dust, which are often found in space. This cloud has its gravitational field, so it slowly starts to collapse on its centre of mass. This is an accelerating process: the denser becomes the core, the faster the matter falls on it. This core is called a protostar. Once the density in the core raises high enough, the nuclear reaction begins, and this moment could be considered as the birth of the star. The powerful hot plasma in the centre of the newborn star produces the necessary pressure to balance the gravitational attraction and prevents the matter from further collapse. Once the star lights up, its radiation swipes away the rest of the dust and gas and the star becomes visible for the outer world.

But not all the stars are so lucky. If there is not enough material, the star may never appear. Still the overall process goes the same way, just the resulting dense object is not as large and hot. Such an object is called a brown dwarf. These substellar objects have enough mass to maintain the nuclear reaction, but this reaction is not as powerful as the reaction in average stars: while in Sun, for example, the hydrogen transforms into helium, brown dwarfs use the fusion of heavier nuclei, which does not produce as much energy. Brown dwarfs are super stable -- they can glow for many billions of years.

Average stars become sooner or later the giant stars. This transformation would take place faster if the star was massive from the beginning. It happens because the hydrogen in the centre burns down to helium (and later to the heavier elements), and the fusion reaction starts to take place in the shell around the core, where the pressure is lower, and there are fewer factors to limit the size of the star. For the Sun (2$\clubsuit$) this phase will come on in about 5 billion years. Then it will probably engulf the Earth (if not, the life on this planet will be unbearable anyway).

When a massive star spends all the nuclear fuel, it explodes in a destructive supernova event (6$\clubsuit$), forming at the end either a black hole (J$\clubsuit$) or a neutron star (Q$\clubsuit$). But if the star was not massive enough, it would end up dumping off the outer shells while leaving the core heated up to 10$^7$ $^\circ$C. The thrown out material, lit by the X-rays from the remaining core, starts to shine, forming a colourful planetary nebula (this term is just historical and has nothing to do with planets). The core, in this case, is called white dwarf -- it is typically as massive as the Sun while being as big as the Earth. The planetary nebulae shine for several tens of thousand years. Then the gas becomes too dispersed, and the remaining faint white dwarf stays cooling down in complete loneliness. Our Sun, after passing the giant phase, will end its life on this branch of the stellar evolution tree.
}{4Clubs.png}

%--------------------------------------------------------------------------------

\card{Ab.png}{club.png}{Main sequence}{
The significant role in understanding the stellar evolution (4$\clubsuit$) played the so-called Hertzsprung–Russell diagrams. It is a kind of chart where the stars are plotted depending on their colour and luminosity. Let's first see what does it mean.

Remember, we told about the wavelength of light while sorting out the quantum mechanics (K$\diamondsuit$). The wavelength is directly related to the colour: for example, the photons that seem to be red to our eyes have the wavelength about 700 nm, while the violet photons have shorter wavelength -- about 400 nm (of course there is a whole spectrum of photon wavelengths). The precise wavelength of a stellar object depends on its temperature, as it was discovered by Planck (K$\diamondsuit$). Thus the blue stars on the sky are hotter than the reddish ones.

The luminosity means the brightness of a star. It describes how many photons it emits. The only way to increase the photon flux from the star is actually to increase the star itself. Thus it would be accurate to say that the bright stars are generally large. Don't forget that the distance also plays its role. But at least it is true that, for example, Betelgeuse, one of the brightest stars on the sky, is almost 1000 times larger than the Sun. While Alcor, one of the faintest stars, seen by the naked eye, is only slightly bigger than the Sun -- and it is tough to spot it, even though it is ten times closer to us than Betelgeuse.

It turns out that if you plot all the stars on the sky on a graph like this, you'll find a sizeable thick line that spans on the diagonal from the bright blue stars to faint red ones. That is for the majority of the stars, the size directly depends on the temperature. This line of stars is called the main sequence. Right after the birth from the cloud of gas and dust, stars come to the main sequence. Depending on their mass, they appear either on the left upper corner or on the right down one. Later they can move to the state of the red giant and then to the very small but hot star like a white dwarf or a neutron star. But the large part of their stellar career they spend on the main sequence.

The Sun (2$\clubsuit$) belongs to the main sequence too, and it is more or less at the centre of Hertzsprung–Russell diagram.
}{AClubs.png}

%--------------------------------------------------------------------------------

\card{6b.png}{club.png}{Supernova}{
Now we skip the cosmic ray card (5$\clubsuit$) to tell first about one of the primary sources of these rays -- supernova, a powerful and destructive explosion of a star.  We already mentioned that the massive stars at the end of their evolution finish up with a supernova. Let's consider how it happens in a little bit more details.

There are two main scenarios of a supernova explosion. The first mechanism is realised in systems of two stars rotating close to each other. It is common that one of these stars after some evolution finally becomes a white dwarf (4$\clubsuit$). This type of stars has a mass limit, called the Chandrasekhar limit. The white dwarf sucks out the material from its companion star. Ones it passes the Chandrasekhar limit it explodes as a supernova.

This is the so-called type one explosions. Since they all happen in the same circumstances -- Chandrasekhar limit is a very well defined value about 1.4 solar masses -- the brightness of these supernovas is an excellent indicator of distance: the fainter the star is, the farther it is from us. So it is possible to use these supernovas as a cosmological ruler. The problem of distance measurement is one of the most important ones in the cosmology (A$\spadesuit$).

Another scenario is possible for the massive stars, heavier than three masses of the Sun. By the end of its life, the star runs out of the nuclear fuel that was keeping star from shrinking under the force of gravity. So the star collapses under its weight, and its core turns to the extremely dense state of a neutron star (Q$\clubsuit$) of a black hole (J$\clubsuit$). At the moment, it happens the core releases its gravitational potential energy, which causes the supernova explosion.

The turbulent magnetic fields in the supersonic shock wave from the supernova can accelerate elementary particles to crazy energies, million times higher than in the largest particle accelerators on Earth (J$\heartsuit$). These particles are called the cosmic rays, and they are the subject of the next card (5 $\heartsuit$).

Supernova explosions are quite rare events: in our galaxy, they happen every about 300 years. The last close supernova was in 1987, but it was no in the Milky Way but in its dwarf satellite galaxy -- the Large Magellanic Cloud, some 168 thousand light-years away. Even on such a vast distance, it was visible to the naked eye. A supernova in 1604, described by many European, Chinese and Arabic scientists, was only 20000 light-years away from us. And it was visible even during the day for over three weeks.

}{6Clubs.png}


%--------------------------------------------------------------------------------

\card{5b.png}{club.png}{Cosmic rays}{
Supernovae (6$\clubsuit$) and other astrophysical sources can create fluxes of charged particles like protons, electrons and nuclei of atoms. These particles can have energy in a large range up to hundreds of exaelectronvolts, which is a hundred million times more than the highest elementary particle energy reached in accelerators. Such particles are called cosmic rays.

The name ``cosmic rays'' comes from the times when scientists started learning about some mysterious radiation but didn't yet know the nature of this phenomenon. The correct name would be ``cosmic particles''.

The number of cosmic rays drops very fast with the growth of energy. The most abundant cosmic rays, as we believe now,  come from the supernova explosions in the Milky Way. But, wait a second. Just a page ago we said, that the supernovae appear in our galaxy every several hundred years! Is it plausible that we have a continuous flux of cosmic rays from the sources which are so rare? In fact, instead of travelling straight through space, cosmic rays are constrained within the galaxy because of the galactic magnetic fields (7$\clubsuit$). On average a particle can spiral for millions of years, turning in the galaxy arms. During this time, the nuclei of the heavy elements, like carbon or oxygen,  decompose in the interactions with the interstellar medium, forming such light atoms like beryllium and boron. 100\% of these elements come from the decomposition of the heavier cosmic rays.

Quite recently scientists started to realise that the cosmic rays have a substantial impact on the star and even galaxy formation. Especially on the early stages of the newly formed galaxy, the so-called relativistic galactic winds which are created by the cosmic rays have a crucial implication on the matter distribution. Without cosmic rays, it is impossible to explain the formation of the galaxy arms (K$\clubsuit$). On the Earth scales, cosmic rays seem to be responsible for the initiation of clouds and bolts of lightning. Though cosmic rays studies started a hundred years ago, only now we begin to realise their real significance.

It is interesting to learn about the detection techniques for the cosmic rays. The ``low'' energy cosmic rays are detected by the specialised satellites on the Earth orbit. Mostly they are ordinary particle detectors, just taken out to space. The mentioned ``low'' energy part goes up to about a hundred TeV. The cosmic rays of even higher energy start being too energetic and too rare to detect them from space, so they are detected on Earth. It is not the primary particles that are detected in this case, but the {\it extensive showers} of the secondary particles, created in the interactions of the cosmic rays with the atmosphere. The detectors for the highest energy particles are required to observe an enormous volume of the atmosphere. The largest cosmic ray observatories have surfaced of the order of a thousand square kilometres.
}{5Clubs.png}

%--------------------------------------------------------------------------------

\card{Qb.png}{club.png}{Jocelyn Bell Burnell -- pulsars}{In 1967 Antony Hewish built the Interplanetary Scintillation Array -- a radio telescope, mainly designed for the observation of the recently discovered quasars (8$\clubsuit$). The young postgraduate student Jocelyn Bell worked there on her PhD thesis. She analysed the experiment data -- long paper stripes of chart-recordings. One day she noticed something strange: a signal from one particular point in the sky, that was pulsing with great regularity of a bit more than one second. Astrophysicists knew already the periodic processes, like, for example, the cepheids, variable stars, that vary their colour and brightness. But their period is in order of days, not seconds! Jocelyn's first idea was that it is the alien's radio signal; she even called it the little green men.

However, later astrophysicists confirmed that these unknown sources are not artificial. Yet, they are quite strange. These are neutron stars, or since they are fast pulsing -- pulsars. We already mentioned this exotic stars before, so you know that they appear after a supernova explosion of a massive star. The fusion reaction doesn't undergo in the neutron star core, so nothing can keep the material from shrinking to an extremely dense state. The atoms approach so close to each other that they disintegrate, the electrons become squeezed into the nuclei, turning the protons into the neutrons. So the star's core starts looking like a gigantic atomic nucleus, that consists of neutrons only and has the size of a city.

Remember you in childhood, when you were playing on a playground on a carousel. My favourite thing was to turn the carousel fast and then move as close as possible to the centre. The carousel was starting to rotating much-much faster! The same happens to the pulsars. The star before its death is rotating with some speed. But when its core turns to the neutron star, which is orders of magnitude smaller than the original star, it starts rotating super-fast. There are powerful jets of radiation from the magnetic poles of the neutron stars. These jets are rotating with the star. For an observer on the Earth, they look like if someone would rotate a flashlight on a lace. The fastest known pulsars make almost a thousand rotations per second.

For the sake of completeness, we should mention that the name of a pulsar is reserved for any astrophysical object that makes the pulsations. Sometimes scientists observe the white dwarfs that manifest the same behaviour. However, most of the pulsars are neutron stars, and quite often these terms are used as synonyms.
}{QClubs.png}

%--------------------------------------------------------------------------------

\card{Jb.png}{club.png}{Stephen Hawking -- black hole}{
We said that the gravity creates the so intense pressure inside a neutron star (Q$\clubsuit$) that it becomes like a giant atomic nucleus. It is so dense that a teaspoon of the neutron star material would weigh a billion tons! But still, you can guess that something creates a counter-force that keeps the neutron star from squeezing even more. It is the so-called pressure of degenerate neutron gas.  However, if you make a neutron star even more massive, the gravitational attraction would overcome even this pressure. The star, in this case, turns into the black hole.

The existence of the black holes was first guessed by a German physicist Karl Schwarzschild, who in 1915 solved the equations of general relativity of Einstein (K$\spadesuit$) for the simplified case of a star -- an isolated spherical non-rotating object. You have probably already heard that the black holes are such massive objects that even light cannot leave it. This is true. It happens because space next to the black hole becomes ``curved''. Moreover, the black hole itself represents an area of space which is extremely curved. In the centre of a black hole, there is the {\it singularity} -- or the point, where the curvature of space reaches infinity. Or, wait for a second, postulating infinities in a physics book is too offensive. Let's better say that we don't know what happens in the centre of a black hole. Still, what {\it would} occur in the point of infinite curvature? Well, if you would go from the border of a black hole to its centre, you would have to cover an endless path, even though you started some limited distance from it. There is an infinite space inside the black hole. So it is literally a hole in the fabric of space. And it is black because the light cannot leave it.

The distance from the centre of a black hole, starting from which a photon can finally leave it, marks its boundary and is called the {\it horizon}. We said that the physical vacuum (6$\heartsuit$) looks like a strange field where particles pop-out from thin air and disappear again at every moment in every point of space. What happens when a pair of particles appears next to the horizon of the black hole? Then it is possible that one particle leaves the vicinity of the black hole and another falls inside. For an outside observer, it looks like the black hole is emitting particles. That is it is losing its mass through this radiation, which is called after Stephen Hawking who discovered it. If the black hole is massive than it is hard for the particles to leave its gravitational field, so they take out only very little of momentum. Thus the large black holes evaporate -- this is the physical term for this phenomenon -- quite slowly. For the supermassive black holes, evaporation can take over $10^{100}$ years.
}{JClubs.png}

%--------------------------------------------------------------------------------

\card{7b.png}{club.png}{Cosmic magnetism}{
Cosmic magnetism starts with the well-known effects of the geomagnetic field,  that protects us from the large part of the cosmic radiation -- very abundant low-energy cosmic rays (5$\clubsuit$) and solar winds (2$\clubsuit$). The Sun also possesses a vast magnetic field. On a large scale, Sun creates the heliomagnetic sphere that spans much farther than the orbit of Neptune. These two fields act like shields from the galactic radiation, important protection that assures the existence of life on Earth.

The magnetic field on the scale of a galaxy forms a spiral-arm structure, which goes parallel to the pattern of the visible arms. The orientation of the field in the neighbouring arms is reversed. The galactic magnetic fields drive the mass flow and affect the formation of the spiral arms. The smaller-scale magnetic fields are also necessary for the star formation: without a magnetic field, the gas cloud starts rotating too fast and is not collapsing to the centre to make a newborn star.

The largest known magnetic fields exist in the clusters of galaxies (4$\spadesuit$). These fields are essential for the cluster dynamics, providing additional pressure to the material.

The larger magnetic field is, the weaker it is. The galactic field is about a million times smaller than it in the geomagnetic sphere. All they arise from different mechanisms, but there is something in common: electric current. Whenever you have an electric current, you inevitably get a magnetic field. There is no other mechanism, because, unlike for the electric field, magnetic field sources don't exist. The geomagnetic field is created by the currents in the core of the Earth. But how the magnetic fields appeared in galaxies is still a mystery. Theories say, that normally, it had to dye away in the young galaxies.
}{7Clubs.png}

%--------------------------------------------------------------------------------

\card{8b.png}{club.png}{Quasar}{
Most galaxies are pretty quiet islands of stars. But some of them have a nucleus, which turns active. The active galactic nuclei are called ``quasars''. Today we know, that there is a supermassive black hole (J$\clubsuit$) in the centre of a quasar. Its mass is typically from millions to billions of solar masses. The material around the black hole forms an accretion disc. When it falls on the black hole, it accelerates to such speed that it starts emitting extremely energetic radiation. It comes out in the form of two jets from the poles of the quasar. These jets are ones of the most probable candidates to create the ultra-high energy cosmic rays (5$\spadesuit$), the most energetic particles ever detected on Earth, that reach the energy of hundreds of exa-electronvolts.

Why not every galaxy possesses a quasar? It turns out that almost every galaxy has a supermassive black hole in the centre. The Milky Way does have one, but it is not a quasar. For a black hole to be a quasar, the accretion disk should be large -- even if you feed a black hole with a star, it provides not enough material to form the accretion disk and to turn the black hole into a quasar.

However, we have evidence that our central black hole (by the way, it is in the constellation of Sagittarius and is called Sgr A$^*$) was a quasar earlier in its life. Today we observe hard photons coming from the gigantic gaseous structures on each side of the Milky Way. They span for about the same distance from its plane as its radius. These are the Fermi bubbles. Perhaps they were created during the quasar phase of Sgr A$^*$.

Nowadays quasars are studied by the means of the multimessenger astronomy: they are observed in a wide range of frequency of electromagnetic radiation, from radio to X-rays. The origin of ultra-high energy cosmic rays from quasars is yet not confirmed. And eventually, we can detect quasar neutrinos (7$\heartsuit$). That would be extremely valuable to constrain the existing models of the extreme quasar environment.

Quasars are the most bright sources of light in the Universe. But a quasar can be even brighter if its jet is pointed towards us. In this case, it is called a blazar -- but physically it is still the same object.
}{8Clubs.png}
%--------------------------------------------------------------------------------

\card{9b.png}{club.png}{Exoplanets}{
Exoplanets are the planets that rotate around other stars, just like the planets of the Solar system (3$\clubsuit$). For today over four thousand exoplanets have been discovered, revolving around nearly three thousand stars.

There are many methods to observe exoplanets. The first exoplanets were found on the orbits around a pulsar (Q$\clubsuit$) in 1992. Those planets influenced the rotation of the host neutron star, which was observed as the variability of its pulses. However, these planets would be better described as debris after the supernova explosion (6$\clubsuit$) that created the neutron star. The first exoplanet that rotates around a star of the main sequence (A$\clubsuit$) was discovered in 1995. It was a super-heavy planet that rotates exceptionally close to its host star: its orbiting period is only three days -- compare it to the period of Mercury which is about two months. Planets of this type were named hot Jupiters: they are hot because they are in the very vicinity of a star, and they are as massive as Jupiter. For this kind of systems, you better say that the star and the planet rotate around each other. You can detect the speed of the star by the slight change of its colour. From this observation, you can conclude about the existence of an exoplanet.

The method of transit photometry, depicted on the card, allows detecting smaller planets. When a planet passes in front of the star, it casts a shadow and makes the star look slightly fainter. This method allows for detecting the size of an exoplanet. But it is only applicable when the exoplanet rotates in the plane, parallel to our line of sight. Most of our detection methods suffer from the {\it observation bias} -- the fact that we only observe the observable objects. So to estimate the real number of exoplanets in our neighbourhood, scientists have to build models and correct them with what we see. The number of observed planets will always be lower than their total amount.

So one can conclude that there is a lot of exoplanets. And many of them are in habitable zones, that is they are close enough to their host stars to have liquid water on the surface. If besides, they have enough carbon, they could become inhabited. If this extraterrestrial life would turn conscious and have civilisation, eventually they could start asking the same questions as us. And maybe, if they would be technologically developed, they would send us a radio signal, or would visit us in person. But it didn't happen so far (some people say that it's for good). This paradox is called after the famous physicist Enrico Fermi: we see there are many exoplanets that can maintain life, but why we don't see the aliens? Could it happen that intelligent life is extremely rare in the Universe and we are alone in our galaxy? Or maybe the developed alien civilisation realise some truth that stops them from sending signals here and there? Nobody knows the answers to these questions, but it should not cease us from the investigation and guessing!
}{9Clubs.png}
%--------------------------------------------------------------------------------

\card{10b.png}{club.png}{Gravitational wave}{
Most of the discoveries in astrophysics are made through the observations of the light. It could be radio-signal, or visible light, or X-ray light, still it is essentially the same -- electromagnetic radiation. One can also study the cosmic rays (5$\clubsuit$), but it is more complicated: since the cosmic rays have charge, they get deflected in the galactic magnetic field (7$\clubsuit$), so they don't point back to their source. Thus we can only guess about their origin. Neutrinos (7$\heartsuit$) are more reliable: they are very light, and they have no charge. However, neutrinos interact so weakly (4$\heartsuit$) that they are tough to detect.

A completely new type of astrophysical detection appeared only a few years ago. In 2015 three observatories in America and Europe detected their first gravitational wave. First predicted by Einstein, these waves imply the periodical contraction of the space-time fabric itself. Only the large masses in tremendous acceleration can create them. For example, that first gravitational wave was created by two black holes, each about 30 solar masses, rotating around each other 75 times per second at the distance of only 350 km. That was just before they merged. At this brief moment, they emitted an incredible amount of energy, equivalent to 3 solar masses, in the form of a gravitational wave. This wave eventually reached us and created the space contraction of just about $10^{-18}$m, or a thousand times smaller than the size of a proton. The detection of such a tiny effect became possible thanks to an ingenious interferometric nature of the observatories and scrupulous accuracy of its construction.

A gravitational wave is a unique instrument to study the black holes, which are otherwise invisible. Knowing the masses of the black holes, scientists can conclude about the stars that created them, what were the supernova events, and how they affected the surroundings. Observation of the gravitational wave is also an unprecedented probability to test the theory of general relativity (K$\spadesuit$), which is the current standard theory of gravity and the foundation of the modern cosmology. 

Up to now, we count about a dozen gravitational wave events. Most of them came from the mergers of two black holes, but there is also one significant event from the merger of two neutron stars. These objects are so dense that they can create a detectable gravitational wave too. The future observatories will be able to detect events like this regularly.
}{10Clubs.png}

%--------------------------------------------------------------------------------

\thispagestyle{fancy}
\fancyhf{}
\renewcommand{\headrulewidth}{0pt}
\lhead{\thepage \hskip14pt Physics Is My Favorite Game}
\fancyfoot{}
{\huge{\textbf{Suggested materials}}}
\vskip12pt
The astrophysics is so broad as a field that it is hard to cover it with just one book or one resource. But it is also the branch of physics that you can learn in the hands-on mode. Buy a cheap telescope -- even the simplest ones you can find today in the shop are much better than the telescope of Galileo -- and you can start studying the Moon craters, see the Jupiter and its moons, see the Saturn's rings and some bright nebulae. Or even without a telescope -- go to a place where you wouldn't have the city lights in the night, and enjoy the majestic view of the Milky Way. Install Stellarium (stellarium.org) on your computer and learn our stellar geography with this virtual planetarium.

Try to start following the astro-news, for example, on universetoday.com. In the beginning, you will feel lost, but after some time, you'll get used to it and will start to understand the discourse. I swear it worth it!

Follow the news from the leading observatories. Most of them have an account on Twitter. Look for the posts of the European Southern Observatory (ESO), ALMA radio observatory, Pierre Auger cosmic ray observatory, IceCube neutrino telescope -- already not bad for starters.

For the popular books, I can recommend the ``Astrophysics for People in a Hurry'' by Neil deGrasse Tyson. Unfortunately, if you look for something really tough and serious on the astrophysics as a whole, it is hard to give you any recommendation. In fact, for every card of the club suit, you would find quite a thick book that explains the subject in details.
\newpage
\mypart{Cosmology -- Spades}{Cosmology \\Spades}{spade.png}{We step into the realm of cosmology -- the branch of physics that studies the largest possible object, the Universe as the whole. How did it appear? What is its content? What is its ultimate fate? These are the questions we address in cosmology.}

%------------------------------------------------------------------------------------

\card{Kb.png}{spade.png}{Albert Einstein -- Relativity}{The theory of relativity takes the beginning at the end of XIX -- the beginning of XX century when physicists experimentally established a bizarre fact: the speed of light depends neither on the speed of the light source nor on the speed of the observer. It sounds weird from the everyday experience: if you drive on a highway with the speed of 120 km$/$h and someone is overtaking you, having speed 130 km$/$h, then his speed relative you will be only 10 km$/$h. But for the light it is not so -- the light still moves with the same speed of about $3\times10^8$m$/$s, no matter how fast you fly.

But it means that the light is always moving this fast, and there is no observer, who would be able to study a photon at rest. It means that nobody that has mass can move with the speed of light. If you apply these conclusions to the mathematical formulae, you inevitably get quite funny results that the time and space can contract and stretch, depending on one's speed {\it relative} the observer. This is the foundation of the special theory of relativity.

If you add the acceleration into the scope, you get even more odd results. Imagine you are in a space rocket. Its engine is on, and it is accelerating very fast. Imagine then that you turn on your flashlight and shine across the rocket. Let's say you hold the flashlight at one meter from the floor. When the light reaches the wall of the rocket, it will hit it slightly lower than one meter, because the rocket is accelerating. So in the {\it reference frame of rocket}, you would observe that the light flew on the curved trajectory.

The general theory of relativity postulates that the gravitational field is equivalent to the situation with the accelerating rocket. And so the light beam gets curved next to the massive bodies. Physicists prefer speaking about the light going on straight lines in a curved space instead. This way, you can introduce the notion of the curved space-time. Its curvature is created by the distribution of matter in the Universe, and it manifests itself by what we call the force of gravity. The general theory of relativity, designed by Albert Einstein, is the standard theory of gravity. And since gravity is the only force that acts on the cosmological distances, this is the base theory of cosmology.

Einstein published his general relativity in 1915, and in 1919 it was confirmed experimentally: during the solar eclipse, the stars that would be normally hidden behind the Sun were still visible thanks to the curvature of space around the Sun. Today, using the large modern telescopes, we can observe even more spectacular confirmations of this theory. Just try searching on the internet the images of Einstein rings.
}{KSpades.png}

%------------------------------------------------------------------------------------

\card{2b.png}{spade.png}{Redshift}{We said that Hubble (K$\spadesuit$) has proved that the Milky Way does not limit the size of the Universe. If you remember, I promised to tell the continuation of that story.

Later, in 1927 – 1929, Edwin Hubble discovered, that the galaxies don’t stay still, but move away from each other. Using spectroscopic data, he deduced the famous law that bears his name: the farther the galaxy, the higher is its receding speed from us.

By the spectroscopic data, we mean the following: the hydrogen atoms in a distant galaxy absorb the light at some well-defined frequencies (4$\diamondsuit$). So if you measure the spectrum of that galaxy, you see some narrow gaps. Hubble observed the spectra of many galaxies and saw everywhere the same pattern of gaps. But oddly the spectra of different galaxies seemed to be shifted relative to each other. Most of them appeared to be redder than they should be. This effect is quantified in the measurement called {\it redshift}. Remember that Hubble knew how to measure the distance using the cepheids (see K$\spadesuit$ again). So he just plotted the distance versus the redshift and discovered, that the farther the galaxy is, the more it is redshifted. Today we know that the redshift is caused by the expansion of the Universe.

How does the light turn red when the Universe expands? The expansion of the Universe in terms of the general relativity (K$\spadesuit$) is the stretching of its space. The photon has a wavelength (K$\diamondsuit$), which is also stretching together with space it is passing. But the stretched light is the red light! The light needs some time to travel from a distant galaxy to us. During this time the Universe gets stretched, and so the light becomes redder. This can be different for the galaxies next to us. For example, the already mentioned Andromeda galaxy is actually blueshifted, because it is moving towards us. But this is because of the local interactions. On the global scale, the Universe is expanding, and it is seen as the redshift of distant objects.
}{2Spades.png}

%------------------------------------------------------------------------------------

\card{Jb.png}{spade.png}{Lema\^{i}tre -- Big Bang}{In 1927 Belgian astronomer and Catholic priest Georges Lema\^{i}tre learned about the Hubble's result and gave his own explanation to the global Universe expansion. He built a model of changing of the space curvature (K$\spadesuit$) radius with time. Actually, he was the first who wrote the Hubble law, mentioned a page ago, and he also made the first estimation of the Hubble constant -- the coefficient of proportionality between the distance to the object and its receding speed. He proposed an interesting idea, that as the Universe now expands, maybe before it was just a point-size. He called this ``hypothesis of the primaeval atom'' or the ``Cosmic Egg''. Less poetically physicists started calling this model as a model of ``hot Universe'', meaning that it was in a hot and dense state at the beginning. However, the correct -- and a bit boring -- term of hot Universe was later replaced by the name ``Big Bang''. This term was first invented by an astrophysicist Fred Hoyle. Ironically, he was a strong opponent of the hot Universe idea -- and yet today we use this naming.

When people hear about the Big Bang they usually imagine a kind of a global explosion. It is not correct. Our Universe has three space dimensions -- and for us, it is hard to understand the expansion of the three-dimension space. Let's simplify the situation. Imagine an infinite rubber band -- this will be our one-dimensional Universe. Then let's draw a mark on every meter of the band. Imagine now that you start stretching this elastic band. This is the expansion of the Universe. By the way, this is a nice illustration of the Hubble law too. If the receding speed for the marks which are 1 meter away from each other is 1 cm per second, then it is easy to realise that the marks which are 2 meters away will recede from each other by 2 cm per second -- twice faster, just as stated by the Hubble law. Our galaxy Milky Way sits on one of these ticks. And for us, every point of the Universe is receding from us. Moreover, it is valid for any observer in this infinite Universe. So we better stop speaking about the galaxies that fly away from us. The righter picture is the expanding space between the galaxies.

The situation with the stretching rubber band was the starting point for Lema\^{i}tre, with the only difference that he worked in three dimensions. Now let's repeat his way of thinking. If we are observing the Universe which is stretching now, it is quite logical to try extrapolating this situation back in time -- in this case, you'll see how the elastic band is contracting as you approach the point of zero time. What is that zero time then? Obviously, it is when the rubber band is contracted completely, down to the infinite density. Does this infinite density state has any physical meaning or not -- you'll learn it in the very end of this book (10$\spadesuit$).
%Of course, for Lema\^{i}tre the idea of the Cosmic Egg had a very strong religious sense.
}{JSpades.png}

%------------------------------------------------------------------------------------

\card{3b.png}{spade.png}{Cosmic microwave background}{
Right after the moment of the Big Bang, the Universe was in a hot and dense state. With time the Universe expanded became less dense, and the temperature has dropped. Why does the temperature drop down? For the same reason as the photons become redshifted: if, for example, in some volume, you had 1000 blueish hot photons, after the expansion of the Universe by factor 10 in size, you would get only ten photons in the same volume, and they will be redshifted by a factor of 10.

Cosmologists computed that after the Big Bang, the Universe was full of the hot plasma. The protons and electrons were flying freely. And when eventually they collided and formed the hydrogen atoms, these newly created atoms were immediately disassembled by the high energy photons that also flew around. This early ionised Universe was opaque to the photons: they could not travel far without colliding with anything. But when the Universe was about 400 thousand years old (very young age, comparing to the total age of the Universe today, which is almost 14 billion years), the temperature of these photons dropped below the point when they could maintain the plasma ionised. The protons combined with electrons, the plasma died away, and the photons were set free. These ancient photons we observe today as the cosmic microwave background.

CMB was discovered by mere chance in 1964 by A. Penzias and R. Wilson from the Crawford Hill Laboratory. The antenna they were using was very sensitive, with a very low noise level. But they found the registered noise exceeded the noise observed in the laboratory. At first, they supposed that this noise was coming from the Earth. But funny enough, they got the same results, completely independent of the antenna orientation. Maybe it was the signal coming from the galaxy? They pointed the antenna on the dark part of the sky outside the Milky Way. Still, the observed signal was substantial. After some study and extensive discussions with colleagues, they understood that they found the relic radiation from the Big Bang.

The cosmic microwave background radiation is one of the most critical evidence for the theory of the hot Universe and takes an outstanding role in modern cosmology. It is a source of extensive information about the early stages of the Universe history. This background has the spectrum of a black body (K$\diamondsuit$) with a temperature of $2.7$ K, or about $-270^\circ$C, and it follows the theoretical spectrum with extremely high precision. It is the most precise measurement in the whole cosmology.
}{3Spades.png}

%------------------------------------------------------------------------------------

\card{4b.png}{spade.png}{Cosmic web}{
The basic cosmological principle says that on the large scales the Universe becomes homogeneous (has equal density everywhere) and isotropic (looks the same in every direction). Indeed under the assumption of the homogeneous Universe, scientists have predicted the Big Bang, the cosmic microwave background, calculated the age of the Universe etc. However, the very book you are holding in your hands is proof that the cosmological principle is wrong on the small scales. The book is dense, while the air around it is sparse, so on the small scales, the Universe is obviously inhomogeneous.

It turns out that the Universe starts being homogeneous only on the very large scales, more than a half-billion light-years. Going from small sizes to large ones, one can build a ladder of scales. Star clusters have a size of about one light-year. Galaxies have diameters on the order of a hundred thousand light-years. The galaxies, on their turn, form clusters of galaxies (10 million light-years) and super-clusters of galaxies (100 million light-years). Now, this is the size when we start seeing the web structure of the Universe: between the galaxy clusters we see the gigantic filaments, filled with galaxies and extragalactic gas. The empty spaces, known as the cosmic voids, are observed between the clusters and filaments. The hierarchical structure of the Universe was produced by gravitational instability of some random perturbations of density very early in the history of the Universe (9$\spadesuit$).

The are many fascinating methods to study the large-scale structure of the Universe. One can observe the supernova explosions of the white dwarf (6$\clubsuit$). They are incredibly bright, so they are visible from far away. And one can measure the distance to them, thanks to the known brightness of this kind of source.

Another method involves the study of quasars (8$\clubsuit$). They are even brighter than the supernovae so that one can measure even farther distances with them.

But most of the large-scale structure studies are done with just the measurement of the redshift (2$\spadesuit$). The redshift is not a very reliable measurement of the distance, because it suffers from different distortions. For example, if the observed galaxy is moving along the line of sight it will become additionally red- or blueshifted (if it moves away or towards us correspondingly), so it will seem more or less distant than it is in reality. However, scientists have methods to take these distortions into account and to reconstruct the global picture.
}{4Spades.png}

%------------------------------------------------------------------------------------

\card{5b.png}{spade.png}{Element abundances}{
From the very beginning of the Big Bang theory, its success was based mainly on the excellent agreement of predictions with observations for the abundances of the light elements in the Universe. Quite an illustrative fact for this is that the Nobel lecture of Arno A. Penzias is called ``The Origin of the Elements''. However, he was awarded the Nobel prize for the discovery of the cosmic microwave background (3$\spadesuit$). At that time physicists thought that all the elements were synthesised in the hot plasma at the beginning of the Universe evolution (precisely, in his Nobel lecture Penzias analyses the history of views on the element formation and marks that the opinion on this issue changed several times between the 1930s and 1970s). It was found out later, that only light elements could form in primordial plasma, and the heavier elements appear much later during the evolution of the stars.

The typical explanation of primordial nucleosynthesis (formation of nuclei) is the following: as the Universe cooled down when the energy of photons dropped below the binding energy of some nuclei, the photons did no longer break these nuclei to protons, and the element began to form. This mechanism is called {\it freezing} -- it is interesting to note that in Russian literature it is called with the same word as used for the steel hardening as if the Universe is a colossal blacksmith's shop. The freezing of the lithium nuclei happened about a couple of minutes after the Big Bang. The lighter elements got formed even earlier.

The average number of different elements in the modern Universe is a well observable variable. In this study, we are interested in the lightest elements: isotopes of hydrogen and helium and lithium. All these atoms could be found in the rare interstellar gas. The most abundant hydrogen atoms have a density of about one atom per cubic centimetre inside the galaxy, but there is much less hydrogen in the intergalactic space. Helium is about four times less abundant. The deuterium -- an atom that has one proton and one neutron in its nucleus -- and the isotope of helium $^3$He are more than ten thousand times less common than hydrogen. And lithium is so rare that there is only one atom of lithium per more than a billion atoms of hydrogen. So this is a measurement that spans nine orders of magnitude. And the most impressive is that all these measurements are in the perfect agreement with the prediction from the Big Bang nucleosynthesis!
}{5Spades.png}

%------------------------------------------------------------------------------------

\card{Qb.png}{spade.png}{Vera Rubin -- Dark matter}{
In 1932 Fritz Zwicky noticed, that besides the luminous baryonic matter of galaxies there are invisible, hidden masses in the Universe that manifest themselves only through the gravitation. Zwicky studied the galaxy cluster in the constellation of Berenice's Hair. And he discovered that the speeds of the galaxies in this cluster are tremendous, up to few thousand kilometres per second. To hold down such fast-moving galaxies within the cluster the vast gravitational force is needed, much higher than the gravitational force from the galaxies themselves. Later, in 1970s Vera Rubin discovered that the hidden masses present not only in the clusters of galaxies but in the isolated galaxies as well. Invisible dark matter forms spherical halos around galaxies. The radius of a halo is typically 5-10 times bigger than the radius of the galaxy. The discovery of Rubin was genuinely ground-breaking and has introduced the dark matter to the standard model of cosmology.

There are more phenomena through which the dark matter can manifest itself. For example, the galaxy clusters are filled with gas. Its high temperature can be explained only by taking into account the dark matter component of the cluster. Or another example: the mass of a galaxy cluster can be independently estimated by the effect of gravitational lensing when the light of a background galaxy gets curved by the cluster. Again, its total mass turns out to be much larger than the mass of the visible matter. Remarkably, all the different observations give the same estimation to the relative amount of dark matter. It is about five times more abundant than the normal (called {\it baryonic}) matter.

Some scientists try to explain the effects listed above by the hypothetical difference of the gravity law on big distances from that on smaller scales. But so far no modification of the theory of gravitation could not explain the observations thoroughly.

The most commonly accepted model for the dark matter says that it consists of weak interacting massive particles (WIMP). If this model is true, then many of these particles are crossing the Solar System, Earth and you without any interaction. Many experiments around the globe are trying to detect the signal of such particles, but yet unsuccessfully. Some WIMP models built on the idea of the supersymmetry (9$\heartsuit$).
}{QSpades.png}

%------------------------------------------------------------------------------------

\card{6b.png}{spade.png}{CMB anisotropies}{
The cosmic microwave background (3$\spadesuit$) is exceptionally uniform. No matter where you would point your microwave antenna, you will always measure the same temperature of about 2.7K. Still, there are tiny variations of its temperature, on the order of a milli-Kelvin. Imagine a one-meter thick brick wall. And this wall is so smooth that the most considerable bumps and deeps on it measure only one-tenth of a millimetre. I guess you would say that this wall is very well polished. So is the cosmic microwave background.

These tiny fluctuations of temperature -- or they also say the {\it anisotropies} -- take place from the non-uniformity of the matter distribution in the early epochs of the Universe expansion. These are the same non-uniformities that later formed the large-scale structure of the Universe (4$\spadesuit$). The regions where the cosmic microwave background is slightly colder correspond to the areas, where the plasma of the early Universe was denser: the light spent some energy to get out of the gravitational potential of this over-density and thus became a bit colder. It means that we can directly study the distribution of matter in the early stages after the Big Bang, and most interesting is that we can do it with high precision. The cosmic microwave background (CMB) anisotropies can tell a lot about the content of the Universe. How much baryonic and dark matter there is? How curved on the global scale is space? How old is the Universe? And many other critical cosmological questions could be answered in this study. Current experiments have established these parameters with a precision below 1\%. But still, there are questions without an answer. One of the most intriguing ones is the amplitude of the so-called {\it primordial B-modes} of the CMB polarisation. It is a weak signal that was imprinted on the CMB by the gravitational waves (10$\clubsuit$) which were created in the epoch of inflation (9$\spadesuit$) right after the Big Bang. Many experiments tried to measure it, but yet unsuccessfully. One of the strongest effects that make this task extremely difficult is the light polarisation from the elongated dust grains in the magnetic field of the Milky Way (7$\clubsuit$).

The different structures that appeared later in the Universe make the foreground for the CMB, and they get imprinted on it. For example, the photons of CMB get lensed by the presence of the cosmic web -- so the latter can be studied through the lensing of the CMB. Another example: when a CMB photon passes through a massive galaxy cluster, it can be heated up by the collision with the hot gas. It is the so-called Sunyaev-Zeldovich effect. It is a powerful tool to study the cluster dynamics.
}{6Spades.png}

%------------------------------------------------------------------------------------

\card{7b.png}{spade.png}{Reionization}{
When the early hot plasma cooled down, and the protons finally recombined with the electrons (3$\spadesuit$), the Universe finally became transparent. And all of a sudden, instead of being full of fire, it became dull and empty. The so-called dark ages of the Universe began. Nothing happened in this time, and only the rare hydrogen atoms were flying around for several dozens of million years.

All this time, the dark matter (Q$\spadesuit$) was continuing to form its dark structures. And the hydrogen followed the gravitational attraction of these structures, creating the clouds. Eventually, these clouds became dense enough to make possible the birth of the first stars (4$\clubsuit$). The stars were quite different from those that we observe today because they were made of the hydrogen and a bit of helium -- but virtually nothing else. In these circumstances, the stars turned out to be much bigger, hotter and brighter than the present ones. Their hard X-ray radiation re-ionized the hydrogen atoms around. This is the epoch of reionization. The hydrogen atoms absorbed the light of the first stars, mainly on the wavelength of 21 cm (4$\diamondsuit$). Thanks to this today we can study this effect in the radio spectrum.

The fusion reaction (10$\diamondsuit$) in the massive early stars was going on an increased rate, creating the heavy elements. When they were exploding in supernova explosions (6$\clubsuit$) they were spreading this material around, setting the environment for the next generations of stars, planets and everything else, like for example this deck of cards.

The early power supernova explosions created the flows in the cold scenery of the dark ages. These flows started clumping up in the already existent dark matter structures. By the end of the day, this material formed the galaxies we see today. The Universe turned from the violent chaos to a beautiful starry world, where several billion years later, on the little rocky planet Earth the life has appeared.
}{7Spades.png}

%------------------------------------------------------------------------------------

\card{8b.png}{spade.png}{Dark energy}{
The observations of the supernova explosions of white dwarfs (6$\clubsuit$) show that today the Universe expands with the growing rate. This fact could be explained by the presence of some form of energy. However, since nobody has yet succeeded to detect it directly, we call it dark energy. In 2011 the Nobel Prize in physics was awarded to Saul Perlmutter, Brian P. Schmidt and Adam G. Riess for their leadership in the discovery of the accelerated expansion of the Universe and hence the existence of the dark energy.

By the way, it is not late yet to clarify one interesting point. The speed of light is finite -- and this gives the cosmologists the superpowers of seeing back in time! Indeed, we can only observe the light that was emitted some time ago. Even for the closest cosmological sources, this time is millions of years. For the furthest ones it reaches almost 14 billion years -- this is true for the CMB light. It means that to study the Universe at earlier stages of its evolution, one has to simply look farther. Or, speaking more precisely, analyse the data with the larger redshift (2$\spadesuit$). Perlmutter, Schmidt and Riess did precisely this. They studied the different slices of the Universe on different redshifts. And they were measuring the change of its expansion rate, which turned out to be increasing in the last $\sim 5$ billion years.

There are at least two more pieces of evidence for the presence of the dark energy. The growth of galaxy clusters is defined by two counteracting processes: gravitational contraction and repulsion due to the dark energy. We can measure the dependence of the galaxy cluster masses on the redshift to get its evolution with time. According to this study, approximately 70\% of the total energy-density budget of the Universe constitutes of dark energy. This result is conforming with the measurement through the expansion rate as described above.

Another independent evidence for the presence of the dark energy could be given by the gravitational lensing of the CMB (6$\spadesuit$).

Since the Universe is now expanding with acceleration, you may ask yourself: what would be with our Universe in the future. The answer to this question is still not clear, and it entirely depends on the intrinsic nature of the dark energy. Some models predict that the Universe will continue accelerating and at some point, all the structures (even the atoms) will be destroyed by the ripping force of the dark energy. Other models predict the bouncing Universe, which will eventually return to the state it had at the moment of the Big Bang and the story will start over. Anyway, all these scenarios require a very long time, many orders of magnitude longer than the current age of the Universe.
}{8Spades.png}

%------------------------------------------------------------------------------------

\card{Ab.png}{spade.png}{Observable Universe}{
It is time to summarise what we have learned so far. About 13.8 billion years ago, the whole Universe was in an extremely hot dense and state. Since then, it was expanding and cooling (J$\spadesuit$). In the first three minutes after the beginning of the expansion (or the Big Bang), the quarks formed the protons and neutrons, which on their turn produced (5$\spadesuit$) the observed today hydrogen, helium and lithium nuclei (the latter two in small quantities). About 400 thousand years after the Big Bang, the primordial plasma died away, and the Universe became transparent. The afterglow of that plasma is observed today as the cosmic microwave background, which is still the dominant form of radiation in the Universe (3$\spadesuit$). The baryonic matter, under the gravity of the dark matter, started clumping, creating gradually growing structures, from galaxies to the cosmic web (4$\spadesuit$). About 5 billion years ago, the matter became so sparse that the dark energy became dominant in the energy budget of the Universe, and the expansion rate started growing (8$\spadesuit$). Today the Universe has zero global curvature (K$\spadesuit$), and it consists on about 70\% of dark energy, 25\% of dark matter and the rest 5\% of the ordinary baryonic matter. This is the standard model of cosmology. More rigorously it is called the $\Lambda$CDM cosmology, where $\Lambda$ (capital greek letter lambda) is the physical term for the dark energy and CDM stands for the Cold Dark Matter.

Thanks to the finite speed of light, we can observe different epochs of the Universe to varying distances from us. The distant objects become redshifted because space gets expanded while the light was travelling from those objects to us (2$\spadesuit$). The most distant object that we can observe is the spherical surface at which the CMB photons last time interacted with the primordial plasma. This is the observable Universe. It is correct to say that we are sitting in its centre. But this is not because of the preferred position of the Earth. Actually, any point in the Universe can be designated to be its centre.

Formally the observable Universe spans even farther. Imagine that at the moment of the Big Bang, a photon was emitted from the point where the Earth is now. And that this photon travelled for 13.8 billion years without getting scattered on anything. Due to the Universe expansion, it would be now not 13.8, but about 46 billion light-years from Earth. This is the radius of what they call the observable Universe -- even though in reality you can't see farther than the CMB sphere.
}{ASpades.png}

%------------------------------------------------------------------------------------

\card{9b.png}{spade.png}{Inflation}{
The $\Lambda$CDM model, described on the previous page (A$\spadesuit$) is not just a beautiful theory -- it allows doing precise measurements of various cosmological parameters in many different independent ways and getting consistent results. Still, there are some open questions. Why don't we observe the antimatter (K$\heartsuit$)? Why the Universe has no curvature on the global scale (K$\spadesuit$)? And why the CMB is so uniform (3$\spadesuit$)? 

Let's consider the last question in more details. One can define the cosmological horizon: it is the maximal distance at which two objects may have influenced each other since the Big Bang (J$\spadesuit$). If a point is outside the horizon of another location, they should normally look completely different. The problem is that the region we see today as CMB is much larger than the horizon at the time when CMB was released. That is a CMB photon that comes from one side of the sky ``knows'' nothing about the CMB photon from another side of the sky. Hence, it is natural to expect that they would have a completely different temperature. And still, they are very much the same. This situation is as strange as if I would travel to the Easter Island (I've never been there) and would find that the islanders know me and I know them.

However, the situation would be less weird if we assume that me and the Easter Island residents left together some time ago and then one day moved -- they to their island and me to Europe. Cosmologists hypothesise that something similar happened to the Universe. Right after the Big Bang, before the Universe became $10^{-34}$ second old, it expanded on the accelerated rate and every tiny region of it suddenly became much larger than the whole observed Universe today. When you take a non-uniform Universe and blow up a tiny piece of it, naturally it becomes uniform. So, no surprise, we observe a uniform CMB today. This theory is called the theory of inflation. It explains the curvature of the Universe too: even if in the beginning the space was curved, it flattened out during the inflation period. And if just by chance it happened that this tiny little piece of the Universe before inflation had a bit more matter than antimatter, than this disbalance will maintain in the mature Universe.

If the inflation took place, it had to create a huge gravitational wave, which after 400 thousand years had to leave an imprint on the polarisation of the CMB. This imprint should have a specific curled pattern called  B-modes. Many CMB experiments are chasing it, but this weak signal suffers from strong foregrounds. One, as we said, comes from the dust grains in the Milky Way. Another comes from the distortion of the CMB signal by the large-scale structures (4$\spadesuit$). Should be said that the theory of inflation doesn't make a part of the $\Lambda$CDM cosmology.  Still, not many cosmologists doubt that it did happen.
}{9Spades.png}

%------------------------------------------------------------------------------------

\card{10b.png}{spade.png}{Planck epoch}{
The whole Universe is described by the combination of quantum mechanics (A$\diamondsuit$) that acts on the small scales and the general relativity (K$\spadesuit$) that works on the large scales. The quantum effects become unnoticeable even on the millimetric scale. And the gravity effects usually start to manifest themselves on the much larger distances. So these two realms never coincide.

The main fundamental constant that rules in quantum mechanics are the Planck constant $h$ (K$\diamondsuit$). And the main one in general relativity is the gravitational constant $G$ that gauges the strength of the gravity. If we add here the speed of light, then from these three, it becomes possible to construct the so-called Planck units: Planck length and Planck time. These units have a fascinating meaning. On the distances smaller than the Planck length and the time scale shorter than the Planck time, one can neglect neither general relativity nor the quantum mechanics. Instead, one has to use some mixture of those two. But the corresponding theory is not yet invented so for the moment we {\it don't know} what happens on the Planck scales. (However, we have some ideas. Take a look at the jokers.)

The Planck length is equal to $1.6\times10^{-35}$m and the Planck time is $5.4\times10^{-44}$s. As you can see, these are extremely short distances and an extremely brief time. However, we can still speak about the time when the Universe was younger than the Planck time and had a size smaller than the Planck length. That is the Planck epoch.

Very often people think about the Big Bang like an actual explosion that happened at the moment zero when all the distances were zero (virtually the Universe was just a point) and the density was infinite. In fact, we know nothing about the Universe before the end of the Planck epoch. We don't even have a theory to try to approach it. Moreover, we don't even know what are the space and time at the Planck scales. It was probably a crazy soup of multidimensions. The Big Bang is the model of expansion of the Universe from some hot and dense state {\it after} the Planck era. It is equally pointless to ask physicists about what was before the moment zero. Some speculate that the Universe existed before and it shrank down. Others think that the Universe was created from the quantum fluctuations of vacuum (6$\heartsuit$). But the most honest answer about the Universe before the Big Bang --  we don't know anything about it.
}{10Spades.png}

%--------------------------------------------------------------------------------

\thispagestyle{fancy}
\fancyhf{}
\renewcommand{\headrulewidth}{0pt}
\lhead{\thepage \hskip14pt Physics Is My Favorite Game}
\fancyfoot{}
{\huge{\textbf{Suggested materials}}}
\vskip12pt
For other parts of this book, I was suggesting non-reading materials first. But here I cannot avoid recommending reading the ``Brief History of Time'' of Stephen Hawking.  Another excellent book is ``The First Three Minutes'' of Steven Weinberg. It will show you all the beautiful connections between cosmology and particle physics. But if you aim to become a prominent specialist in the field, start with the ``Modern Cosmology'' by Scott Dodelson.

For the non-printed resources, cosmology very much coincides with the astrophysics (in fact, not everybody makes a distinction between the two). So here my recommendations would be the same: www.universetoday.com, ESO's Twitter... Don't forget the phys.org news website. The fields of cosmology and astrophysics go there under the same tab: ``Astronomy and Space''.

There are very exotic notions in cosmology, which I didn't dare mention. You would be interested to learn about the cosmic strings, monopoles, white holes and multiverse. I recommend the Fermilab channel on Youtube with a bunch of brilliant lectures.
\newpage

\mypart{Theories of everything -- Jokers}{Theories of everything \\Jokers}{}{The theories of everything, as it follows from the title, try to explain all the different interactions with one single idea, with the same approach. There are two leading candidates to be such a theory: the string theory and the loop quantum gravity. However, both of them are yet incomplete. Physicists are trying to solve the perplexing mathematics of these theories, but still, there are a lot of open questions.}
%------------------------------------------------------------------------------------

\joker{String theory}{
We mentioned the lack (10$\spadesuit$) of knowledge about the concordance between the quantum mechanics (A$\diamondsuit$) and the general relativity (K$\spadesuit$). Indeed, these two types of interaction look entirely different. The elementary particles interact by exchanging the intermediate bosons (2$\heartsuit$), while the gravitational interaction happens through the curvature of the space-time.

The string theory approach tries to apply the idea of the curved space to the elementary particle interactions. But wait for a second, the curved space creates the force of gravity, right? And if you curve it again, you will get nothing but a bit modified gravity. However, other forces act very differently from the gravitation. For example, the electromagnetic force can attract and repulse depending on the interacting particle charge, while gravity implies attraction only.

To solve this problem, physicists came with a very bright idea to introduce new hidden dimensions. While the curvature of the ordinary space-time causes gravity, the curvature of the hidden dimensions creates other forces.  Theodor Kaluza tried to add one additional dimension to the Einstein's general relativity and -- what a miracle! -- the electromagnetic force just popped out. In order to reduce that redundant dimension  Oskar Klein proposed to {\it compactify} it. Imagine an ant walking along a wire. For the ant the wire is two dimensional: it can go around and along it. But for us, who look at the scene from far, there is only one dimension along the wire, and we are not aware of the additional compactified one. Same way, the extra dimension in the Kaluza-Klein theory is compactified, creating tiny loopy strings on the microscopic scale of the space-time fabric.

Later physicists discovered the existence of two more fundamental forces, the weak (4$\heartsuit$) and strong ones (5$\heartsuit$). To add them to the string theory, they had to add even more dimensions. And that's were the mathematical nightmare starts. Instead of simple loops, space start being full of 21-dimensional superstrings. The vibrations of these complex objects create the elementary particles. Different vibration frequency corresponds to different particles, so according to this theory, the whole world is nothing but the vibrations of tiny superstring orchestra. That is a true spirit for a theory of everything: reducing the whole variety of Nature to the very few fundamental notions, which are the same for the pages of this book, for the black holes and the entire Universe.
}{JokerBlack.png}%------------------------------------------------------------------------------------

\joker{Loop quantum gravity}{
There is another possibility to build a theory of everything, competing with the string theory. While the last one tries to apply the curvature to the particle physics, the quantum gravity approach goes the opposite direction of using the quantum ideas to the gravity. Simplistically, you postulate the existence of a massless boson -- graviton, so any pair of massive particles can interact through it just like by other interactions: weak, strong and electromagnetic.

But the reality is much more complicated than that. To build the complete theory of quantum gravity, you need to quantise not just the gravitational interaction, but space and time too! The quantisation length is equal to the Planck length and the quantisation time is equal to the Planck time, about $10^{-35}$m and $10^{-44}$s correspondingly. So the space in this theory becomes like pixels on your computer screen: one pixel can be black or white, the next one can be red or blue, and there is nothing in between of them. And similarly, the time, which turns to be not continuous. There are brief stops in time, we are jumping from one moment to another, and there is nothing in between. These bits of space-time are linked to each other, and these links are called loops.

If the space-time becomes quantised, you can apply the formalism of wave-functions to it (3$\diamondsuit$). So instead of moving in a well-defined space-time, you start moving in a fuzzy quantum space-time, which also gets curved by the presence of particles. At this moment, the mathematics of this theory becomes so fierce that it is still far from completeness. However, scientists hope to solve it one day.

Again, I would point you to the Fermilab Youtube channel, where you will find several lectures exactly on the subjects of the string theory and loop quantum gravity. And to complete the impression that this science is crazy difficult look for the ``Bohemian Gravity'' video of the A Capella Science.
}{JokerRed.png}



\end{document}