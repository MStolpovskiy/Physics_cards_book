\mypart{Quantum Mechanics -- Diamonds}{Quantum Mechanics \\Diamonds}{diamond.png}{Quantum mechanics is the part of physics that works with the very tiny particles. In everyday life, we used to see material objects, for example, this book. And we intuitively know that if you grind it down, you'll have a handful of dust, where every little particle consists of the same paper as the entire book. Here is where our intuition lets us down. The tiny constituents of the usual matter behave utterly different from ordinary things. To see these effects you need to make really fine dust, about a million times smaller than the smallest dust particle, but what you'll be able to see will blow your mind!}

%------------------------------------------------------------------------------------

\card{Kr.png}{diamond.png}{Max Planck -- the black body spectrum}{
The black body is a physical term meaning a body that accumulates all the light, emitted towards it. Typically such a body should also emit light. The dependence of the intensity of this light on its wavelength (wavelength is the distance between two subsequent peaks of the wave) could be theoretically predicted. Such dependence is called the spectrum. And according to the calculations of XIX century physicists, this theoretically predicted spectrum was dramatically different from the experimentally observed one.

Although the problem sounds quite abstruse, it is very practical. Any hot body emits light. For example, a red-hot steel billet on a factory glows in the black body spectrum. Your body glows too, though you don't see it. The reason why you don't see it is that the spectrum of the black body falls steeply on short wavelengths, and the visible light has very short waves. However, the theoretical model predicts the opposite: the intensity should increase to infinity on the short wavelengths. So every object around you must be glowing very bright, according to the theory (this problem was called \textit{the ultra-violet catastrophe}).

And what was this theory? It is the theory, developed 100 years before by Thomas Young, that tells that the light is a wave. Right. That's what we told -- the light, the wavelength... And then came Max Planck (1858 - 1947) who in 1900 solved the long-standing mystery with the black body spectrum. He postulated that the light waves are \textit{quantum}, which means they are particles. Look, here comes the only formula in our book:
$$ E = n h \nu , $$
where $E$ is the energy of light, $h$ is just a constant, $n$ is the number of quanta and $\nu$ (it is a greek letter, called \textit{nu}) is the frequency of light, which is directly related to the wavelength through the speed of the wave. The constant $h$ was later called the Planck constant. It is one of the fundamental constants of Nature. So here is the frequency $\nu$, which is a wave characteristic. But there is also $n$ for the number of... what? Waves? But you can't count waves! A wave is a continuous object! You can count only particles!

So this equation implies that the light is at the same time a wave and a particle. This self-contradictory Planck's equation and his idea of quantization of light laid the foundation of the totally new quantum physics. The odd combination of wave and particle natures, introduced in quantum mechanics, is called \textit{wave-particle duality}.
}{{Cards.001}.png}

%------------------------------------------------------------------------------------

\card{2r.png}{diamond.png}{The photoelectric effect}{
At the end of the XIX century, many physicists thought that physics is over. There will be no more big discoveries. The overall theory is already developed. Sometimes professors in the Universities even told students to do not choose physics, because there’s nothing left to do. There were only a few unsolved problems. One of them we already discussed, it’s the black body spectrum (K$\diamondsuit$). And it brought a surprisingly new concept to the field -- the quantization. Nevertheless, explanation of the black body spectrum was somehow indirect evidence for the quantization.

Another unexplained phenomenon was the photoelectric effect. When you shine on a metallic surface, you can detect electrons, freed from the metal by the action of light. Now, remember that at that time, the light was thought to have the wave nature. If it is so, then you should expect that the more intense light will create more energetic electrons, right? That would be logical, because more intense light means more energy deposited on the surface, hence the electrons having more energy.

The photoelectric effect was studied carefully and it has been found that it acts just in the opposite way: the energy of the freed electrons doesn’t depend on the intensity of the light, but it does depend on its color.

It was Albert Einstein (K$\spadesuit$) who in 1905 proposed to apply the Planck’s equation (K$\diamondsuit$) for the photoelectric effect. And it fitted to this problem perfectly: the predicted by the quantum theory behaviour of the freed electrons was exactly the same, as the observed in the experiment. Thus Einstein has given the solid proof to the revolution in physics, started by Planck.
Let’s see what is the photoelectric effect, according to Einstein. It is the phenomenon of kicking out the electrons from the metal when it is exposed to light. The particles of light -- photons -- act here like the little pellets, which pass their momentum to the electrons of the material. Thus the resulting speed of the freed electrons depends on the energy of each photon and it doesn’t depend on the number of the photons. Note also, that the photon energy, in its turn, depends on its frequency. Thus again, we see these strange quantum particles with a wavy characteristic - frequency, which we first encountered when discussed the Planck’s equation.

It happened that Einstein was way more confident in the idea of quantization than its actual farther Planck. While Planck first tried to avoid quantization, thinking that it is just a kind of a mathematical trick that hides some deeper physics behind, Einstein immediately gripped this theory and promoted it as much as he could. He recognized that the mixed wave-particle nature of microscopic particles is the accurate description of the World.

}{{Cards.002}.png}

%------------------------------------------------------------------------------------

\card{Jr.png}{diamond.png}{Werner Heisenberg --\\ the uncertainty principle}{
Remember, when we introduced the quantization (K$\diamondsuit$) we told that the waves could not be counted. You might be asking yourself -- why is it? The waves breaking on the shore are easy to count. One wave after another, it's two, and so on.

You are partially right in your suspect. Let's clarify: the wave in physics is considered in an idealistic approach. When a physicist talks about a wave, he thinks about an infinite sequence of equal bumps and valleys. In everyday life, we call each bump a wave. But in physics, it is different. Can we localize this physical wave? Yes and no. No, because the wave is infinite, hence it has no specific location. But if we start to add one wave to another, we can make such a combination of waves with different wavelengths that the infinite wave will transform to a single well-localized bump.

Now let's remember that we speak about the waves of quantum objects. Let's take a photon with the given energy. According to Planck's equation, when we define the energy of a photon, we set its wavelength. It is a single, well-defined wavelength, hence the single, well-defined energy. So far, everything is fine. We are very well accustomed to well-defined things.

But let me ask you, where is this photon? As we said before, it cannot have any specific location, because it is a wave. In quantum physics, we say that we are completely uncertain about the position of the photon.

What would happen if we start summing up different wavelengths to obtain a single bump? If we succeed, then the position of the photon will be well defined. But what would happen with its energy? Since we added up multiple wavelengths -- and to obtain a clear bump we need an infinite number of different wavelengths -- we cannot say which wavelength has this particular photon. Hence we can not say anything about its energy. We found ourselves in the opposite position: we defined the location of the photon, but we are completely uncertain about its energy.

It is interesting to specify the role of the observer. When you measure the position of the particle, you do it by interacting with it. By measuring the particle position, you change the particle, so its energy becomes uncertain because of the act of measurement. The same thing happens with the energy measurement. It is the fundamental principle of quantum experiments that you can never have complete knowledge of the properties of your system.

It was Werner Heisenberg who first described all this mess in his famous uncertainty principle. It is one of the essential things in quantum mechanics.
}{{Cards.003}.png}

%------------------------------------------------------------------------------------

\card{3r.png}{diamond.png}{The wave function}{
How can we describe the behaviour of quantum particles? To do it, we use the so-called wave function formalism.

In the framework of classical mechanics, where everything is certain, we can identify the position of each particle in each moment. In quantum mechanics it is impossible. However, we can determine the probability for a particle to be in a specific position. It means that we cannot know the location of the particle until we measure it. The probability of finding a particle in a given position is defined by the wave function.

Here we implicitly introduced the very important concept of measurement. In quantum mechanics, every property of the particle remains uncertain until measured. If we try to measure two quantities simultaneously, we are limited in the precision by the uncertainty principle. Each measurement changes the particle state to make the measured property sure and other properties -- uncertain. So far, there is no mystery: we measure the state of the particle by interacting with it, so it is not wholly surprising that we change the particle by the act of measurement.

However, it has some peculiar consequences. Let's imagine the experiment: the quantum particles fly one by one towards the screen with two slits. Since the position of each particle is uncertain, the particle looks like the wave, which passes through both slits simultaneously. The waves create the interference pattern on the sensitive plate beyond the screen (interference is the effect of summation of the waves from the two slits, which creates the striped pattern on the plate).

But what would happen if we place a detector in front of the slits to identify through which of the slits the particle went through? Then the position of the particle becomes certain through the act of measurement, hence the particle looks less like the wave, and you won't observe the interference pattern on the sensitive plate.

But how the particle could know that we will measure it and it should not be wavy and try to go through both slits at the same time? The answer is that the position of the particle is described in a completely non-classical way. One cannot say that the particle is passing through one slit or another. Neither one can say that it's going through two slits simultaneously. The particle behaviour is quantum, meaning that it is described through the wave function.

If you read a bit about the wave function and especially about the double-slit experiment, you'll find a curious notion of the wave function collapse. This is exactly what we were talking about: the phenomenon of the sudden appearance of certainty in the behaviour of the quantum particle.
}{{Cards.004}.png}

%------------------------------------------------------------------------------------

\card{4r.png}{diamond.png}{Bohr's atom}{
The quantum mechanics, which describes each particle as a wave and a particle in the same time, is very successful in explaining some phenomena with the free particles, for example, the black body spectrum (K$\diamondsuit$) or the photoelectric effect (2$\diamondsuit$). But can it predict the behaviour of such a complex system as an atom? Yes, it can. Let's see how it works on the simple example of the hydrogen atom (a bound system of a heavy positively charged proton and a light negatively charged electron).

Let's conduct a little experiment. Take a guitar (if you don't have a guitar,  use your imagination) and pick an open string. You will hear the sound with the frequency $\nu$. Now put the finger on one of the strings just above the twelfth fret, don't push it down and pick the string again, releasing the finger at the same moment. This technique is called the flageolet. If you did everything right, you should hear the sound one octave higher -- it's frequency is $2\nu$.

What happened? You made the same string to sound at two distinguish frequencies, which are defined by its tension and length. High frequency means faster trembling of the string. Hence, it gets more energy, when you pick it with a flageolet than if you pick the open string. It means that the string can have two distinct energy states.

You can make flageolets also on the seventh and fifth frets. They will give the frequencies $3\nu$ and $4\nu$ correspondingly. But the higher flageolet you take, the harder it is. In other words, the high energy states of the string are not stable.

The string has these discrete energy states, because of the wave nature of its trembling. The atom is just the same: since the electron has the wave nature, it can take only some discrete number of energy states in the atom. It is traditionally illustrated as if the electron as a particle rotates on different orbits around the atom nucleus (I adopted this tradition for the picture on the card). Still, in reality, the electron in the atom is a wave, which can take some number of discrete energy states. When the electron passes from the high energy state to the low one, it releases the extra energy in the form of a photon. The high energy states of the atom are not stable, so the atom tends to return to the low energy state.

Historically, it was first understood that the atom has a dense positive nucleus and electrons are flying around it. And such a system seemed quite tricky: according to all the known laws of physics the electrons had to fall very fast on the nuclei. Only the creation of quantum theory helped to unravel this scientific mystery.}{{Cards.005}.png}

%------------------------------------------------------------------------------------

\card{Qr.png}{diamond.png}{Marie Curie -- radioactivity}{
Marie Curie is one of the most famous scientists of all the times, and certainly the most prominent scientific woman. She was the first-ever person who received two Nobel Prizes, and several other ``first-ever" too. Curie worked on the topic of radioactivity. Let's see what made her work so important to us.

The term radioactivity describes the radiations of different nature which appear when the atom nucleus decays. After that, the atom often becomes a different chemical element. The radioactivity has many practical applications. It is used to treat cancer, to date archaeological findings, to produce energy in the nuclear reactor and many other fields. Now let's consider how the quantum nature of microscopic particles rules the radioactivity.

Remember, we told that the wave function (3$\diamondsuit$) doesn't corner the exact state of the particle, but describes only the probability for the particle to be in some state. This probability is linked to the fundamental uncertainty principle (J$\diamondsuit$), which means that quantum mechanics makes the realisation of a truly random process.

Random process? What's that? To explain, let's see what is not random. When you throw a dice, the resulting score is not accidental: it is well-defined by the way you've thrown the dice. Every coincidence in our life is not random. There is almost no truly random processes in our life -- everything is well-defined by the previous events. Many processes are indeed practically impossible to predict. But even though they are not random: if you track down every little motion that constitutes the complicated process, you'll see that everything in the process was quite certain.

However, the quantum mechanical processes are genuinely random. It means that there is no reasoning for the given particle to get this or that final state. The particle ``chooses'' its state at random.

The radioactivity is a quantum process. That is the nuclei behaviour described by the nuclear wave function. Hence the decay of the nuclei occurs at random.

The deep understanding of the physics of radioactivity became one of the cornerstones of the quantum mechanics and turned it to the new level: remember that the quantum mechanics started with free massless photons (K\&2$\diamondsuit$). Then we considered the behaviour of electrons in an atom (4$\diamondsuit$). Now it turned to the much more complex system, the atom nuclei.
}{{Cards.006}.png}

%------------------------------------------------------------------------------------

\card{Ar.png}{diamond.png}{Schr\"odinger's cat}{
Now we have introduced all the ingredients to sort out the famous Schr\"odinger's cat paradox. Let's recall: the behaviour of the quantum objects is defined by the wave function (3$\diamondsuit$), which makes it behave randomly (Q$\diamondsuit$): the result of the quantum process appears random after the measurement. Otherwise, we have to assume that the quantum object takes all the possible states at once. It is probably fine for you to think that all this mess appears only on the microscopic scales and doesn't touch you directly. But actually, things are not so clear.

Erwin Schr\"odinger, whose the main contribution to the physics was the writing of the equation, which describes the transformation of the wave function in time, presented the following paradox. Let's imagine a box, which contains an ampule with poison. The ampule can be broken by the mechanism, which is triggered by the radioactive decay of an atom. Let's also imagine that the probability for the atom to decay is 50\%: after the finish of the experiment, there is 50\% chance to find the atom decayed and hence the poison ampule was broken, and 50\% chance to see everything in its place. 

Now let's put a cat in this box. What is the state of the radioactive atom, while the box is closed? According to quantum mechanics, it takes both possible states at once: decayed and not decayed one. But it means that the ampule with poison is both broken and unbroken too! Hence the cat is dead and alive at the same time. And the cat, which is quantum here, takes the specific state -- either dead or alive -- only after the measurement. That is, only when you open the box.

The Schr\"odinger's paradox is not just some theoretical construction, which you can neglect. It reveals the deep problem: we are living in the world, where each tiny particle is quantum. And at the same time we, people, seem to be not quantum at all. It's not quantum physics which is strange. Instead, it is weird that we, who are built of quantum particles, are not quantum too! The paradox with the cat is all about it. 

To solve this paradox, we introduce different ``interpretations'' of quantum mechanics. If you believe that the wave function of radioactive atom collapses when its radiation interacts with the detector, then you are an adept of the Copenhagen interpretation. If by contrary you think that cat shares its wave function with the atom and the cat is indeed dead-and-alive until you open the box, you believe in many world interpretation. Unfortunately, I ran out of place on this page. So try to study some of the materials suggested at the end of the diamond suit.
}{{Cards.007}.png}

%------------------------------------------------------------------------------------

\card{5r.png}{diamond.png}{Isotope stability}{
We have introduced the basics of quantum mechanics. Now we are stepping farther and start exploring the current research activities in this area. And the first thing to learn is the isotope stability. This subject is directly related to the work of Marie Curie (Q$\diamondsuit$) because the instability of the isotopes implies the radioactive decay of them. Let's see how it works precisely.

We've seen how the atom of hydrogen is constructed (4$\diamondsuit$). But hydrogen, the lightest chemical element, is the simplest example of an atom. The nucleus in its centre consists of just single proton. Hence it can hold only one electron: the negatively charged electron attracts to the positively charged nucleus. To obtain a different chemical element, we need to put more electrons in it. It is possible only by increasing the charge of the nucleus. But you can't do it by just adding more protons because the equally charged protons push each other with electric force. To bind them together we need neutrons -- particles with no electric charge, otherwise pretty similar to protons.

You may ask how it works because the negatively charged neutrons don't electrically attract protons. To fully answer this question, you should glance forward in this book and learn a bit about the exchanging interactions (2$\heartsuit$) and strong force (5$\heartsuit$). In a few words, the binding force between protons and neutrons in the nucleus is created by the exchange of $\pi$-mesons, which are created by the strong interaction. This $\pi$-meson interaction doesn't work between the neutrons. Thus we have to balance between the nuclei with too many protons and the nuclei with too many neutrons. The first ones decay, because they cannot hold the protons, which push each other away. The second ones decay too because they miss the binding $\pi$-mesons-guarded force. The different sets with various numbers of protons and neutrons are called isotopes. Some of them are well balanced, so they are stable. Some are not -- they decay and do it faster as they are farther from the stable state.

Nowadays, many laboratories around the globe work on the creation and study of the new heavy elements. It is one of the hottest topics in modern physics, which promises some significant discoveries in the upcoming years. For example, the so-called island of stability: the protons and neutrons stay on the energy levels in the nucleus, similar to the electrons in the atom (4$\diamondsuit$). Energy levels on their turn form the \textit{shells}. The nucleus with a filled shell is way more stable than other configurations. Scientists hope to find stable elements with full shell and a very high number of constituent protons and neutrons. Such elements will be separated from the main sequence of stability (depicted on the card) by a gap, so they stay on an ``island'' of stability.
}{{Cards.008}.png}

%------------------------------------------------------------------------------------

\card{6r.png}{diamond.png}{Laser}{
Another interesting topic is the laser. Did you know that the word laser is actually an acronym? It means ``light amplification by stimulated emission radiation''. Nowadays lasers are used everywhere, from laser pointers to advanced applications in surgery. The laser is the only way to produce the very intense, well-controlled light beam. The modern scientific lasers are used to turn the matter to the new extreme states and to push the border of our knowledge in physics.

When we studied the atom (4$\diamondsuit$), we told that the electrons in the atom could be in various energy states. The specific set of possible energy states depends on the charge of the nucleus. The nucleus charge, on its turn, defines the chemical elements. Hence all the atoms of some chemical element have the same energy states. We also told that when the electron passes from the high energy state to the low one, it releases the extra energy in the form of a photon, that is it emits the light. Since the energy states are the same for all the atoms of a certain chemical element, all such atoms emit light on the same well-defined frequency. In fact, it is a set of frequencies, because there are many possible transitions between the energy states in the atom.

The decay of an excited atom (in high energy state) occurs at random, as any quantum process. However, we can push the atom to decay. As we said, the high energy states are not stable. Which means an excited atom decays if you only push it a bit.

In laser, you do it by the light, emitted by the other atoms in the same system. It goes like this: one atom decays, the radiated photon pushes another atom to decay too. On the next step two atoms decay, then four, eight, and so on. The number of photons grows like a snowball. When the bunch of light releases from the laser, it consists of identical photons, flying in the same direction. This characteristic of laser light is crucially important for numerous applications.

Finally, let's mention how do we excite atoms. It is pretty simple. All the atoms undergo the thermal movement. When you heat the material, this movement becomes more intensive. At some point, the atoms move so fast, that when they hit each other, they become excited. That's it. You simply need to heat the body of the laser a bit, and it will start to shine on its own.
}{{Cards.009}.png}

%------------------------------------------------------------------------------------

\card{7r.png}{diamond.png}{Superconductivity}{
Another funny effect of quantum mechanics is the superconductivity. Did you know that every electric device suffers from the resistance of the conductive materials to the electric current? When the electrons pass by in the wires, they get bumped by the atoms and lose their energy. So when you pay your bill for the electricity, there is always a part to cover this loose.

To reduce the electric resistance, the manufacturers use different materials for the wires. Coper, for example, is cheap and has pretty low resistance. Silver and gold are known for their excellently low resistance too. But quantum mechanics proposes the ultimate solution to the problem.

If you cool down the conductive material, the thermal motion of the atoms reduces. At some temperature, lower than some critical temperature (Tc), when the atoms move very slow, an interesting effect occurs. The electrons start to form so-called Cooper pairs. The size of such a couple is larger than the typical distance between the atoms. That means that the coupled electrons begin to move like a kind of cloud through the material. It is a purely quantum phenomenon, which involves all the quantum mechanical basics discussed before. In the warm material, the fast-moving atoms easily break the Cooper pairs. That's why you need a low temperature.

The funny thing is that when the electrons start to form the Cooper pairs, the resistance drops down virtually to zero. This effect is called superconductivity. It means you can, in principle, make a superconductive wire and pass a high current through it, and you won't miss a single penny for the excessive heat. Excellent, isn't it?

Not so much, because the typical value for the critical temperature is about several Kelvin ($-270^\circ$C or $-450^\circ$F). There are works to produce a material which would hold the superconductivity to higher temperatures, but the highest critical temperature reached so far is about $-100^\circ$C ($-150^\circ$F). To cool down the wires to this temperature you have to spend a lot of electricity for the fridges, so finally you even loose. The superconductivity is used in practice, but very rarely, mostly in scientific devices.

Another exciting thing is that the superconductor expels the magnetic field from itself -- it is called the Meissner effect. This expulsion results in a force that makes a piece of a superconductor to levitate above the magnet. Many people believe that this effect will be the base of future zero friction transport. But for the moment, when we don't yet dispose of the warm superconductors, this idea is just fantastic.
}{{Cards.010}.png}

%------------------------------------------------------------------------------------

\card{8r.png}{diamond.png}{Bose-Einstein condensate}{Let's return to our favourite Planck's equation (K$\diamondsuit$). If we reduce the energy of the particle, then what is going on with its frequency $\nu$? It also reduces. Hence the wavelength increases. How much can it grow? Pretty much. For the very low energy particles, their wavelengths are so huge that they start to overlap. In practice, it looks like a single sizeable wavy particle instead of multiple small ones. This strange quantum mechanical state of matter is called Bose-Einstein condensate after the names of two theorists who predicted it (K$\spadesuit$).

Bose-Einstein condensate cannot be made of any particles, but only of bosons (as you correctly guessed they are called after Bose). And to explain what are bosons we have to introduce the notion of spin. We'll do it with another little experiment. Let's spin a spinning top and then push it slightly. You'll see that the spinning top doesn't fall after the push, as any non-rotating body would do. The particles behave quite similar: they tend to keep their orientation. The spin quantifies this particle quality. You can force the particle to change the orientation of spin if you put it into the magnetic field. Similar to the energy levels of the electron in the atom (4$\diamondsuit$), particles can have only a predefined set of orientations of spin in the magnetic field. If a particle takes half-integer values of spin orientation like $\pm 0.5$, $\pm 1.5$ etc. then it is called a fermion. If the possible values are integers like $0$, $\pm 1$, $\pm 2$ etc. then it is a boson. The fundamental difference between fermions and bosons is that the fermions cannot form the Bose-Einstein condensate. It could be illustrated as follows: imagine a glass full of marbles. Each marble takes some space and doesn't let others take the same position. The fermions behave the same. While the bosons do not obey this logic. Instead, the extremely cold bosons would look like a single marble in the glass, while you know that you have put there a dozen of them.

The Bose-Einstein condensate could be made in practice, although it is very challenging and was done only quite recently. The study of this exotic state is crucially essential for the experimental test of the fundamental laws of quantum mechanics, that's why many laboratories around the globe try to get pure condensate.

An interesting effect appears when you try to cool down the fermions. Fermions expel each other, so they cannot form the condense state as bosons. However, if fermions pair up, they get an integer spin so that they can condensate. An example of a fermion is the electron. And we already considered the situation when the electrons form pairs at low energy: it is the formation of Cooper pairs while transiting to the superconductivity state (7$\diamondsuit$).}{{Cards.011}.png}

%------------------------------------------------------------------------------------

\card{9r.png}{diamond.png}{Entanglement}{
Entanglement is one of the most studied questions in physics with several attractive practical applications. And it's one of the strangest. Even Albert Einstein (K$\spadesuit$) has called the entanglement ``spooky''. First, let's repeat what we know about the quantum particles. Their behaviour is described by the wave function (3$\diamondsuit$), which implies that the state of the particle is defined only under the measurement. Otherwise, we have to assume that the particle takes all the possible states at once. Particles also have a spin, which value obeys this quantum behaviour too. What I haven't yet told about the spin is that the total spin conserves in the processes of creation or annihilation of the particles.

What would happen if two particles are created in one process? In this case, they become entangled, which means that they share the single wave function for both. We already learned about the strange behaviour of the individual quantum particles. When there are two of them, it becomes even more bizarre. It turns out that the measurement of one particle in some mysterious way immediately changes the properties of another particle. Let's consider the most often example with the measurement of the spin.

As soon as we haven't measured the particles, their spins remain undefined. But at the moment of the measurement, something interesting happens. The total spin conserves, so if, for example, the spin before the creation of our pair of particles was zero and the spin of the measured particle turns out to be oriented up, then the spin of another particle must be down. And if you measure it, it turns out to be down! But how? The particles were in a completely uncertain state before the measurement, and under the measurement of the first one, the second becomes immediately certain. Immediately -- means that the mysterious information about the orientation of the spin of the first particle reaches the second one with speed faster than the speed of light, which is prohibited by the relativity theory of Einstein.

But maybe the particles keep their spin from the moment of their creation? No, this is excluded by the quantum nature of spin. Imagine having a particle with the spin to the right. And we measure it in the vertical orientation. Then it will have a 50/50 chance to have the spin up or down after the measurement. So the act of measurement changes the spin. But if now you measure the spin of the entangled particle, again in the vertical orientation, you will necessarily find it down. The second particle receives information about the measurement result of the first one with speed faster than the speed of light. That's indeed spooky!
}{{Cards.012}.png}

%------------------------------------------------------------------------------------

\card{10r.png}{diamond.png}{Fusion reaction}{This topic of quantum mechanics has probably the most attractive possible application. We already learned that the protons and neutrons could form stable and non-stable structures -- nuclei (5$\diamondsuit$). It means that these systems possess binding energy, like tiny strained springs that hold the protons and neutrons together.

It turns out that heavy nuclei if being broken, radiate a bit of energy. Moreover, in some cases, the broken atom emits a few neutrons. When those neutrons reach other atoms, they break them too. Thus the reaction becomes chained -- more and more atoms become involved in the reaction, and you get a nuclear explosion. In a nuclear reactor, some neutrons are absorbed, so the intensity of the reaction remains stable. Nuclear fuel is the most energy capacious among other sources like coil or gas.

But there is a possibility to get even more energy from light atoms. Unlike the heavy ones, which break apart releasing energy, the light atoms tend to merge, with even larger energy deposit. This is called the fusion reaction. And this process produces much-much more energy, than fission of heavy nuclei. Fusion reaction undergoes in the interior of the Sun (2$\clubsuit$) and other stars (A$\clubsuit$). As a possible source of energy fusion reaction is way more advantageous than the fission reaction. But why don't we use it?

The problem is that it is tough to control. The energy deposit is so large that it blows away the light atoms which you tried to use as the fuel for the reaction. As a result, the fusion doesn't become "chainy" -- the energy released in one reaction doesn't help other atoms to fuse because the last ones are already too far from each other. However, if you try to increase the density of your nuclear fuel, the reaction turns to the explosion, which is called thermonuclear. It is the most powerful and destructive explosion mechanism invented by people.

Different techniques are tried to keep the fusion reaction inside a reactor. But so far no one succeeded in this task. While in the stars (4$\clubsuit$) the hot plasma is held tight by the gravitational force, we don't dispose of such a vast mass to afford it. Instead, we use electromagnetic force to keep the plasma. In the same time, you have to feed somehow the plasma with the fuel to maintain the fusion reaction. This task is extremely challenging and involves dozens of very complex components. But the return would be very attractive too. Many scientists believe that this will be the ultimate solution for all our energy problems. So far, all the existing fusion reactors consume more energy to maintain the plasma than they radiate.
}{{Cards.013}.png}

%--------------------------------------------------------------------------------

\thispagestyle{fancy}
\fancyhf{}
\renewcommand{\headrulewidth}{0pt}
\lhead{\thepage \hskip14pt Physics Is My Favorite Game}
\fancyfoot{}
{\huge{\textbf{Suggested materials}}}
\vskip12pt

For this book, I've chosen the format of one page - one card. It limits the number of explanations I can give, although I tried to do not lose in terms of quality. There are different levels of understanding. One is when you grasp only the basic principles and terms. Another level is when you are fluent in handling the formulas (and you are the head of the theoretical division in an international institute). There are plenty of intermediate levels, so you can choose yourself the deepness of understanding that suits to you. I tried to compose the list of suggested readings and videos to help you to go a bit deeper.

Subscribe to Veritasium channel on Youtube. There are plenty of interesting videos, in particular on quantum mechanics. Their explanation of the entanglement is just brilliant! The video ``Is This What Quantum Mechanics Looks Like?" is also quite interesting.

Read the news. For example, subscribe to ITER on Twitter. ITER is the future largest fusion reactor, which is in the construction phase now. It is impossible to give a fast introduction to all the innovations they use. But when you encounter these notions from time to time, you learn it very efficiently. Read also the phys.org website. The base I've given you should suffice for the understanding of most of the articles.

If you like reading books, and I think you do, because you are holding one in your hands right now, then I would advise you to read ``How to Teach Physics to Your Dog'', by Chad Orzel. The book has a lot of beautiful analogies, and introduces in a simple manner the complicated notions as quantum teleportation, for example. And this book is quite recent, so all the modern researches are there.

And if you are looking for something more serious, then there is nothing better than Feynman's lectures: ``The Feynman Lectures on Physics, Volume 3''. This book reaches for a higher level -- roughly speaking it is a half of the university course of quantum mechanics (and twice the knowledge that remains in the student's head after s/he passes the exam). If you want a deep solid understanding, this is the best choice for you.

\newpage