\mypart{Astrophysics -- Clubs}{Astrophysics \\Clubs}{club.png}{This section will be quite mixed. Astrophysics is pretty much about everything: interstellar gas and black holes, star clusters and galaxies, gravitational waves and cosmic particles.}

%------------------------------------------------------------------------------------

\card{Kb.png}{club.png}{Edwin Hubble -- galaxies}{
In XVIII-XIX centuries scientists believed that Milky Way (our galaxy) was itself the Universe. The question about the actual size of the Universe was especially keenly posed at the beginning of the twentieth century when scientists started to think about the nature of numerous nebulae that could be seen in telescopes. In 1920 a discussion between two authoritative American astronomers Harlow Shapley and Heber Curtis arose. The dispute was about the nature of nebulae. Shapley affirmed that all the nebulae were nothing but gas formations situated in our galaxy. Meanwhile, Curtis contended that many nebulae were individual galaxies, containing billions of stars and are located far away of our galaxy. According to Curtis, our world is by many orders of magnitude larger than any of galaxies. Both scientists gave observational and theoretical arguments for their concepts, but couldn't conclude.

In 1917 the Mount Wilson Observatory was equipped with the largest telescope at the time with the primary mirror diameter 2.5m. Edwin Hubble (1889 – 1953) started to work there. Using the photographic method in 1923 – 1924 he resolved, for the first time, three spiral nebulae to individual stars. Among the stars of the Andromeda nebula, he found some variable stars – cepheids, which allowed to measure the cosmic distances. According to the estimation, the distance to the Andromeda nebula is much higher than the size of our galaxy. Thereby Hubble proved that the Andromeda nebula is situated outside the Milky Way and constitutes a large star system, as big as our galaxy. Thus, with the inauguration of a new telescope, the size of the Universe was increased severely. The Curtis's point of view won.

Hubble studied multiple galaxies and composed a classification of their morphology to make the first step to an understanding of the galaxy formation and evolution. It is so-called ``Hubble's tuning-fork''. According to this scheme, at the early stages, the galaxies are elliptical, without any apparent features. Then they start to develop the spiral arms. For the spiral galaxies, Hubble differentiated the types with or without the central bar. Nowadays, we understand that the spiral structure is inherent mostly to the young galaxies, while the ellipticals dd are in the main old ones. And the bar can appear and disappear along with the history of a galaxy. However, even today, we find useful the Hubble's classification.

Hubble also made a significant discovery of the recession of all the galaxies from each other. We will touch upon this later (J$\spadesuit$). Although we omit it on this page, it cannot be neglected as it is one of the most important breakthroughs in XX century science.
}{{Cards.027}.png}

%------------------------------------------------------------------------------------

\card{2b.png}{club.png}{Sun}{
The Sun, the ancient god for humans, the subject of extensive research for scientists. An ordinary star in the Galaxy (A$\clubsuit$), and the most important source of information about stellar physics (4$\clubsuit$).

All the cycles on Earth are driven by the Sun. For example, the biological material from which we are made is always in circulation: from us to the ground, then to the plants and back to us with food. And the source of energy for this cycle is the Sun.

The Sun was created about 4.6 billion years ago. Before that, there was a cloud of hydrogen and dust. Eventually, due to fluctuations in the cloud density, it started to rotate and shrink. As the density in the centre of the cloud grown, the temperature increased. At some point, the temperature was high enough to light up the fusion reaction (10$\diamondsuit$). This moment could be considered as the birth of the Sun.

The Sun is enormously large -- its diameter is 109 times bigger than that of Earth. The outer shells of the Sun are not so hot, ``just'' about 6000 $^\circ$C (about 10000 $^\circ$F). But its core, where the fusion reactions take place, has temperature more than ten million $^\circ$C and it is more than ten times denser than lead. On Earth, we cannot handle the fusion reaction. But in the Sun's core, the high pressure of this powerful reaction is balanced by the enormous gravitational field. 

The hot plasma from the centre of the Sun continually goes up to the surface, where it cools down and falls back. This plasma motion makes complex structure of layers. Since the plasma is charged, its movement also creates the magnetic field of the Sun.

The complex structure of the Sun is studied with the same seismic method we use to analyse the interior of Earth. Every disturbance in the Sun's core results in the vibrations of the surface, like the earthquakes. Observing these vibrations, we can learn what caused it.

The Sun does not only give us warm and light. Unfortunately, its bright power is unavoidably linked to the radiation and the magnetic storms. The magnetic field of Earth blocks most of the Sun's radiation, but the rest is still pretty harmful to us, especially for our electronics and for the people sensitive to it. Such phenomena as Sun's black spots and solar flares are the main features due which we can predict the magnetic storms and fluctuations in the flux of the solar radiation. By the way, the latter one is considered as one of the main challenges during the human exploration of the Solar System (3$\clubsuit$).
}{{Cards.028}.png}

%--------------------------------------------------------------------------------

\card{3b.png}{club.png}{Solar System}{
To explain in details this card I would have to write another book. All I can do on one page is to list the main features of our home planetary system.

Since the dawn of time, humans knew the six planets of the Solar, the ones closest to the Sun: Mercury, Venus, Earth, Mars, Jupiter and Saturn. Uranus and Neptune can not be seen with a naked eye. So they were discovered much later -- in XVIII and XIX centuries correspondingly. The terrestrial planets (the four closest to the Sun) are composed of rocks and metals. Giant planets: Jupiter, Saturn (two most massive planets of our system), Uranus and Neptune consist of light materials like gas or ice. This discrimination is caused by the way the Solar System was formed: the light elements and ice could exist in solid form in the outer regions, where they formed the giant planets. The remaining material forms the asteroids and comets. Most of them are concentrated in the asteroid belt between Martian and Jovian orbits and in Kuiper belt beyond Neptunian orbit. The largest body of the Kuiper belt, Pluto, has recently lost the title of a planet. However, studies show that there is probably yet another planet in the very outer regions of the Solar System. Yet undiscovered Planet Nine would explain the similarities in the orbital motion of the trans-Neptunian objects. 

The borders of the Solar System are still not well studied. The Oort cloud, comprising billions of comets, spans from about 1000 to 100,000 AU (AU -- astronomical unit -- the distance between Earth and Sun, equal to about 150 million kilometres or 90 million miles). The most distant spacecraft Voyager 1 for 40 years of its mission has passed only about 0.1\% of the Solar System's radius.

Today we reached so high level of development that we are even attempting to spread our civilization to the other planets. But here we face significant challenges. Other planets are either too cold (Mars), or too hot for us (Venus). The cosmic radiation is also a very striking factor. However, in 2030s, we are aiming to land people on Mars. And this is only the start of the future massive exploitation of the planets and moons of the Solar System.

The mentioned asteroids and comets constitute a severe menace to life on Earth. Although small meteors burn down in the atmosphere, the large ones can cause serious damage: explosive impact, tsunami, earthquakes. A large deposit of dust in the atmosphere will cause the reduction of Sun warmth reaching the ground. Sixty-six million years ago such impact resulted in the extinction of the dinosaurs. But today we hope to dispose of enough means to avoid collisions. Much effort is made to monitor the dangerous asteroids around the Earth. If any large body were spotted on a hazardous orbit, it would suffice to deflect it slightly and thus save our planet.
}{{Cards.029}.png}

%--------------------------------------------------------------------------------

\card{4b.png}{club.png}{Stellar evolution}{
The life cycle of any star begins in the disperse cloud of gas and dust, which are often found in space. This cloud has its gravitational field, so it slowly starts to collapse on its centre of mass. This is an accelerating process: the denser becomes the core, the faster the matter falls on it. This core is called a protostar. Once the density in the core raises high enough, the nuclear reaction begins, and this moment could be considered as the birth of the star. The powerful hot plasma in the centre of the newborn star produces the necessary pressure to balance the gravitational attraction and prevents the matter from further collapse. Once the star lights up, its radiation swipes away the rest of the dust and gas and the star becomes visible for the outer world.

But not all the stars are so lucky. If there is not enough material, the star may never appear. Still the overall process goes the same way, just the resulting dense object is not as large and hot. Such an object is called a brown dwarf. These substellar objects have enough mass to maintain the nuclear reaction, but this reaction is not as powerful as the reaction in average stars: while in Sun, for example, the hydrogen transforms into helium, brown dwarfs use the fusion of heavier nuclei, which does not produce as much energy. Brown dwarfs are super stable -- they can glow for many billions of years.

Average stars become sooner or later the giant stars. This transformation would take place faster if the star was massive from the beginning. It happens because the hydrogen in the centre burns down to helium (and later to the heavier elements), and the fusion reaction starts to take place in the shell around the core, where the pressure is lower, and there are fewer factors to limit the size of the star. For the Sun (2$\clubsuit$) this phase will come on in about 5 billion years. Then it will probably engulf the Earth (if not, the life on this planet will be unbearable anyway).

When a massive star spends all the nuclear fuel, it explodes in a destructive supernova event (6$\clubsuit$), forming at the end either a black hole (J$\clubsuit$) or a neutron star (Q$\clubsuit$). But if the star was not massive enough, it would end up dumping off the outer shells while leaving the core heated up to 10$^7$ $^\circ$C. The thrown out material, lit by the X-rays from the remaining core, starts to shine, forming a colourful planetary nebula (this term is just historical and has nothing to do with planets). The core, in this case, is called white dwarf -- it is typically as massive as the Sun while being as big as the Earth. The planetary nebulae shine for several tens of thousand years. Then the gas becomes too dispersed, and the remaining faint white dwarf stays cooling down in complete loneliness. Our Sun, after passing the giant phase, will end its life on this branch of the stellar evolution tree.
}{{Cards.030}.png}

%--------------------------------------------------------------------------------

\card{Ab.png}{club.png}{Main sequence}{
The significant role in understanding the stellar evolution (4$\clubsuit$) played the so-called Hertzsprung–Russell diagrams. It is a kind of chart where the stars are plotted depending on their colour and luminosity. Let's first see what does it mean.

Remember, we told about the wavelength of light while sorting out the quantum mechanics (K$\diamondsuit$). The wavelength is directly related to the colour: for example, the photons that seem to be red to our eyes have the wavelength about 700 nm, while the violet photons have shorter wavelength -- about 400 nm (of course there is a whole spectrum of photon wavelengths). The precise wavelength of a stellar object depends on its temperature, as it was discovered by Planck (K$\diamondsuit$). Thus the blue stars on the sky are hotter than the reddish ones.

The luminosity means the brightness of a star. It describes how many photons it emits. The only way to increase the photon flux from the star is actually to increase the star itself. Thus it would be accurate to say that the bright stars are generally large. Don't forget that the distance also plays its role. But at least it is true that, for example, Betelgeuse, one of the brightest stars on the sky, is almost 1000 times larger than the Sun. While Alcor, one of the faintest stars, seen by the naked eye, is only slightly bigger than the Sun -- and it is tough to spot it, even though it is ten times closer to us than Betelgeuse.

It turns out that if you plot all the stars on the sky on a graph like this, you'll find a sizeable thick line that spans on the diagonal from the bright blue stars to faint red ones. That is for the majority of the stars, the size directly depends on the temperature. This line of stars is called the main sequence. Right after the birth from the cloud of gas and dust, stars come to the main sequence. Depending on their mass, they appear either on the left upper corner or on the right down one. Later they can move to the state of the red giant and then to the very small but hot star like a white dwarf or a neutron star. But the large part of their stellar career they spend on the main sequence.

The Sun (2$\clubsuit$) belongs to the main sequence too, and it is more or less at the centre of Hertzsprung–Russell diagram.
}{{Cards.031}.png}

%--------------------------------------------------------------------------------

\card{6b.png}{club.png}{Supernova}{
Now we skip the cosmic ray card (5$\clubsuit$) to tell first about one of the primary sources of these rays -- supernova, a powerful and destructive explosion of a star.  We already mentioned that the massive stars at the end of their evolution finish up with a supernova. Let's consider how it happens in a little bit more details.

There are two main scenarios of a supernova explosion. The first mechanism is realised in systems of two stars rotating close to each other. It is common that one of these stars after some evolution finally becomes a white dwarf (4$\clubsuit$). This type of stars has a mass limit, called the Chandrasekhar limit. The white dwarf sucks out the material from its companion star. Ones it passes the Chandrasekhar limit it explodes as a supernova.

This is the so-called type one explosions. Since they all happen in the same circumstances -- Chandrasekhar limit is a very well defined value about 1.4 solar masses -- the brightness of these supernovas is an excellent indicator of distance: the fainter the star is, the farther it is from us. So it is possible to use these supernovas as a cosmological ruler. The problem of distance measurement is one of the most important ones in the cosmology (A$\spadesuit$).

Another scenario is possible for the massive stars, heavier than three masses of the Sun. By the end of its life, the star runs out of the nuclear fuel that was keeping star from shrinking under the force of gravity. So the star collapses under its weight, and its core turns to the extremely dense state of a neutron star (Q$\clubsuit$) of a black hole (J$\clubsuit$). At the moment, it happens the core releases its gravitational potential energy, which causes the supernova explosion.

The turbulent magnetic fields in the supersonic shock wave from the supernova can accelerate elementary particles to crazy energies, million times higher than in the largest particle accelerators on Earth (J$\heartsuit$). These particles are called the cosmic rays, and they are the subject of the next card (5 $\heartsuit$).

Supernova explosions are quite rare events: in our galaxy, they happen every about 300 years. The last close supernova was in 1987, but it was no in the Milky Way but in its dwarf satellite galaxy -- the Large Magellanic Cloud, some 168 thousand light-years away. Even on such a vast distance, it was visible to the naked eye. A supernova in 1604, described by many European, Chinese and Arabic scientists, was only 20000 light-years away from us. And it was visible even during the day for over three weeks.

}{{Cards.032}.png}


%--------------------------------------------------------------------------------

\card{5b.png}{club.png}{Cosmic rays}{
Supernovae (6$\clubsuit$) and other astrophysical sources can create fluxes of charged particles like protons, electrons and nuclei of atoms. These particles can have energy in a large range up to hundreds of exaelectronvolts, which is a hundred million times more than the highest elementary particle energy reached in accelerators. Such particles are called cosmic rays.

The name ``cosmic rays'' comes from the times when scientists started learning about some mysterious radiation but didn't yet know the nature of this phenomenon. The correct name would be ``cosmic particles''.

The number of cosmic rays drops very fast with the growth of energy. The most abundant cosmic rays, as we believe now,  come from the supernova explosions in the Milky Way. But, wait a second. Just a page ago we said, that the supernovae appear in our galaxy every several hundred years! Is it plausible that we have a continuous flux of cosmic rays from the sources which are so rare? In fact, instead of travelling straight through space, cosmic rays are constrained within the galaxy because of the galactic magnetic fields (7$\clubsuit$). On average a particle can spiral for millions of years, turning in the galaxy arms. During this time, the nuclei of the heavy elements, like carbon or oxygen,  decompose in the interactions with the interstellar medium, forming such light atoms like beryllium and boron. 100\% of these elements come from the decomposition of the heavier cosmic rays.

Quite recently scientists started to realise that the cosmic rays have a substantial impact on the star and even galaxy formation. Especially on the early stages of the newly formed galaxy, the so-called relativistic galactic winds which are created by the cosmic rays have a crucial implication on the matter distribution. Without cosmic rays, it is impossible to explain the formation of the galaxy arms (K$\clubsuit$). On the Earth scales, cosmic rays seem to be responsible for the initiation of clouds and bolts of lightning. Though cosmic rays studies started a hundred years ago, only now we begin to realise their real significance.

It is interesting to learn about the detection techniques for the cosmic rays. The ``low'' energy cosmic rays are detected by the specialised satellites on the Earth orbit. Mostly they are ordinary particle detectors, just taken out to space. The mentioned ``low'' energy part goes up to about a hundred TeV. The cosmic rays of even higher energy start being too energetic and too rare to detect them from space, so they are detected on Earth. It is not the primary particles that are detected in this case, but the {\it extensive showers} of the secondary particles, created in the interactions of the cosmic rays with the atmosphere. The detectors for the highest energy particles are required to observe an enormous volume of the atmosphere. The largest cosmic ray observatories have surfaced of the order of a thousand square kilometres.
}{{Cards.033}.png}

%--------------------------------------------------------------------------------

\card{Qb.png}{club.png}{Jocelyn Bell Burnell -- pulsars}{In 1967 Antony Hewish built the Interplanetary Scintillation Array -- a radio telescope, mainly designed for the observation of the recently discovered quasars (8$\clubsuit$). The young postgraduate student Jocelyn Bell worked there on her PhD thesis. She analysed the experiment data -- long paper stripes of chart-recordings. One day she noticed something strange: a signal from one particular point in the sky, that was pulsing with great regularity of a bit more than one second. Astrophysicists knew already the periodic processes, like, for example, the cepheids, variable stars, that vary their colour and brightness. But their period is in order of days, not seconds! Jocelyn's first idea was that it is the alien's radio signal; she even called it the little green men.

However, later astrophysicists confirmed that these unknown sources are not artificial. Yet, they are quite strange. These are neutron stars, or since they are fast pulsing -- pulsars. We already mentioned this exotic stars before, so you know that they appear after a supernova explosion of a massive star. The fusion reaction doesn't undergo in the neutron star core, so nothing can keep the material from shrinking to an extremely dense state. The atoms approach so close to each other that they disintegrate, the electrons become squeezed into the nuclei, turning the protons into the neutrons. So the star's core starts looking like a gigantic atomic nucleus, that consists of neutrons only and has the size of a city.

Remember you in childhood, when you were playing on a playground on a carousel. My favourite thing was to turn the carousel fast and then move as close as possible to the centre. The carousel was starting to rotating much-much faster! The same happens to the pulsars. The star before its death is rotating with some speed. But when its core turns to the neutron star, which is orders of magnitude smaller than the original star, it starts rotating super-fast. There are powerful jets of radiation from the magnetic poles of the neutron stars. These jets are rotating with the star. For an observer on the Earth, they look like if someone would rotate a flashlight on a lace. The fastest known pulsars make almost a thousand rotations per second.

For the sake of completeness, we should mention that the name of a pulsar is reserved for any astrophysical object that makes the pulsations. Sometimes scientists observe the white dwarfs that manifest the same behaviour. However, most of the pulsars are neutron stars, and quite often these terms are used as synonyms.
}{{Cards.034}.png}

%--------------------------------------------------------------------------------

\card{Jb.png}{club.png}{Stephen Hawking -- black hole}{
We said that the gravity creates the so intense pressure inside a neutron star (Q$\clubsuit$) that it becomes like a giant atomic nucleus. It is so dense that a teaspoon of the neutron star material would weigh a billion tons! But still, you can guess that something creates a counter-force that keeps the neutron star from squeezing even more. It is the so-called pressure of degenerate neutron gas.  However, if you make a neutron star even more massive, the gravitational attraction would overcome even this pressure. The star, in this case, turns into the black hole.

The existence of the black holes was first guessed by a German physicist Karl Schwarzschild, who in 1915 solved the equations of general relativity of Einstein (K$\spadesuit$) for the simplified case of a star -- an isolated spherical non-rotating object. You have probably already heard that the black holes are such massive objects that even light cannot leave it. This is true. It happens because space next to the black hole becomes ``curved''. Moreover, the black hole itself represents an area of space which is extremely curved. In the centre of a black hole, there is the {\it singularity} -- or the point, where the curvature of space reaches infinity. Or, wait for a second, postulating infinities in a physics book is too offensive. Let's better say that we don't know what happens in the centre of a black hole. Still, what {\it would} occur in the point of infinite curvature? Well, if you would go from the border of a black hole to its centre, you would have to cover an endless path, even though you started some limited distance from it. There is an infinite space inside the black hole. So it is literally a hole in the fabric of space. And it is black because the light cannot leave it.

The distance from the centre of a black hole, starting from which a photon can finally leave it, marks its boundary and is called the {\it horizon}. We said that the physical vacuum (6$\heartsuit$) looks like a strange field where particles pop-out from thin air and disappear again at every moment in every point of space. What happens when a pair of particles appears next to the horizon of the black hole? Then it is possible that one particle leaves the vicinity of the black hole and another falls inside. For an outside observer, it looks like the black hole is emitting particles. That is it is losing its mass through this radiation, which is called after Stephen Hawking who discovered it. If the black hole is massive than it is hard for the particles to leave its gravitational field, so they take out only very little of momentum. Thus the large black holes evaporate -- this is the physical term for this phenomenon -- quite slowly. For the supermassive black holes, evaporation can take over $10^{100}$ years.
}{{Cards.035}.png}

%--------------------------------------------------------------------------------

\card{7b.png}{club.png}{Cosmic magnetism}{
Cosmic magnetism starts with the well-known effects of the geomagnetic field,  that protects us from the large part of the cosmic radiation -- very abundant low-energy cosmic rays (5$\clubsuit$) and solar winds (2$\clubsuit$). The Sun also possesses a vast magnetic field. On a large scale, Sun creates the heliomagnetic sphere that spans much farther than the orbit of Neptune. These two fields act like shields from the galactic radiation, important protection that assures the existence of life on Earth.

The magnetic field on the scale of a galaxy forms a spiral-arm structure, which goes parallel to the pattern of the visible arms. The orientation of the field in the neighbouring arms is reversed. The galactic magnetic fields drive the mass flow and affect the formation of the spiral arms. The smaller-scale magnetic fields are also necessary for the star formation: without a magnetic field, the gas cloud starts rotating too fast and is not collapsing to the centre to make a newborn star.

The largest known magnetic fields exist in the clusters of galaxies (4$\spadesuit$). These fields are essential for the cluster dynamics, providing additional pressure to the material.

The larger magnetic field is, the weaker it is. The galactic field is about a million times smaller than it in the geomagnetic sphere. All they arise from different mechanisms, but there is something in common: electric current. Whenever you have an electric current, you inevitably get a magnetic field. There is no other mechanism, because, unlike for the electric field, magnetic field sources don't exist. The geomagnetic field is created by the currents in the core of the Earth. But how the magnetic fields appeared in galaxies is still a mystery. Theories say, that normally, it had to dye away in the young galaxies.
}{{Cards.036}.png}

%--------------------------------------------------------------------------------

\card{8b.png}{club.png}{Quasar}{
Most galaxies are pretty quiet islands of stars. But some of them have a nucleus, which turns active. The active galactic nuclei are called ``quasars''. Today we know, that there is a supermassive black hole (J$\clubsuit$) in the centre of a quasar. Its mass is typically from millions to billions of solar masses. The material around the black hole forms an accretion disc. When it falls on the black hole, it accelerates to such speed that it starts emitting extremely energetic radiation. It comes out in the form of two jets from the poles of the quasar. These jets are ones of the most probable candidates to create the ultra-high energy cosmic rays (5$\spadesuit$), the most energetic particles ever detected on Earth, that reach the energy of hundreds of exa-electronvolts.

Why not every galaxy possesses a quasar? It turns out that almost every galaxy has a supermassive black hole in the centre. The Milky Way does have one, but it is not a quasar. For a black hole to be a quasar, the accretion disk should be large -- even if you feed a black hole with a star, it provides not enough material to form the accretion disk and to turn the black hole into a quasar.

However, we have evidence that our central black hole (by the way, it is in the constellation of Sagittarius and is called Sgr A$^*$) was a quasar earlier in its life. Today we observe hard photons coming from the gigantic gaseous structures on each side of the Milky Way. They span for about the same distance from its plane as its radius. These are the Fermi bubbles. Perhaps they were created during the quasar phase of Sgr A$^*$.

Nowadays quasars are studied by the means of the multimessenger astronomy: they are observed in a wide range of frequency of electromagnetic radiation, from radio to X-rays. The origin of ultra-high energy cosmic rays from quasars is yet not confirmed. And eventually, we can detect quasar neutrinos (7$\heartsuit$). That would be extremely valuable to constrain the existing models of the extreme quasar environment.

Quasars are the most bright sources of light in the Universe. But a quasar can be even brighter if its jet is pointed towards us. In this case, it is called a blazar -- but physically it is still the same object.
}{{Cards.037}.png}
%--------------------------------------------------------------------------------

\card{9b.png}{club.png}{Exoplanets}{
Exoplanets are the planets that rotate around other stars, just like the planets of the Solar system (3$\clubsuit$). For today over four thousand exoplanets have been discovered, revolving around nearly three thousand stars.

There are many methods to observe exoplanets. The first exoplanets were found on the orbits around a pulsar (Q$\clubsuit$) in 1992. Those planets influenced the rotation of the host neutron star, which was observed as the variability of its pulses. However, these planets would be better described as debris after the supernova explosion (6$\clubsuit$) that created the neutron star. The first exoplanet that rotates around a star of the main sequence (A$\clubsuit$) was discovered in 1995. It was a super-heavy planet that rotates exceptionally close to its host star: its orbiting period is only three days -- compare it to the period of Mercury which is about two months. Planets of this type were named hot Jupiters: they are hot because they are in the very vicinity of a star, and they are as massive as Jupiter. For this kind of systems, you better say that the star and the planet rotate around each other. You can detect the speed of the star by the slight change of its colour. From this observation, you can conclude about the existence of an exoplanet.

The method of transit photometry, depicted on the card, allows detecting smaller planets. When a planet passes in front of the star, it casts a shadow and makes the star look slightly fainter. This method allows for detecting the size of an exoplanet. But it is only applicable when the exoplanet rotates in the plane, parallel to our line of sight. Most of our detection methods suffer from the {\it observation bias} -- the fact that we only observe the observable objects. So to estimate the real number of exoplanets in our neighbourhood, scientists have to build models and correct them with what we see. The number of observed planets will always be lower than their total amount.

So one can conclude that there is a lot of exoplanets. And many of them are in habitable zones, that is they are close enough to their host stars to have liquid water on the surface. If besides, they have enough carbon, they could become inhabited. If this extraterrestrial life would turn conscious and have civilisation, eventually they could start asking the same questions as us. And maybe, if they would be technologically developed, they would send us a radio signal, or would visit us in person. But it didn't happen so far (some people say that it's for good). This paradox is called after the famous physicist Enrico Fermi: we see there are many exoplanets that can maintain life, but why we don't see the aliens? Could it happen that intelligent life is extremely rare in the Universe and we are alone in our galaxy? Or maybe the developed alien civilisation realise some truth that stops them from sending signals here and there? Nobody knows the answers to these questions, but it should not cease us from the investigation and guessing!
}{{Cards.038}.png}
%--------------------------------------------------------------------------------

\card{10b.png}{club.png}{Gravitational wave}{
Most of the discoveries in astrophysics are made through the observations of the light. It could be radio-signal, or visible light, or X-ray light, still it is essentially the same -- electromagnetic radiation. One can also study the cosmic rays (5$\clubsuit$), but it is more complicated: since the cosmic rays have charge, they get deflected in the galactic magnetic field (7$\clubsuit$), so they don't point back to their source. Thus we can only guess about their origin. Neutrinos (7$\heartsuit$) are more reliable: they are very light, and they have no charge. However, neutrinos interact so weakly (4$\heartsuit$) that they are tough to detect.

A completely new type of astrophysical detection appeared only a few years ago. In 2015 three observatories in America and Europe detected their first gravitational wave. First predicted by Einstein, these waves imply the periodical contraction of the space-time fabric itself. Only the large masses in tremendous acceleration can create them. For example, that first gravitational wave was created by two black holes, each about 30 solar masses, rotating around each other 75 times per second at the distance of only 350 km. That was just before they merged. At this brief moment, they emitted an incredible amount of energy, equivalent to 3 solar masses, in the form of a gravitational wave. This wave eventually reached us and created the space contraction of just about $10^{-18}$m, or a thousand times smaller than the size of a proton. The detection of such a tiny effect became possible thanks to an ingenious interferometric nature of the observatories and scrupulous accuracy of its construction.

A gravitational wave is a unique instrument to study the black holes, which are otherwise invisible. Knowing the masses of the black holes, scientists can conclude about the stars that created them, what were the supernova events, and how they affected the surroundings. Observation of the gravitational wave is also an unprecedented probability to test the theory of general relativity (K$\spadesuit$), which is the current standard theory of gravity and the foundation of the modern cosmology. 

Up to now, we count about a dozen gravitational wave events. Most of them came from the mergers of two black holes, but there is also one significant event from the merger of two neutron stars. These objects are so dense that they can create a detectable gravitational wave too. The future observatories will be able to detect events like this regularly.
}{{Cards.039}.png}

%--------------------------------------------------------------------------------

\thispagestyle{fancy}
\fancyhf{}
\renewcommand{\headrulewidth}{0pt}
\lhead{\thepage \hskip14pt Physics Is My Favorite Game}
\fancyfoot{}
{\huge{\textbf{Suggested materials}}}
\vskip12pt
The astrophysics is so broad as a field that it is hard to cover it with just one book or one resource. But it is also the branch of physics that you can learn in the hands-on mode. Buy a cheap telescope -- even the simplest ones you can find today in the shop are much better than the telescope of Galileo -- and you can start studying the Moon craters, see the Jupiter and its moons, see the Saturn's rings and some bright nebulae. Or even without a telescope -- go to a place where you wouldn't have the city lights in the night, and enjoy the majestic view of the Milky Way. Install Stellarium (stellarium.org) on your computer and learn our stellar geography with this virtual planetarium.

Try to start following the astro-news, for example, on universetoday.com. In the beginning, you will feel lost, but after some time, you'll get used to it and will start to understand the discourse. I swear it worth it!

Follow the news from the leading observatories. Most of them have an account on Twitter. Look for the posts of the European Southern Observatory (ESO), ALMA radio observatory, Pierre Auger cosmic ray observatory, IceCube neutrino telescope -- already not bad for starters.

For the popular books, I can recommend the ``Astrophysics for People in a Hurry'' by Neil deGrasse Tyson. Unfortunately, if you look for something really tough and serious on the astrophysics as a whole, it is hard to give you any recommendation. In fact, for every card of the club suit, you would find quite a thick book that explains the subject in details.
\newpage