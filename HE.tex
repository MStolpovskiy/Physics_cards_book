\mypart{Particle Physics -- Hearts}{Particle Physics \\Hearts}{heart.png}{Now, as you understand the principles of quantum mechanics, it is time to discuss the fundamental constituents of matter. We already mentioned electrons, photons, as well as protons and neutrons. But first, the latest two are not elementary particles, that is they are divisible into smaller parts. And second, there are way more particles existing in nature. This part of physics is often called the high energy physics, on the contrary to the low energy quantum physics which we discussed before.}

\card{Qr.png}{heart.png}{Emmy Noether -- symmetry}{
Emmy Noether is a very exceptional figure in this deck. While other scientists depicted here are physicists, she is the only mathematician. You may ask: what does she do here then? Oh, she is here on purpose. If I only could, I would make her the king. Her idea of symmetry serves as the foundation of the whole area of particle physics.

It's quite easy to express the Noether's idea, but not easy to explain. The idea is the following: any symmetry in Nature implies some conservation law. Let's first clarify what here meant by symmetry. Usually, we say that an object is symmetric if, for example, its left side looks the same as its mirrored right side. But what if the object looks the same after being moved at some distance? From the mathematical point of view, it is also a symmetry, called translation symmetry (here an object may mean not only a single object, but a system of objects too). If the object looks the same after some time passed, then this kind of symmetry is called time translation symmetry.

If a body has the same speed after passing some distance, then it is symmetric under space and time translations. But from the simple mechanics, we know that if a body has the same speed, then it's momentum and energy are conserved. Precisely, the space translation symmetry corresponds to the momentum conservation and the time translation -- to the energy. If you ever wondered why the physics teacher in school insists that these quantities conserve, here is the reason: because Nature is symmetric.

In the field of particle physics, the Noether theorem becomes truly special. Sometimes the particle physicists are even called the "symmetry hunters". What does it mean and how does it work we will be learning all through the hearts suit. Generally, we are looking for some hidden symmetries, which imply some conservation laws. These symmetries -- hence the conservation laws -- could be sometimes broken, which means that the corresponding quantity doesn't always conserve and the processes, being symmetrically translated, don't always go the same. Particle physics is all about these symmetries. And very often we even use the words ``symmetry'' and ``conservation law'' as synonyms.}{QHearts.png}

%------------------------------------------------------------------------------------

\card{Kr.png}{heart.png}{Paul Dirac -- antimatter}{
In 1928 Paul Dirac considered the movement of an electron on the orbit around an atom nucleus. We've spoken already about this system (4$\diamondsuit$). In turned out that the equation Dirac wrote for the electron also has another solution. It describes a particle identical to the electron but with the opposite electric charge. Such a particle was named positron because it has a positive charge (while the charge of an electron is negative). The Dirac's equation is symmetric on the charge reversal. It necessarily means that, since the equation is mathematically correct, there should exist these anti-electrons, the positrons.

Soon the positrons were discovered in the cosmic rays (5$\clubsuit$). But what's more, it turned out that every elementary particle has its anti-companion. Hence, as the ordinary matter consists of particles, the matter that consists of antiparticles is called antimatter. The particles and antiparticles annihilate upon collision: they disappear, radiating a large quantity of energy and (optionally) creation of new particles. According to the Noether theorem (Q$\heartsuit$), there should exist a conservation law that corresponds to the symmetry between matter and antimatter. It is the law of the charge conservation: in any interaction the total charge conserves.

But if the antimatter is just a symmetric reflection of ordinary matter, then where is it? If the Big Bang (J$\spadesuit$), from which the Universe started, was purely symmetric, then there should exist the equal quantity of antimatter as the amount of ordinary matter. Still, one of the biggest questions to modern physics is the absence of antimatter in the Universe. It is called the baryogenesis problem because in cosmology the ordinary matter is generally called the baryonic matter (the term \textit{baryons} for the particle physics means protons and neutrons and other similar particles). Thinking about this problem, scientists came up to the minimal requirements to generate this disbalance. I'm speaking about the Sakharov conditions for baryogenesis. One of them is -- guess what -- the violation of the charge symmetry. Then this violated symmetry should produce the observed disbalance of matter and antimatter. Another approach to the solution of this problem we will mention later when we will be discussing the inflation theory (9$\spadesuit$).

Finally, one should mention that matter-antimatter symmetry implies not only conversion of the electric charge, but all the charges. For example, when we will get acquainted with the strong interaction (5$\heartsuit$), we will learn about the colour charge. So if a particle (a quark) has, let's say, the red colour charge and electric charge $2\over3$, then its antiparticle will have colour charge anti-red and electric charge $-{2\over3}$. 
}{KHearts.png}

%------------------------------------------------------------------------------------

\card{2r.png}{heart.png}{Gauge}{
Gauge is one of the essential notions in particle physics. It is the way to introduce all the interaction.

In classical physics, we had two types of interactions. The first type is the direct interactions of bodies when they touch each other: for example, the collision of two billiards balls on the table. A different kind of interactions is more interesting because it can act without a touch, on distance. For example, if you rub a balloon on your hair and then move it away from your head, you will see that the hair will be pulled towards the balloon. But there is nothing between the balloon and the hair. It is an example of field interaction. The hair is pulled to the balloon because of the electromagnetic field between them.

Particle physics attempts to explain the field interactions by the exchange of some mediate particle.  The diagram on the right shows a process like this: an elastic scattering of a quark on an electron. The time axis is from left to right. At the beginning (left) there are a quark and an electron, and in the final state (right) it is the same, that's what the term ``elastic'' means. But there was a moment when these two particles interacted. They interact by the exchange of a photon. In this kind of charts (called Feynman diagram) most of the particles, like quarks or electrons, are denoted by solid lines with arrows. Unlike in this example, particles can change to another particle after the interaction. The mediate particles are shown with wavy lines. Each line type is reserved for some mediate particle. The mediate particles are bosons (remember? These guys have integer spin). One of the fundamental bosons which we already acquainted is the photon, the mediate particle of the electromagnetic interaction.

So between your hair and the balloon, there are photons, flying there are back and transmitting the attractive interaction. The same is going on the microscopic scale when we speak about the collisions of tiny particles. The particles don't collide like the billiards balls. Instead, they exchange a photon, or another boson, and thus interact. What we told before about the collision of billiards balls -- on the fundamental level -- is nonsense: if you zoom in this collision and look what's going on between the atoms of one ball and another you will see that the atoms don't collide. Instead, when they approach each other, they exchange the photons and thus repulse. The way to explain the field interactions by the exchange of a mediate particle is called gauge theory. All the fundamental interactions are gauge interactions, with one exception: gravity. Gravity physics is entirely different. We will learn this difference later (K$\spadesuit$).
}{2Hearts.png}

%------------------------------------------------------------------------------------

\card{4r.png}{heart.png}{Electroweak interaction}{
Here is the first fundamental interaction that we will be considering -- the electroweak one. Quite often, scientists speak about two distinct interactions: weak and electromagnetic. We are all familiar with the electromagnetic part. It is not only the basis of all the electronic devices. The electromagnetic force holds the electrons within the atoms and the atoms within the molecules. On its turn, the molecules attract each other again by the electromagnetic force. That is the existence of matter as we know it in everyday life is due to the electromagnetism. This interaction is mediated (2$\heartsuit$) by the photons (K$\diamondsuit$ and 2$\diamondsuit$)

Now let's consider the weak part, which for a long time believed to be quite distinct from the electromagnetic interaction. The weak interaction can apply to all the particles, but its strength is extremely low. That is the weak interactions are the rarest of all. A classic example of the weak interaction is the $\beta$-decay of a neutron (remember that the atom nuclei consist of protons and nuclei). The neutron on its turn consists of three quarks -- we will be learning about the quarks few cards later (5$\heartsuit$). In the weak neutron decay, one of the down quarks (\textit{down} here is just a name), depicted on the picture with a little blue ball with a letter \textit{d}, changes to a small red ball with \textit{u} -- the up quark. In this process, the quark changes its electric charge from $-1\over3$ to $2\over3$, so it gains $+1$. This charge the quark gets from the W boson (yellow wavy line). W is the fundamental boson of the weak interaction. It is the only boson that possesses the electric charge. There is also another weak interaction boson called Z (which has zero charge). Both W and Z bosons are quite distinct from all other bosons because they have non-zero mass. Their mass is not just non-zero, but these are ones of the most massive known fundamental particles. They are about 90 times heavier than a proton.

The fact that these particles are so massive explains the weakness of the weak interaction. As stated in the uncertainty principle (J$\diamondsuit$), the energy of a particle can be uncertain. It means that we can ``borrow'' some energy from nowhere, from the uncertainty, to create a massive W or Z boson. However, the probability of succeeding in it is reduced due to the large mass of these particles. Remember also that the considerable borrowed energy means very precisely defined position. That is the typical range of weak interaction should be quite short. It is about $10^{-17}$ meters or one-hundredth of the size of a proton. Hence to see two particles interacting weakly, you have to approach them very-very close to each other.
}{4Hearts.png}

%------------------------------------------------------------------------------------
\card{3r.png}{heart.png}{Symmetry breaking}{
While considering the electroweak interaction (4$\heartsuit$), we mentioned that it consists of two, but didn't explain what does it mean. Let's fill this gap now.

We told about the notion of symmetry (Q$\heartsuit$) and its importance to the particle physics. But what is even more interesting is the phenomenon of the symmetry breaking, when the symmetry becomes imperfect. Hence, the corresponding conservation law violates.

Take a wine bottle. As you know, the wine bottles have a convex on the bottom (if you look from inside). Take a marble and drop it inside. Since you drop the marble right on top of the centre of the bottle, one could naively expect that the marble will stay there atop the convex. However, you will never get this situation. Instead, the marble will roll down to the side of the bottle. Now, what happened: you started with a symmetric situation -- symmetric bottle and symmetric position of the marble relative the central bottle axis. But you end up with a non-symmetric state, though this asymmetric final state is the most stable one.

The same happens in particle physics and is called spontaneous symmetry breaking. The ``wine bottle bottom'', depicted on the card, is the \textit{potential}. When the particle rolls down the potential to reach its minimum, it obtains mass. The electromagnetic force, mediated by photons, represents the symmetric realization of the electroweak potential, without a convex. However, sometimes the circumstances change, and the potential becomes like it's shown on the picture. Then the photon becomes massive and turns to the W or Z boson.

What makes the electromagnetic force turn to the weak force? What changes the potential? It is the Higgs mechanism, associated with another fundamental particle (which later became very famous) -- the Higgs boson. Higgs mechanism, initially introduced for the electroweak interaction, also acts to other particles and makes them massive. On the fundamental level, the particles have mass because they interact with the Higgs boson.

The Higgs mechanisms explain why the tiny constituents of the matter have mass. But it is very little responsible for the masses of macroscopic bodies. For example, a proton consists of three quarks, but each of these quarks has masses 200-500 times smaller than the mass of the proton. It means that the mass of a proton is only by 1\% explained by the mass of the quarks (hence by the Higgs mechanism: the quarks have their masses because of Higgs boson). The extra 99\% of the mass is actually due to the binding energy of quarks within the proton. The same is true for neutrons. The protons and neutrons are the constituents of atomic nuclei. Thus they are the main contributors to the mass of macroscopic bodies.
}{3Hearts.png}


%------------------------------------------------------------------------------------
\card{5r.png}{heart.png}{Strong interaction}{
We have already mentioned several times the quarks -- the constituents of protons and neutrons. They are probably the most unusual particles. All other particles have charge either -1 (like an electron) or 0 (like Z boson) or 1 (like $W^+$ or proton). The quarks are different. They have charge either $2\over3$ or $-{1 \over 3}$. There are six quarks, and two of them are the most important ones -- the \textit{up} and \textit{down} quarks. The up quark has charge $2\over3$, and the down one has $-{1\over3}$. The proton is a bound state of two up quarks and one down. And the neutron is a bound state of one up and two downs.

Since the quarks have the electric charge, they can interact electromagnetically. We have also seen that they can interact weekly (4$\heartsuit$). But the most interesting is that they
have an interaction exclusively reserved for them -- the strong force. The mediating particle (2$\heartsuit$) for the strong interaction is gluon (because it ``glues'' the quarks together).

Quarks not only possess a fractional electric charge. They have also another charge of different nature, called \textit{color} charge. While electric charge can be either negative or positive, so two options, the colour charge can have six possible values: red, green and blue, and also antired, antigreen and antiblue. Each quark has one of these colours. Gluons have two colour charges at the same time. For example, there could be a red-antigreen gluon. Quarks in a proton are unceasingly exchanging gluons. Thus their colour charge always changes. Don't let this colour notation perplex you. There is no actual colour for the elementary particles. What we call here colour is just a strong charge. And the colour terms were artificially chosen to illustrate the idea that the observed states, like protons and neutrons, are always white, or strongly-neutral.

And this is the most exciting property of the strong interaction. The strong force is so strong, that when you try to pull one quark from some bound state of several quarks, it immediately creates new quarks: one to replace the quark that was pulled out, and another to make a pair to that lonely pulled out one. In an experiment, we can never see a lone quark. We can only see systems of three or two quarks. The system of three quarks exists because the quarks have colour charges red -- green -- blue, hence white in total. And the two quarks may have, for example, red -- antired charges, so be again white.

It should be mentioned that the strong force not only binds the quarks inside the protons and neutrons but also indirectly explains the attraction between protons and neutrons. This attraction is also a gauge interaction (2$\heartsuit$), but it doesn't act through any fundamental boson. Instead, the protons and neutrons exchange with the double quark states called pions. The pions are created inside the protons and neutrons due to the uncertainty principle. }{5Hearts.png}

%------------------------------------------------------------------------------------
\card{Ar.png}{heart.png}{Standard Model}{
Mainly we have already introduced all the fundamental interactions that we know exist in Nature. The Standard Model (note the Upper Case!) is the way to systematise this knowledge.

First, there are six quarks (5$\heartsuit$), divided into three generations (or flavours). The first generation is the already mentioned up and down quarks, the second is the charmed and strange quarks, and the third is the top (or truth) and bottom (or beautiful) quarks (from those names you learn that physicists are quite romantic). You can also divide the quarks into two other groups: the up-type and down-type. The up-type is: up, charmed and top. The down-type is: down, strange and bottom. The up-type quarks are more massive than the down-type quarks from the same generation. And the up-type quarks have electric charge $2\over3$, while the down-type ones have charge $-{1\over3}$. The masses of the quarks increase with each generation. For example, the top quark is more than $50000$ times heavier than the down quark. On the card, the relative masses of the particles are illustrated with the darkness of the colours.

Then there are six leptons. They also divide into three flavours. In each flavour there are one charged particle: electron, muon and tau (the latter two have the greek letters to denote them: $\mu$ -- mu and $\tau$ -- tau). And there are also neutral particles, neutrinos. We will learn about them later (7$\heartsuit$). The charged leptons also become more massive from electron to tau: the electron is about 200 times lighter than the muon, and almost 4000 times lighter than the tau. The charged leptons don't interact strongly, because they have no colour charge. But they can interact with the electroweak bosons (4$\heartsuit$): photon (denoted with the Greek letter $\gamma$ -- gamma), W (these bosons are charged. Depending on their charge they are indicated as W$^+$ or W$^-$), and Z boson. The neutrinos have no electric charge, so the only interaction left for them is the weak one. Hence, the probability that a neutrino interacts with even a considerable amount of matter is negligibly small. For example, the Earth is almost entirely transparent for neutrinos.

All these particles obtain their masses by interaction with the Higgs boson (H in the centre). This particle was found experimentally in 2012, and it finalises our understanding of the Universe on the elementary level.

Does it mean that there is nothing left to study? Not at all. First, not all the parameters of the Standard Model are well measured. But the most exciting task is to look for the particles beyond the Standard Model -- for example, the hypothesised supersymmetry particles (9$\heartsuit$).
}{AHearts.png}

%------------------------------------------------------------------------------------
\card{Jr.png}{heart.png}{Synchrotron}{
We already mentioned that if you want to observe some interactions, you need to draw particles very close to each other. For the electroweak interaction (4$\heartsuit$) it is, for example, about $10^{-17}$m. The strong interaction (5$\heartsuit$) is not much easier to achieve: since the gluons don't allow the quarks to fly away from them, the typical range of the strong force is about $10^{-15}$m. However, accomplishing such a rapprochement, it is not that easy. Typically, particles repel. To overcome this effect, physicists smash particles into each other on the speeds close to the speed of light. But accelerating particles is not an easy task.

Think of the electric current. Essentially it is just the movement of the electrons in the wires under the action of the electric field. The same is valid for the other charged particles. By simply putting a proton between the plus and minus charges, you can accelerate it to some energy. But the more working idea is to put a bunch of protons in a variable field and make them fly on a circle. Veksler first showed that in this case the protons are ``surfing'' the wave and don't fly apart. The particle accelerating machine that works on this principle is called the {\it synchrotron} -- because the electric field is synchronous to the particle beam. Lap after lap the protons gain the energy. Then, once the particles reach the required energy, they are expelled from the accelerator ring and fly to the experimental setup.

Why the protons? Well, you need some charge to accelerate the particles, so obviously one can't use the neutrons. Electrons could also be used in the accelerators. However, the electrons moving on a circular orbit emit their energy, so you cannot accelerate them very much. Protons are more massive, and this effect is way less severe for them. In modern accelerators, they also use the nuclei of the elements. The most popular choice is the nuclei of carbon or lead. Also, you can study a lot of exciting physics by accelerating muons (A$\heartsuit$), which are $\sim200$ times heavier than electrons. But the projects to build such machines have only recently appeared.

While the energy emission due to the circular movement of a particle is unwanted in the task of gaining maximum energy, this emission itself could be used in a lot of applications. Since it has high intensity and comes in very narrow beams, it is used, for example, to make ultra-high resolution images of in biology or material science. Nowadays the mood has changed, and precisely this type of machines are now called synchrotrons -- or being more precise, sources of the synchrotron radiation.

Most modern accelerators are colliders. It means that they have two accelerators, usually implemented in a single ring, where the bunches of particles rotate towards each other. It allows increasing the energy of the collision by factor 2 while using the machinery of the same power.
}{JHearts.png}


%------------------------------------------------------------------------------------
\card{6r.png}{heart.png}{Vacuum}{
What is the vacuum? In the most simplistic explanation, it is just emptiness. In the more accurate form vacuum is what lefts when you remove everything from some space. You may ask: what is the difference? If you remove everything, then there is nothing lefts, just empty space! No, not really.

Let's take a closer look at the vacuum and consider a very tiny spot in it. Very tiny means that you know the position and the size of this spot with high precision. According to the uncertainty principle (J$\diamondsuit$) if you know the position very well, then you become very uncertain about the energy in this spot. This uncertain energy manifests itself in the form of particles that emerge out of thin air (though there is not even air in the vacuum). These are pairs of symmetric particles: a particle and its antiparticle. They appear and immediately annihilate, releasing the energy borrowed from the vacuum. Such particles are called \textit{virtual}. Their appearance is also called \textit{vacuum fluctuation}.

There are no effects of this physical vacuum in everyday life. However, it changes everything once you go down to microscopic scales. In any interaction, the virtual particles may appear at the place and time of the interaction and influence the resulting process. It is very useful for physicists: the mass of the particles that you can create in the experiment is limited by the energy of the collision. While the mass of the virtual particles is not limited at all. Hense, through the vacuum you can access to the energies, you would not be able to reach with just an accelerator. The vacuum can also make the impossible interactions possible. For example, photons do not interact with each other. However, when two photons approach each other, a vacuum fluctuation may create a pair of charged particles in the same spot. In this case, the photons would interact with these particles. Efficiently at the end of the day, we get the photon-photon scattering.

To be precise, the bosons that mediate any interaction, and all the other possible intermediate particles are also called virtual. To explain this, let's return to the electroweak card (4$\heartsuit$) and ask ourselves a simple question: what charge does the W boson have? Let's consider it is flying from top to bottom. Then it carries the negative electric charge, obtained from the down quark. Thus it is W$^-$. But similarly, it could be a positive W$^+$ that flies from bottom to top and balances the negative charge of the electron. What option do you prefer? Although in this particular example the first possibility seems more logical (otherwise you should assume a W-e-$\bar\nu^e$ triplet appearing out of nowhere) I assure you that there is no actual difference. This W boson is virtual, and it could be equivalently considered as W$^+$ or W$^-$.}{6Hearts.png}


%------------------------------------------------------------------------------------
\card{7r.png}{heart.png}{Neutrino}{
The last fundamental particle that we haven't yet considered is neutrino (denoted with the Greek letter $\nu$ -- nu). It belongs to the class of leptons, but unlike the electron, muon and tau, the neutrino has no electric charge. What is also interesting, they have almost zero mass. What mass they have remains undefined. But they are not massless, that's for sure. As we already told, there are three flavors of neutrino: $\nu^e$, $\nu^\mu$, $\nu^\tau$. Each of those may convert by weak interaction to the corresponding charged lepton. There is no other way for neutrons to interact with matter, only weakly. And the weak interaction is so weak and rare that a neutrino can fly through the Earth without any interaction.

The most exciting phenomenon with neutrino is its oscillations. Neutrino changes its flavour on the flight. Thus, for example, a tau neutrino can suddenly become $\nu^\mu$ and then $\nu^e$. This effect was found when physicists measured the flux of solar neutrinos. In the detector, we can see the signal of the muon or the electron, created when the neutrino by chance interacts in the detector medium. But the tau lepton from the $\nu^\tau$ is usually missed. Hence the observed flux of neutrinos is always much lower than the expected one because the detectable $\nu^\mu$ and $\nu^e$ convert to the undetectable $\nu^\tau$.

This behaviour is related to the wave nature of any particle. The neutrino ``consists'' of three oscillating parts. Or better saying the neutrino is the quantum mixture of three neutrino mass states. The relative amount of these states in each neutrino changes with time. To illustrate it, let's conduct a little experiment. And this one will be again with a music instrument. A tremolo harmonica this time (if you don't have one, try to search a sample sound on the internet). The tremolo harmonica sound is unique. It sounds like ``whoa-whoa-whoa''. It happens because, inside this instrument, there are two reeds for each note. They are slightly mistuned relative to each other. This ``whoa-whoa-whoa'' is created when the waves from the different reeds add up.

The same happens inside the neutrino. The mass states of neutrino vibrate inside it like the reeds. These vibrations interfere with each other, resulting in a kind of neutrino ``whoa-whoa-whoa'' -- neutrino oscillations. The only difference is that inside the neutrino there are three ``reeds''. The oscillations necessarily imply that the neutrino has mass (which is not measured so far). Still, they are the lightest massive particles.

Nowadays the scientists suspect that there is the fourth neutrino flavour, which manifests itself by the oscillations observed very close to the neutrino source (usually we use a nuclear reactor as a neutrino source). This hypothetical fourth state is called \textit{sterile} neutrino.
}{7Hearts.png}


%------------------------------------------------------------------------------------
\card{8r.png}{heart.png}{Quark-gluon plasma}{
When we considered the strong interactions (5$\heartsuit$), we told that the quarks could not fly out from the colourless bound state of two or three quarks. This property we call \textit{confinement}. In terms of force, this could be explained as follows: the attraction between the quarks increases with the distance. It is very unusual if you think of it a bit: all the attraction forces we see in everyday life decrease with distance. The planets ``feel'' the attraction of the Sun, because they are relatively close to it. And the attraction to the other stars in the Galaxy is practically negligible. If the quarks were planets, the situation would be the opposite: the farther the planet is from the star, the stronger the attraction between them is. However, this doesn't work forever. At some point, when you take away two quarks from each other, the gluon connecting them breaks, creating a pair of new quarks. And you end up with four quarks in two pairs.

A bit of terminology: the possible colorless bound states of quarks are called -- \textit{mesons} for two quarks state, \textit{baryons} for three quark state and \textit{glueballs} for the unstable colorless multiquark states. Mesons and baryons are also called \textit{hadrons}. 

On practice, how can you pull a quark from a hadron? You cannot grip it with the tweezers! No, instead you accelerate the hadron in the synchrotron (J$\heartsuit$) and smash it into another hadron. If the energy of the collision is greater than the binding energy of the hadron, it breaks apart on several less energetic hadrons. If they are still too energetic, they can divide again. And again and again. At the end of the day, you see the \textit{hadron jet}: the shower of a vast number of quarks. This process is called \textit{hadronization}.

As you can see, by an increase of the collision energy, you cannot see the free quark, because they fly apart creating the new hadrons. But what if instead you hold the quarks in a tiny volume and increase the energy? Like this, you would break the connections between the quarks and would not let them fuzz. This exotic state is called the \textit{quark-gluon plasma}. To create this state, you smash the nuclei in the collider. Then, for a tiny fraction of second, at the collision point, you have this quark-gluon plasma, where they fly freely.

The quark-gluon plasma is believed to fill the Universe just a millisecond after the Big Bang (J$\spadesuit$). Thus the study of this state is directly related to the cosmology (A$\spadesuit$).
}{8Hearts.png}


%------------------------------------------------------------------------------------
\card{9r.png}{heart.png}{Supersymmetry}{
We considered already a lot of symmetries (Q$\heartsuit$) in elementary particle physics. Some of them didn't mention explicitly. So let's list them: the symmetry between the particles and anti-particles; gauge symmetry (yes, on the fundamental level it is symmetry too); the symmetry between the generations of quarks; symmetry between different colour charges; symmetry between the different flavours of leptons. Admittedly, there are more symmetries in this highly symmetric branch of physics. Now we'll consider the symmetry between the bosons and fermions.

First, let's remind what bosons and fermions are. By definition, the bosons are the particles -- not necessarily elementary ones (that is they can be composed of other particles) -- with the integer spin. We have already told about spin earlier while speaking about the Bose-Einstein condensate (8$\diamondsuit$). There are non-elementary bosons: all the mesons, the nuclei with the even total number of protons and neutrons etc. And there are fundamental bosons: photon, W$^\pm$ and Z, gluon and Higgs boson. Now we are interested in the fundamental ones. Besides, there are fermions with the half-integer spin. Again, there are composite fermions, like protons, neutrons and all other baryons, nuclei with an odd number of protons and neutrons etc. And there are fundamental fermions: leptons and quarks. The fundamental fermions are the constituents of matter (A$\heartsuit$), and the fundamental bosons are the mediators (2$\heartsuit$).

The symmetry between fermions and bosons, if it exists, is very strongly broken. Remember what happened when the symmetry in the electroweak interaction (4$\heartsuit$) was broken (3$\heartsuit$)? The corresponding W$^\pm$ and Z bosons became very heavy. Similarly, the particles that correspond to this fermion-boson symmetry are way heavier than all the other particles. This symmetry is called supersymmetry.

The hypothetical supersymmetric particles have names with the suffix ``-ino'': for the photon, there is a heavy-fermion particle photino; for the electron, there is a heavy boson electrino etc. All these particles, if they exist, are not discovered yet. Current searches limit the masses of the supersymmetric particles to be about one thousand times larger than the masses of the corresponding ordinary particles.

Supersymmetry offers a natural explanation to some long-standing problems in particle physics. For example, it is not yet understood why the gravity is so weak even relative to the weak force. Also, if the supersymmetric particles exist, the lightest of them would be stable and would not interact with the ordinary matter. Particles with precisely these properties are searched as the dark matter particles -- what is a dark matter we will be speaking later (Q$\spadesuit$).
}{9Hearts.png}


%------------------------------------------------------------------------------------
\card{10r.png}{heart.png}{Grand Unification}{
We managed to get through the elementary particle physics without even mentioning the energy scales. Let's fill this gap now. The units of energy used in this field of physics are called electronvolts. One electronvolt measures a very small amount of energy: for example a tennis ball on the court has an energy of billions of billions of electronvolts. In fact, one electronvolt energy is pretty small even for the elementary particles. In the modern accelerators (J$\heartsuit$) the energy of particles reach billions and even thousand of billions of electronvolts. One billion electronvolts is called giga-electronvolt, or GeV.

Nature, being more powerful than humans, can accelerate particles to the higher energies. For example, the remnants of the supernova explosions (6$\clubsuit$) can accelerate particles up to millions GeV. But we can observe particles with energies reaching a hundred billion GeV (5$\clubsuit$)! Now we'll change the subject a bit, but please keep in mind these numbers.

We considered two types of interactions in particle physics: strong (5$\heartsuit$) and electroweak ones (4$\heartsuit$). The electroweak interaction is interesting because at low energies -- below about one hundred GeV -- it looks like two quite distinct interactions. It means that the electromagnetic interactions appear way more intensive than the weak ones. However, above one hundred GeV they merge, forming an electroweak interaction.

It may seem natural to expect that the electroweak and strong interactions can fuse at some very high energy. Scientists suspect that this is indeed the case. They call the theory that describes this phenomenon the Grand Unified Theory. Such a theory can explain how the fundamental forces emerged from the one unified interaction that existed at the very first moment after the Big Bang (J$\spadesuit$) and clarify, for example, why the protons have the same electric charge as electrons (just with the opposite sign).

But how high the energy should be for the three interactions to merge? The intensity of the fundamental forces changes with energy. So measuring these intensities on the low energies, one can extrapolate the measurements to the high energies and try to find the point where these extrapolated intensities are equal. It turns out that the Grand Unification should happen on the energy about one million billion GeV! It is ten thousand times more than the highest energies ever observed on Earth (in the cosmic rays -- 5$\clubsuit$). It seems impossible that we will be ever able to study such energies. Directly -- probably not. But it is not excluded that we will be able to research this theory indirectly. That is without creating the Unified particles in the accelerator, but observing their signature in virtual particles (6$\heartsuit$) which is allowed by the uncertainty principle (J$\diamondsuit$).

%The primary motivation for the Grand Unified Theory is the fact that the electric charge of a proton is precisely equal to that of the electron (except that they have the opposite sign). The Standard Model (A$\heartsuit$) can give a natural explanation to this. So the Grand Unification is needed to avoid this unnaturalness in particle physics.
}{10Hearts.png}

%--------------------------------------------------------------------------------

\thispagestyle{fancy}
\fancyhf{}
\renewcommand{\headrulewidth}{0pt}
\lhead{\thepage \hskip14pt Physics Is My Favorite Game}
\fancyfoot{}
{\huge{\textbf{Suggested materials}}}
\vskip12pt

Particle physics has one crucial advantage for a non-specialist. The experiments in this field can be only conducted using huge machines -- accelerators, the experimental setups are super-expensive, and the teams of scientists working on each research are large. In these circumstances, if the team put even just a little per cent of their effort to the outreach activities, you get a lot of exciting projects in the result.

One of such projects is, for example, the ATLAS Minecraft map, where you can learn quite a serious physics in the environment of a popular video game.

CERN is today the world-leading centre for the particle physics research, and it will remain in this position for a long time. Check out their Youtube channel, and you'll find a lot of cute videos on their activities.

Don't neglect the TED lectures. They are usually short and straightforward, though scientifically precise. Don't miss Brian Cox's speech on the CERN's supercollider. It will give you the sense of the scales for the experimental setups.

As for the reading, the best pop-book on particle physics I ever held in hands is the ``Particle Physics: A Very Short Introduction'' by Frank Close. However -- and this remark should be taken into account for any book on this subject -- the particle physics is a fast-changing field. This particular book was published in 2004 -- eight years before the discovery of the Higgs boson. Keep in mind that quite often the information given even in excellent writing may be incomplete.

And the best serious introduction to particle physics is, on my opinion, the book of Donald Perkins ``Introduction to High Energy Physics''. It is a kind of classics -- so no surprise this book is pretty old. It was last edited in 2000.

\newpage