\mypart{Cosmology -- Spades}{Cosmology \\Spades}{spade.png}{We step into the realm of cosmology -- the branch of physics that studies the largest possible object, the Universe as the whole. How did it appear? What is its content? What is its ultimate fate? These are the questions we address in cosmology.}

%------------------------------------------------------------------------------------

\card{Kb.png}{spade.png}{Albert Einstein -- Relativity}{The theory of relativity takes the beginning at the end of XIX -- the beginning of XX century when physicists experimentally established a bizarre fact: the speed of light depends neither on the speed of the light source nor on the speed of the observer. It sounds weird from the everyday experience: if you drive on a highway with the speed of 120 km$/$h and someone is overtaking you, having speed 130 km$/$h, then his speed relative you will be only 10 km$/$h. But for the light it is not so -- the light still moves with the same speed of about $3\times10^8$m$/$s, no matter how fast you fly.

But it means that the light is always moving this fast, and there is no observer, who would be able to study a photon at rest. It means that nobody that has mass can move with the speed of light. If you apply these conclusions to the mathematical formulae, you inevitably get quite funny results that the time and space can contract and stretch, depending on one's speed {\it relative} the observer. This is the foundation of the special theory of relativity.

If you add the acceleration into the scope, you get even more odd results. Imagine you are in a space rocket. Its engine is on, and it is accelerating very fast. Imagine then that you turn on your flashlight and shine across the rocket. Let's say you hold the flashlight at one meter from the floor. When the light reaches the wall of the rocket, it will hit it slightly lower than one meter, because the rocket is accelerating. So in the {\it reference frame of rocket}, you would observe that the light flew on the curved trajectory.

The general theory of relativity postulates that the gravitational field is equivalent to the situation with the accelerating rocket. And so the light beam gets curved next to the massive bodies. Physicists prefer speaking about the light going on straight lines in a curved space instead. This way, you can introduce the notion of the curved space-time. Its curvature is created by the distribution of matter in the Universe, and it manifests itself by what we call the force of gravity. The general theory of relativity, designed by Albert Einstein, is the standard theory of gravity. And since gravity is the only force that acts on the cosmological distances, this is the base theory of cosmology.

Einstein published his general relativity in 1915, and in 1919 it was confirmed experimentally: during the solar eclipse, the stars that would be normally hidden behind the Sun were still visible thanks to the curvature of space around the Sun. Today, using the large modern telescopes, we can observe even more spectacular confirmations of this theory. Just try searching on the internet the images of Einstein rings.
}{KSpades.png}

%------------------------------------------------------------------------------------

\card{2b.png}{spade.png}{Redshift}{We said that Hubble (K$\spadesuit$) has proved that the Milky Way does not limit the size of the Universe. If you remember, I promised to tell the continuation of that story.

Later, in 1927 – 1929, Edwin Hubble discovered, that the galaxies don’t stay still, but move away from each other. Using spectroscopic data, he deduced the famous law that bears his name: the farther the galaxy, the higher is its receding speed from us.

By the spectroscopic data, we mean the following: the hydrogen atoms in a distant galaxy absorb the light at some well-defined frequencies (4$\diamondsuit$). So if you measure the spectrum of that galaxy, you see some narrow gaps. Hubble observed the spectra of many galaxies and saw everywhere the same pattern of gaps. But oddly the spectra of different galaxies seemed to be shifted relative to each other. Most of them appeared to be redder than they should be. This effect is quantified in the measurement called {\it redshift}. Remember that Hubble knew how to measure the distance using the cepheids (see K$\spadesuit$ again). So he just plotted the distance versus the redshift and discovered, that the farther the galaxy is, the more it is redshifted. Today we know that the redshift is caused by the expansion of the Universe.

How does the light turn red when the Universe expands? The expansion of the Universe in terms of the general relativity (K$\spadesuit$) is the stretching of its space. The photon has a wavelength (K$\diamondsuit$), which is also stretching together with space it is passing. But the stretched light is the red light! The light needs some time to travel from a distant galaxy to us. During this time the Universe gets stretched, and so the light becomes redder. This can be different for the galaxies next to us. For example, the already mentioned Andromeda galaxy is actually blueshifted, because it is moving towards us. But this is because of the local interactions. On the global scale, the Universe is expanding, and it is seen as the redshift of distant objects.
}{2Spades.png}

%------------------------------------------------------------------------------------

\card{Jb.png}{spade.png}{Lema\^{i}tre -- Big Bang}{In 1927 Belgian astronomer and Catholic priest Georges Lema\^{i}tre learned about the Hubble's result and gave his own explanation to the global Universe expansion. He built a model of changing of the space curvature (K$\spadesuit$) radius with time. Actually, he was the first who wrote the Hubble law, mentioned a page ago, and he also made the first estimation of the Hubble constant -- the coefficient of proportionality between the distance to the object and its receding speed. He proposed an interesting idea, that as the Universe now expands, maybe before it was just a point-size. He called this ``hypothesis of the primaeval atom'' or the ``Cosmic Egg''. Less poetically physicists started calling this model as a model of ``hot Universe'', meaning that it was in a hot and dense state at the beginning. However, the correct -- and a bit boring -- term of hot Universe was later replaced by the name ``Big Bang''. This term was first invented by an astrophysicist Fred Hoyle. Ironically, he was a strong opponent of the hot Universe idea -- and yet today we use this naming.

When people hear about the Big Bang they usually imagine a kind of a global explosion. It is not correct. Our Universe has three space dimensions -- and for us, it is hard to understand the expansion of the three-dimension space. Let's simplify the situation. Imagine an infinite rubber band -- this will be our one-dimensional Universe. Then let's draw a mark on every meter of the band. Imagine now that you start stretching this elastic band. This is the expansion of the Universe. By the way, this is a nice illustration of the Hubble law too. If the receding speed for the marks which are 1 meter away from each other is 1 cm per second, then it is easy to realise that the marks which are 2 meters away will recede from each other by 2 cm per second -- twice faster, just as stated by the Hubble law. Our galaxy Milky Way sits on one of these ticks. And for us, every point of the Universe is receding from us. Moreover, it is valid for any observer in this infinite Universe. So we better stop speaking about the galaxies that fly away from us. The righter picture is the expanding space between the galaxies.

The situation with the stretching rubber band was the starting point for Lema\^{i}tre, with the only difference that he worked in three dimensions. Now let's repeat his way of thinking. If we are observing the Universe which is stretching now, it is quite logical to try extrapolating this situation back in time -- in this case, you'll see how the elastic band is contracting as you approach the point of zero time. What is that zero time then? Obviously, it is when the rubber band is contracted completely, down to the infinite density. Does this infinite density state has any physical meaning or not -- you'll learn it in the very end of this book (10$\spadesuit$).
%Of course, for Lema\^{i}tre the idea of the Cosmic Egg had a very strong religious sense.
}{JSpades.png}

%------------------------------------------------------------------------------------

\card{3b.png}{spade.png}{Cosmic microwave background}{
Right after the moment of the Big Bang, the Universe was in a hot and dense state. With time the Universe expanded became less dense, and the temperature has dropped. Why does the temperature drop down? For the same reason as the photons become redshifted: if, for example, in some volume, you had 1000 blueish hot photons, after the expansion of the Universe by factor 10 in size, you would get only ten photons in the same volume, and they will be redshifted by a factor of 10.

Cosmologists computed that after the Big Bang, the Universe was full of the hot plasma. The protons and electrons were flying freely. And when eventually they collided and formed the hydrogen atoms, these newly created atoms were immediately disassembled by the high energy photons that also flew around. This early ionised Universe was opaque to the photons: they could not travel far without colliding with anything. But when the Universe was about 400 thousand years old (very young age, comparing to the total age of the Universe today, which is almost 14 billion years), the temperature of these photons dropped below the point when they could maintain the plasma ionised. The protons combined with electrons, the plasma died away, and the photons were set free. These ancient photons we observe today as the cosmic microwave background.

CMB was discovered by mere chance in 1964 by A. Penzias and R. Wilson from the Crawford Hill Laboratory. The antenna they were using was very sensitive, with a very low noise level. But they found the registered noise exceeded the noise observed in the laboratory. At first, they supposed that this noise was coming from the Earth. But funny enough, they got the same results, completely independent of the antenna orientation. Maybe it was the signal coming from the galaxy? They pointed the antenna on the dark part of the sky outside the Milky Way. Still, the observed signal was substantial. After some study and extensive discussions with colleagues, they understood that they found the relic radiation from the Big Bang.

The cosmic microwave background radiation is one of the most critical evidence for the theory of the hot Universe and takes an outstanding role in modern cosmology. It is a source of extensive information about the early stages of the Universe history. This background has the spectrum of a black body (K$\diamondsuit$) with a temperature of $2.7$ K, or about $-270^\circ$C, and it follows the theoretical spectrum with extremely high precision. It is the most precise measurement in the whole cosmology.
}{3Spades.png}

%------------------------------------------------------------------------------------

\card{4b.png}{spade.png}{Cosmic web}{
The basic cosmological principle says that on the large scales the Universe becomes homogeneous (has equal density everywhere) and isotropic (looks the same in every direction). Indeed under the assumption of the homogeneous Universe, scientists have predicted the Big Bang, the cosmic microwave background, calculated the age of the Universe etc. However, the very book you are holding in your hands is proof that the cosmological principle is wrong on the small scales. The book is dense, while the air around it is sparse, so on the small scales, the Universe is obviously inhomogeneous.

It turns out that the Universe starts being homogeneous only on the very large scales, more than a half-billion light-years. Going from small sizes to large ones, one can build a ladder of scales. Star clusters have a size of about one light-year. Galaxies have diameters on the order of a hundred thousand light-years. The galaxies, on their turn, form clusters of galaxies (10 million light-years) and super-clusters of galaxies (100 million light-years). Now, this is the size when we start seeing the web structure of the Universe: between the galaxy clusters we see the gigantic filaments, filled with galaxies and extragalactic gas. The empty spaces, known as the cosmic voids, are observed between the clusters and filaments. The hierarchical structure of the Universe was produced by gravitational instability of some random perturbations of density very early in the history of the Universe (9$\spadesuit$).

The are many fascinating methods to study the large-scale structure of the Universe. One can observe the supernova explosions of the white dwarf (6$\clubsuit$). They are incredibly bright, so they are visible from far away. And one can measure the distance to them, thanks to the known brightness of this kind of source.

Another method involves the study of quasars (8$\clubsuit$). They are even brighter than the supernovae so that one can measure even farther distances with them.

But most of the large-scale structure studies are done with just the measurement of the redshift (2$\spadesuit$). The redshift is not a very reliable measurement of the distance, because it suffers from different distortions. For example, if the observed galaxy is moving along the line of sight it will become additionally red- or blueshifted (if it moves away or towards us correspondingly), so it will seem more or less distant than it is in reality. However, scientists have methods to take these distortions into account and to reconstruct the global picture.
}{4Spades.png}

%------------------------------------------------------------------------------------

\card{5b.png}{spade.png}{Element abundances}{
From the very beginning of the Big Bang theory, its success was based mainly on the excellent agreement of predictions with observations for the abundances of the light elements in the Universe. Quite an illustrative fact for this is that the Nobel lecture of Arno A. Penzias is called ``The Origin of the Elements''. However, he was awarded the Nobel prize for the discovery of the cosmic microwave background (3$\spadesuit$). At that time physicists thought that all the elements were synthesised in the hot plasma at the beginning of the Universe evolution (precisely, in his Nobel lecture Penzias analyses the history of views on the element formation and marks that the opinion on this issue changed several times between the 1930s and 1970s). It was found out later, that only light elements could form in primordial plasma, and the heavier elements appear much later during the evolution of the stars.

The typical explanation of primordial nucleosynthesis (formation of nuclei) is the following: as the Universe cooled down when the energy of photons dropped below the binding energy of some nuclei, the photons did no longer break these nuclei to protons, and the element began to form. This mechanism is called {\it freezing} -- it is interesting to note that in Russian literature it is called with the same word as used for the steel hardening as if the Universe is a colossal blacksmith's shop. The freezing of the lithium nuclei happened about a couple of minutes after the Big Bang. The lighter elements got formed even earlier.

The average number of different elements in the modern Universe is a well observable variable. In this study, we are interested in the lightest elements: isotopes of hydrogen and helium and lithium. All these atoms could be found in the rare interstellar gas. The most abundant hydrogen atoms have a density of about one atom per cubic centimetre inside the galaxy, but there is much less hydrogen in the intergalactic space. Helium is about four times less abundant. The deuterium -- an atom that has one proton and one neutron in its nucleus -- and the isotope of helium $^3$He are more than ten thousand times less common than hydrogen. And lithium is so rare that there is only one atom of lithium per more than a billion atoms of hydrogen. So this is a measurement that spans nine orders of magnitude. And the most impressive is that all these measurements are in the perfect agreement with the prediction from the Big Bang nucleosynthesis!
}{5Spades.png}

%------------------------------------------------------------------------------------

\card{Qb.png}{spade.png}{Vera Rubin -- Dark matter}{
In 1932 Fritz Zwicky noticed, that besides the luminous baryonic matter of galaxies there are invisible, hidden masses in the Universe that manifest themselves only through the gravitation. Zwicky studied the galaxy cluster in the constellation of Berenice's Hair. And he discovered that the speeds of the galaxies in this cluster are tremendous, up to few thousand kilometres per second. To hold down such fast-moving galaxies within the cluster the vast gravitational force is needed, much higher than the gravitational force from the galaxies themselves. Later, in 1970s Vera Rubin discovered that the hidden masses present not only in the clusters of galaxies but in the isolated galaxies as well. Invisible dark matter forms spherical halos around galaxies. The radius of a halo is typically 5-10 times bigger than the radius of the galaxy. The discovery of Rubin was genuinely ground-breaking and has introduced the dark matter to the standard model of cosmology.

There are more phenomena through which the dark matter can manifest itself. For example, the galaxy clusters are filled with gas. Its high temperature can be explained only by taking into account the dark matter component of the cluster. Or another example: the mass of a galaxy cluster can be independently estimated by the effect of gravitational lensing when the light of a background galaxy gets curved by the cluster. Again, its total mass turns out to be much larger than the mass of the visible matter. Remarkably, all the different observations give the same estimation to the relative amount of dark matter. It is about five times more abundant than the normal (called {\it baryonic}) matter.

Some scientists try to explain the effects listed above by the hypothetical difference of the gravity law on big distances from that on smaller scales. But so far no modification of the theory of gravitation could not explain the observations thoroughly.

The most commonly accepted model for the dark matter says that it consists of weak interacting massive particles (WIMP). If this model is true, then many of these particles are crossing the Solar System, Earth and you without any interaction. Many experiments around the globe are trying to detect the signal of such particles, but yet unsuccessfully. Some WIMP models built on the idea of the supersymmetry (9$\heartsuit$).
}{QSpades.png}

%------------------------------------------------------------------------------------

\card{6b.png}{spade.png}{CMB anisotropies}{
The cosmic microwave background (3$\spadesuit$) is exceptionally uniform. No matter where you would point your microwave antenna, you will always measure the same temperature of about 2.7K. Still, there are tiny variations of its temperature, on the order of a milli-Kelvin. Imagine a one-meter thick brick wall. And this wall is so smooth that the most considerable bumps and deeps on it measure only one-tenth of a millimetre. I guess you would say that this wall is very well polished. So is the cosmic microwave background.

These tiny fluctuations of temperature -- or they also say the {\it anisotropies} -- take place from the non-uniformity of the matter distribution in the early epochs of the Universe expansion. These are the same non-uniformities that later formed the large-scale structure of the Universe (4$\spadesuit$). The regions where the cosmic microwave background is slightly colder correspond to the areas, where the plasma of the early Universe was denser: the light spent some energy to get out of the gravitational potential of this over-density and thus became a bit colder. It means that we can directly study the distribution of matter in the early stages after the Big Bang, and most interesting is that we can do it with high precision. The cosmic microwave background (CMB) anisotropies can tell a lot about the content of the Universe. How much baryonic and dark matter there is? How curved on the global scale is space? How old is the Universe? And many other critical cosmological questions could be answered in this study. Current experiments have established these parameters with a precision below 1\%. But still, there are questions without an answer. One of the most intriguing ones is the amplitude of the so-called {\it primordial B-modes} of the CMB polarisation. It is a weak signal that was imprinted on the CMB by the gravitational waves (10$\clubsuit$) which were created in the epoch of inflation (9$\spadesuit$) right after the Big Bang. Many experiments tried to measure it, but yet unsuccessfully. One of the strongest effects that make this task extremely difficult is the light polarisation from the elongated dust grains in the magnetic field of the Milky Way (7$\clubsuit$).

The different structures that appeared later in the Universe make the foreground for the CMB, and they get imprinted on it. For example, the photons of CMB get lensed by the presence of the cosmic web -- so the latter can be studied through the lensing of the CMB. Another example: when a CMB photon passes through a massive galaxy cluster, it can be heated up by the collision with the hot gas. It is the so-called Sunyaev-Zeldovich effect. It is a powerful tool to study the cluster dynamics.
}{6Spades.png}

%------------------------------------------------------------------------------------

\card{7b.png}{spade.png}{Reionization}{
When the early hot plasma cooled down, and the protons finally recombined with the electrons (3$\spadesuit$), the Universe finally became transparent. And all of a sudden, instead of being full of fire, it became dull and empty. The so-called dark ages of the Universe began. Nothing happened in this time, and only the rare hydrogen atoms were flying around for several dozens of million years.

All this time, the dark matter (Q$\spadesuit$) was continuing to form its dark structures. And the hydrogen followed the gravitational attraction of these structures, creating the clouds. Eventually, these clouds became dense enough to make possible the birth of the first stars (4$\clubsuit$). The stars were quite different from those that we observe today because they were made of the hydrogen and a bit of helium -- but virtually nothing else. In these circumstances, the stars turned out to be much bigger, hotter and brighter than the present ones. Their hard X-ray radiation re-ionized the hydrogen atoms around. This is the epoch of reionization. The hydrogen atoms absorbed the light of the first stars, mainly on the wavelength of 21 cm (4$\diamondsuit$). Thanks to this today we can study this effect in the radio spectrum.

The fusion reaction (10$\diamondsuit$) in the massive early stars was going on an increased rate, creating the heavy elements. When they were exploding in supernova explosions (6$\clubsuit$) they were spreading this material around, setting the environment for the next generations of stars, planets and everything else, like for example this deck of cards.

The early power supernova explosions created the flows in the cold scenery of the dark ages. These flows started clumping up in the already existent dark matter structures. By the end of the day, this material formed the galaxies we see today. The Universe turned from the violent chaos to a beautiful starry world, where several billion years later, on the little rocky planet Earth the life has appeared.
}{7Spades.png}

%------------------------------------------------------------------------------------

\card{8b.png}{spade.png}{Dark energy}{
The observations of the supernova explosions of white dwarfs (6$\clubsuit$) show that today the Universe expands with the growing rate. This fact could be explained by the presence of some form of energy. However, since nobody has yet succeeded to detect it directly, we call it dark energy. In 2011 the Nobel Prize in physics was awarded to Saul Perlmutter, Brian P. Schmidt and Adam G. Riess for their leadership in the discovery of the accelerated expansion of the Universe and hence the existence of the dark energy.

By the way, it is not late yet to clarify one interesting point. The speed of light is finite -- and this gives the cosmologists the superpowers of seeing back in time! Indeed, we can only observe the light that was emitted some time ago. Even for the closest cosmological sources, this time is millions of years. For the furthest ones it reaches almost 14 billion years -- this is true for the CMB light. It means that to study the Universe at earlier stages of its evolution, one has to simply look farther. Or, speaking more precisely, analyse the data with the larger redshift (2$\spadesuit$). Perlmutter, Schmidt and Riess did precisely this. They studied the different slices of the Universe on different redshifts. And they were measuring the change of its expansion rate, which turned out to be increasing in the last $\sim 5$ billion years.

There are at least two more pieces of evidence for the presence of the dark energy. The growth of galaxy clusters is defined by two counteracting processes: gravitational contraction and repulsion due to the dark energy. We can measure the dependence of the galaxy cluster masses on the redshift to get its evolution with time. According to this study, approximately 70\% of the total energy-density budget of the Universe constitutes of dark energy. This result is conforming with the measurement through the expansion rate as described above.

Another independent evidence for the presence of the dark energy could be given by the gravitational lensing of the CMB (6$\spadesuit$).

Since the Universe is now expanding with acceleration, you may ask yourself: what would be with our Universe in the future. The answer to this question is still not clear, and it entirely depends on the intrinsic nature of the dark energy. Some models predict that the Universe will continue accelerating and at some point, all the structures (even the atoms) will be destroyed by the ripping force of the dark energy. Other models predict the bouncing Universe, which will eventually return to the state it had at the moment of the Big Bang and the story will start over. Anyway, all these scenarios require a very long time, many orders of magnitude longer than the current age of the Universe.
}{8Spades.png}

%------------------------------------------------------------------------------------

\card{Ab.png}{spade.png}{Observable Universe}{
It is time to summarise what we have learned so far. About 13.8 billion years ago, the whole Universe was in an extremely hot dense and state. Since then, it was expanding and cooling (J$\spadesuit$). In the first three minutes after the beginning of the expansion (or the Big Bang), the quarks formed the protons and neutrons, which on their turn produced (5$\spadesuit$) the observed today hydrogen, helium and lithium nuclei (the latter two in small quantities). About 400 thousand years after the Big Bang, the primordial plasma died away, and the Universe became transparent. The afterglow of that plasma is observed today as the cosmic microwave background, which is still the dominant form of radiation in the Universe (3$\spadesuit$). The baryonic matter, under the gravity of the dark matter, started clumping, creating gradually growing structures, from galaxies to the cosmic web (4$\spadesuit$). About 5 billion years ago, the matter became so sparse that the dark energy became dominant in the energy budget of the Universe, and the expansion rate started growing (8$\spadesuit$). Today the Universe has zero global curvature (K$\spadesuit$), and it consists on about 70\% of dark energy, 25\% of dark matter and the rest 5\% of the ordinary baryonic matter. This is the standard model of cosmology. More rigorously it is called the $\Lambda$CDM cosmology, where $\Lambda$ (capital greek letter lambda) is the physical term for the dark energy and CDM stands for the Cold Dark Matter.

Thanks to the finite speed of light, we can observe different epochs of the Universe to varying distances from us. The distant objects become redshifted because space gets expanded while the light was travelling from those objects to us (2$\spadesuit$). The most distant object that we can observe is the spherical surface at which the CMB photons last time interacted with the primordial plasma. This is the observable Universe. It is correct to say that we are sitting in its centre. But this is not because of the preferred position of the Earth. Actually, any point in the Universe can be designated to be its centre.

Formally the observable Universe spans even farther. Imagine that at the moment of the Big Bang, a photon was emitted from the point where the Earth is now. And that this photon travelled for 13.8 billion years without getting scattered on anything. Due to the Universe expansion, it would be now not 13.8, but about 46 billion light-years from Earth. This is the radius of what they call the observable Universe -- even though in reality you can't see farther than the CMB sphere.
}{ASpades.png}

%------------------------------------------------------------------------------------

\card{9b.png}{spade.png}{Inflation}{
The $\Lambda$CDM model, described on the previous page (A$\spadesuit$) is not just a beautiful theory -- it allows doing precise measurements of various cosmological parameters in many different independent ways and getting consistent results. Still, there are some open questions. Why don't we observe the antimatter (K$\heartsuit$)? Why the Universe has no curvature on the global scale (K$\spadesuit$)? And why the CMB is so uniform (3$\spadesuit$)? 

Let's consider the last question in more details. One can define the cosmological horizon: it is the maximal distance at which two objects may have influenced each other since the Big Bang (J$\spadesuit$). If a point is outside the horizon of another location, they should normally look completely different. The problem is that the region we see today as CMB is much larger than the horizon at the time when CMB was released. That is a CMB photon that comes from one side of the sky ``knows'' nothing about the CMB photon from another side of the sky. Hence, it is natural to expect that they would have a completely different temperature. And still, they are very much the same. This situation is as strange as if I would travel to the Easter Island (I've never been there) and would find that the islanders know me and I know them.

However, the situation would be less weird if we assume that me and the Easter Island residents left together some time ago and then one day moved -- they to their island and me to Europe. Cosmologists hypothesise that something similar happened to the Universe. Right after the Big Bang, before the Universe became $10^{-34}$ second old, it expanded on the accelerated rate and every tiny region of it suddenly became much larger than the whole observed Universe today. When you take a non-uniform Universe and blow up a tiny piece of it, naturally it becomes uniform. So, no surprise, we observe a uniform CMB today. This theory is called the theory of inflation. It explains the curvature of the Universe too: even if in the beginning the space was curved, it flattened out during the inflation period. And if just by chance it happened that this tiny little piece of the Universe before inflation had a bit more matter than antimatter, than this disbalance will maintain in the mature Universe.

If the inflation took place, it had to create a huge gravitational wave, which after 400 thousand years had to leave an imprint on the polarisation of the CMB. This imprint should have a specific curled pattern called  B-modes. Many CMB experiments are chasing it, but this weak signal suffers from strong foregrounds. One, as we said, comes from the dust grains in the Milky Way. Another comes from the distortion of the CMB signal by the large-scale structures (4$\spadesuit$). Should be said that the theory of inflation doesn't make a part of the $\Lambda$CDM cosmology.  Still, not many cosmologists doubt that it did happen.
}{9Spades.png}

%------------------------------------------------------------------------------------

\card{10b.png}{spade.png}{Planck epoch}{
The whole Universe is described by the combination of quantum mechanics (A$\diamondsuit$) that acts on the small scales and the general relativity (K$\spadesuit$) that works on the large scales. The quantum effects become unnoticeable even on the millimetric scale. And the gravity effects usually start to manifest themselves on the much larger distances. So these two realms never coincide.

The main fundamental constant that rules in quantum mechanics are the Planck constant $h$ (K$\diamondsuit$). And the main one in general relativity is the gravitational constant $G$ that gauges the strength of the gravity. If we add here the speed of light, then from these three, it becomes possible to construct the so-called Planck units: Planck length and Planck time. These units have a fascinating meaning. On the distances smaller than the Planck length and the time scale shorter than the Planck time, one can neglect neither general relativity nor the quantum mechanics. Instead, one has to use some mixture of those two. But the corresponding theory is not yet invented so for the moment we {\it don't know} what happens on the Planck scales. (However, we have some ideas. Take a look at the jokers.)

The Planck length is equal to $1.6\times10^{-35}$m and the Planck time is $5.4\times10^{-44}$s. As you can see, these are extremely short distances and an extremely brief time. However, we can still speak about the time when the Universe was younger than the Planck time and had a size smaller than the Planck length. That is the Planck epoch.

Very often people think about the Big Bang like an actual explosion that happened at the moment zero when all the distances were zero (virtually the Universe was just a point) and the density was infinite. In fact, we know nothing about the Universe before the end of the Planck epoch. We don't even have a theory to try to approach it. Moreover, we don't even know what are the space and time at the Planck scales. It was probably a crazy soup of multidimensions. The Big Bang is the model of expansion of the Universe from some hot and dense state {\it after} the Planck era. It is equally pointless to ask physicists about what was before the moment zero. Some speculate that the Universe existed before and it shrank down. Others think that the Universe was created from the quantum fluctuations of vacuum (6$\heartsuit$). But the most honest answer about the Universe before the Big Bang --  we don't know anything about it.
}{10Spades.png}

%--------------------------------------------------------------------------------

\thispagestyle{fancy}
\fancyhf{}
\renewcommand{\headrulewidth}{0pt}
\lhead{\thepage \hskip14pt Physics Is My Favorite Game}
\fancyfoot{}
{\huge{\textbf{Suggested materials}}}
\vskip12pt
For other parts of this book, I was suggesting non-reading materials first. But here I cannot avoid recommending reading the ``Brief History of Time'' of Stephen Hawking.  Another excellent book is ``The First Three Minutes'' of Steven Weinberg. It will show you all the beautiful connections between cosmology and particle physics. But if you aim to become a prominent specialist in the field, start with the ``Modern Cosmology'' by Scott Dodelson.

For the non-printed resources, cosmology very much coincides with the astrophysics (in fact, not everybody makes a distinction between the two). So here my recommendations would be the same: www.universetoday.com, ESO's Twitter... Don't forget the phys.org news website. The fields of cosmology and astrophysics go there under the same tab: ``Astronomy and Space''.

There are very exotic notions in cosmology, which I didn't dare mention. You would be interested to learn about the cosmic strings, monopoles, white holes and multiverse. I recommend the Fermilab channel on Youtube with a bunch of brilliant lectures.
\newpage

\mypart{Theories of everything -- Jokers}{Theories of everything \\Jokers}{}{The theories of everything, as it follows from the title, try to explain all the different interactions with one single idea, with the same approach. There are two leading candidates to be such a theory: the string theory and the loop quantum gravity. However, both of them are yet incomplete. Physicists are trying to solve the perplexing mathematics of these theories, but still, there are a lot of open questions.}
%------------------------------------------------------------------------------------

\joker{String theory}{
We mentioned the lack (10$\spadesuit$) of knowledge about the concordance between the quantum mechanics (A$\diamondsuit$) and the general relativity (K$\spadesuit$). Indeed, these two types of interaction look entirely different. The elementary particles interact by exchanging the intermediate bosons (2$\heartsuit$), while the gravitational interaction happens through the curvature of the space-time.

The string theory approach tries to apply the idea of the curved space to the elementary particle interactions. But wait for a second, the curved space creates the force of gravity, right? And if you curve it again, you will get nothing but a bit modified gravity. However, other forces act very differently from the gravitation. For example, the electromagnetic force can attract and repulse depending on the interacting particle charge, while gravity implies attraction only.

To solve this problem, physicists came with a very bright idea to introduce new hidden dimensions. While the curvature of the ordinary space-time causes gravity, the curvature of the hidden dimensions creates other forces.  Theodor Kaluza tried to add one additional dimension to the Einstein's general relativity and -- what a miracle! -- the electromagnetic force just popped out. In order to reduce that redundant dimension  Oskar Klein proposed to {\it compactify} it. Imagine an ant walking along a wire. For the ant the wire is two dimensional: it can go around and along it. But for us, who look at the scene from far, there is only one dimension along the wire, and we are not aware of the additional compactified one. Same way, the extra dimension in the Kaluza-Klein theory is compactified, creating tiny loopy strings on the microscopic scale of the space-time fabric.

Later physicists discovered the existence of two more fundamental forces, the weak (4$\heartsuit$) and strong ones (5$\heartsuit$). To add them to the string theory, they had to add even more dimensions. And that's were the mathematical nightmare starts. Instead of simple loops, space start being full of 21-dimensional superstrings. The vibrations of these complex objects create the elementary particles. Different vibration frequency corresponds to different particles, so according to this theory, the whole world is nothing but the vibrations of tiny superstring orchestra. That is a true spirit for a theory of everything: reducing the whole variety of Nature to the very few fundamental notions, which are the same for the pages of this book, for the black holes and the entire Universe.
}{JokerBlack.png}%------------------------------------------------------------------------------------

\joker{Loop quantum gravity}{
There is another possibility to build a theory of everything, competing with the string theory. While the last one tries to apply the curvature to the particle physics, the quantum gravity approach goes the opposite direction of using the quantum ideas to the gravity. Simplistically, you postulate the existence of a massless boson -- graviton, so any pair of massive particles can interact through it just like by other interactions: weak, strong and electromagnetic.

But the reality is much more complicated than that. To build the complete theory of quantum gravity, you need to quantise not just the gravitational interaction, but space and time too! The quantisation length is equal to the Planck length and the quantisation time is equal to the Planck time, about $10^{-35}$m and $10^{-44}$s correspondingly. So the space in this theory becomes like pixels on your computer screen: one pixel can be black or white, the next one can be red or blue, and there is nothing in between of them. And similarly, the time, which turns to be not continuous. There are brief stops in time, we are jumping from one moment to another, and there is nothing in between. These bits of space-time are linked to each other, and these links are called loops.

If the space-time becomes quantised, you can apply the formalism of wave-functions to it (3$\diamondsuit$). So instead of moving in a well-defined space-time, you start moving in a fuzzy quantum space-time, which also gets curved by the presence of particles. At this moment, the mathematics of this theory becomes so fierce that it is still far from completeness. However, scientists hope to solve it one day.

Again, I would point you to the Fermilab Youtube channel, where you will find several lectures exactly on the subjects of the string theory and loop quantum gravity. And to complete the impression that this science is crazy difficult look for the ``Bohemian Gravity'' video of the A Capella Science.
}{JokerRed.png}

